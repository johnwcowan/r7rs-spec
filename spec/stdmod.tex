\chapter{Standard Libraries}
\label{stdlibraries}

%% Note, this is used to generate stdmod.tex.  The bindings could be
%% extracted automatically from the document, but this lets us choose
%% the ordering and optionally format manually where needed.

This section lists the exports provided by the standard libraries.  The
libraries are factored so as to separate features which might not be
supported by all implementations, or which might be expensive to load.

The {\cf scheme} library prefix is used for all standard libraries, and
is reserved for use by future standards.

\textbf{Base Library}

The \texttt{(scheme base)} library exports many of the procedures and
syntax bindings that are traditionally associated with Scheme.

\begin{scheme}
{\cf *}               {\cf +}               {\cf -}
{\cf ...}             {\cf /}               {\cf <}
{\cf <=}              {\cf =}               {\cf =>}
{\cf >}               {\cf >=}              {\cf \_}
{\cf abs}             {\cf and}             {\cf append}
{\cf apply}           {\cf assoc}           {\cf assq}
{\cf assv}            {\cf begin}           {\cf binary-port?}
{\cf boolean=?\ }      {\cf boolean?\ }       {\cf bytevector-copy}
{\cf bytevector-copy!}
{\cf bytevector-copy-partial}
{\cf bytevector-copy-partial!}
{\cf bytevector-length}
{\cf bytevector-u8-ref}
{\cf bytevector-u8-set!}               {\cf bytevector?}
{\cf caaar}           {\cf caadr}           {\cf caar}
{\cf caar}            {\cf cadar}           {\cf caddr}
{\cf cadr}            {\cf cadr}
{\cf call-with-current-continuation}   {\cf call-with-port}
{\cf call-with-values}                 {\cf call/cc}
{\cf car}             {\cf case}            {\cf cdaar}
{\cf cdadr}           {\cf cdar}            {\cf cddar}
{\cf cdddr}           {\cf cddr}            {\cf cdr}
{\cf ceiling}         {\cf char->integer}   {\cf char-ready?}
{\cf char<=?\ }        {\cf char<?\ }         {\cf char=?}
{\cf char>=?\ }        {\cf char>?\ }         {\cf char?}
{\cf close-input-port}
{\cf close-output-port}                {\cf close-port}
{\cf complex?\ }       {\cf cond}            {\cf cond-expand}
{\cf cons}            {\cf current-error-port}
{\cf current-input-port}
{\cf current-output-port}              {\cf define}
{\cf define-record-type}               {\cf define-syntax}
{\cf define-values}   {\cf denominator}     {\cf do}
{\cf dynamic-wind}    {\cf else}            {\cf eof-object?}
{\cf eq?\ }            {\cf equal?\ }         {\cf eqv?}
{\cf error}           {\cf error-object-irritants}
{\cf error-object-message}             {\cf error-object?}
{\cf even?\ }          {\cf exact}
{\cf exact-integer-sqrt}               {\cf exact-integer?}
{\cf exact?\ }         {\cf expt}            {\cf floor}
{\cf flush-output-port}                {\cf for-each}
{\cf gcd}             {\cf get-output-bytevector}
{\cf get-output-string}                {\cf guard}
{\cf if}              {\cf import}          {\cf inexact}
{\cf inexact?\ }       {\cf input-port?\ }    {\cf integer->char}
{\cf integer?\ }       {\cf lambda}          {\cf lcm}
{\cf length}          {\cf let}             {\cf let*}
{\cf let*-values}     {\cf let-syntax}      {\cf let-values}
{\cf letrec}          {\cf letrec*}         {\cf letrec-syntax}
{\cf list}            {\cf list->string}    {\cf list->vector}
{\cf list-copy}       {\cf list-ref}        {\cf list-set!}
{\cf list-tail}       {\cf list?\ }          {\cf make-bytevector}
{\cf make-list}       {\cf make-parameter}  {\cf make-string}
{\cf make-vector}     {\cf map}             {\cf max}
{\cf member}          {\cf memq}            {\cf memv}
{\cf min}             {\cf modulo}          {\cf negative?}
{\cf newline}         {\cf not}             {\cf null?}
{\cf number->string}  {\cf number?\ }        {\cf numerator}
{\cf odd?\ }           {\cf open-input-bytevector}
{\cf open-input-string}
{\cf open-output-bytevector}
{\cf open-output-string}               {\cf or}
{\cf output-port?\ }   {\cf pair?\ }          {\cf parameterize}
{\cf peek-char}       {\cf peek-u8}         {\cf port-open?}
{\cf port?\ }          {\cf positive?\ }      {\cf procedure?}
{\cf quasiquote}      {\cf quote}           {\cf quotient}
{\cf raise}           {\cf raise-continuable}
{\cf rational?\ }      {\cf rationalize}     {\cf read-bytevector}
{\cf read-bytevector!}                 {\cf read-char}
{\cf read-line}       {\cf read-u8}         {\cf real?}
{\cf remainder}       {\cf reverse}         {\cf round}
{\cf set!}            {\cf set-car!}        {\cf set-cdr!}
{\cf string}          {\cf string->list}    {\cf string->number}
{\cf string->symbol}  {\cf string->utf8}    {\cf string->vector}
{\cf string-append}   {\cf string-copy}     {\cf string-fill!}
{\cf string-for-each} {\cf string-length}   {\cf string-map}
{\cf string-ref}      {\cf string-set!}     {\cf string<=?}
{\cf string<?\ }       {\cf string=?\ }       {\cf string>=?}
{\cf string>?\ }       {\cf string?\ }        {\cf substring}
{\cf symbol->string}  {\cf symbol=?\ }       {\cf symbol?}
{\cf syntax-error}    {\cf syntax-rules}    {\cf textual-port?}
{\cf truncate}        {\cf u8-ready?\ }      {\cf unless}
{\cf unquote}         {\cf unquote-splicing}
{\cf utf8->string}    {\cf values}          {\cf vector}
{\cf vector->list}    {\cf vector->string}  {\cf vector-copy}
{\cf vector-fill!}    {\cf vector-for-each} {\cf vector-length}
{\cf vector-map}      {\cf vector-ref}      {\cf vector-set!}
{\cf vector?\ }        {\cf when}
{\cf with-exception-handler}
{\cf write-bytevector}                 {\cf write-char}
{\cf write-partial-bytevector}         {\cf write-u8}
{\cf zero?}
\end{scheme}

\textbf{Inexact Library}

The \texttt{(scheme inexact)} library exports procedures which are
typically only useful with inexact values.

\begin{scheme}
{\cf acos}            {\cf asin}            {\cf atan}
{\cf cos}             {\cf exp}             {\cf finite?}
{\cf log}             {\cf nan?\ }           {\cf sin}
{\cf sqrt}            {\cf tan}
\end{scheme}

\textbf{Complex Library}

The \texttt{(scheme complex)} library exports procedures which are
typically only useful with complex values.

\begin{scheme}
{\cf angle}           {\cf imag-part}       {\cf magnitude}
{\cf make-polar}      {\cf make-rectangular}
{\cf real-part}
\end{scheme}

\textbf{Division Library}

The \texttt{(scheme division)} library exports procedures for integer
division.

\begin{scheme}
{\cf ceiling-quotient}
{\cf ceiling-remainder}                {\cf ceiling/}
{\cf centered-quotient}
{\cf centered-remainder}               {\cf centered/}
{\cf euclidean-quotient}
{\cf euclidean-remainder}              {\cf euclidean/}
{\cf floor-quotient}  {\cf floor-remainder} {\cf floor/}
{\cf round-quotient}  {\cf round-remainder} {\cf round/}
{\cf truncate-quotient}
{\cf truncate-remainder}               {\cf truncate/}
\end{scheme}

\textbf{Lazy Library}

The \texttt{(scheme lazy)} library exports procedures and syntax keywords for lazy evaluation.

\begin{scheme}
{\cf delay}           {\cf delay-force}     {\cf force}
{\cf make-promise}
\end{scheme}

\textbf{Case-Lambda Library}

The \texttt{(scheme case-lambda)} library exports the {\cf case-lambda}
syntax.

\begin{scheme}
{\cf case-lambda}
\end{scheme}

\textbf{CxR Library}

These sixteen procedures are the compositions of {\cf car} and {\cf
cdr} with more than three {\cf car} and {\cf cdr} operations.  For
example {\cf caddar} could be defined by

\begin{scheme}
(define caddar (lambda (x) (car (cdr (cdr (car x)))))){\rm.}%
\end{scheme}

Similar procedures with fewer operations are included in the base
library.  See section~\ref{listsection}.

\begin{scheme}
{\cf caaar}           {\cf caadr}
\ldots
{\cf cdddar}          {\cf cddddr}
\end{scheme}

\textbf{Eval Library}

The \texttt{(scheme eval)} library exports procedures for evaluating Scheme
data as programs.

\begin{scheme}
{\cf environment}     {\cf eval}
{\cf null-environment}
{\cf scheme-report-environment}
\end{scheme}

\textbf{Repl Library}

The \texttt{(scheme repl)} library exports the {\cf
  interaction-environment} procedure.

\begin{scheme}
{\cf interaction-environment}
\end{scheme}

\textbf{Process Context Library}

The \texttt{(scheme process-context)} library exports procedures for
accessing with the program's calling context.

\begin{scheme}
{\cf command-line}    {\cf exit}
{\cf get-environment-variable}
{\cf get-environment-variables}
\end{scheme}

\textbf{Load Library}

The \texttt{(scheme load)} library exports procedures for loading
Scheme expressions from files.

\begin{scheme}
{\cf load}
\end{scheme}

\textbf{File Library}

The \texttt{(scheme file)} library provides procedures for accessing
files.

\begin{scheme}
{\cf call-with-input-file}
{\cf call-with-output-file}            {\cf delete-file}
{\cf file-exists?\ }   {\cf open-binary-input-file}
{\cf open-binary-output-file}          {\cf open-input-file}
{\cf open-output-file}
{\cf with-input-from-file}
{\cf with-output-to-file}
\end{scheme}

\textbf{Read Library}

The \texttt{(scheme read)} library provides procedures for reading
Scheme objects.

\begin{scheme}
{\cf read}
\end{scheme}

\textbf{Write Library}

The \texttt{(scheme write)} library provides procedures for writing
Scheme objects.

\begin{scheme}
{\cf display}         {\cf write}           {\cf write-simple}
\end{scheme}

\textbf{Char Library}

The \texttt{(scheme char)} library provides procedures for dealing
with Unicode character operations.

\begin{scheme}
{\cf char-alphabetic?\ }                {\cf char-ci<=?}
{\cf char-ci<?\ }      {\cf char-ci=?\ }      {\cf char-ci>=?}
{\cf char-ci>?\ }      {\cf char-downcase}   {\cf char-foldcase}
{\cf char-lower-case?\ }                {\cf char-numeric?}
{\cf char-upcase}     {\cf char-upper-case?}
{\cf char-whitespace?\ }                {\cf digit-value}
{\cf string-ci<=?\ }   {\cf string-ci<?\ }    {\cf string-ci=?}
{\cf string-ci>=?\ }   {\cf string-ci>?\ }    {\cf string-downcase}
{\cf string-foldcase} {\cf string-upcase}
\end{scheme}

\textbf{Char Normalization Library}

The \texttt{(scheme char normalization)} library provides procedures
for dealing with Unicode normalization operations.

\begin{scheme}
{\cf string-ni<=?\ }   {\cf string-ni<?\ }    {\cf string-ni=?}
{\cf string-ni>=?\ }   {\cf string-ni>?}
\end{scheme}

\textbf{Time}

The \texttt{(scheme time)} library provides access to the system time.

\begin{scheme}
{\cf current-jiffy}   {\cf current-second}
{\cf jiffies-per-second}
\end{scheme}
