\chapter{Standard Modules}
\label{stdmodules}

%% Note, this is used to generate stdmod.tex.  The bindings could be
%% extracted automatically from the document, but this lets us choose
%% the ordering and optionally format manually where needed.

This section lists the exports provided by the standard modules.  The
modules are factored so as to separate features which may not be
supported by all implementations, or which may be expensive to load.

The {\cf scheme} module prefix is used for all standard modules, and
is reserved for use by future standards.

\textbf{Base Module}

The \texttt{(scheme base)} module exports many of the procedures and
syntax bindings that are traditionally associated with Scheme.

\begin{scheme}
{\cf \_}               {\cf ...}             {\cf *}
{\cf +}               {\cf -}               {\cf /}
{\cf <=}              {\cf <}               {\cf =>}
{\cf =}               {\cf >=}              {\cf >}
{\cf abs}             {\cf and}             {\cf append}
{\cf apply}           {\cf assoc}           {\cf assq}
{\cf assv}            {\cf begin}           {\cf boolean?}
{\cf bytevector-copy} {\cf bytevector-copy!}
{\cf bytevector-length}
{\cf bytevector-u8-ref}
{\cf bytevector-u8-set!}               {\cf bytevector?}
{\cf caar}            {\cf cadr}
{\cf call-with-current-continuation}
{\cf call-with-values}                 {\cf call/cc}
{\cf car}             {\cf case-lambda}     {\cf case}
{\cf cdddar}          {\cf cddddr}          {\cf cdr}
{\cf ceiling}         {\cf char->integer}   {\cf char<=?}
{\cf char<?\ }         {\cf char=?\ }         {\cf char>=?}
{\cf char>?\ }         {\cf char?\ }          {\cf complex?}
{\cf cond}            {\cf cond-expand}     {\cf cons}
{\cf define-syntax}   {\cf define}
{\cf define-record-type}               {\cf denominator}
{\cf do}              {\cf dynamic-wind}    {\cf else}
{\cf eq?\ }            {\cf equal?\ }         {\cf eqv?}
{\cf error}           {\cf error-object?}
{\cf error-object-message}
{\cf error-object-irritants}           {\cf even?}
{\cf exact->inexact}  {\cf exact-integer-sqrt}
{\cf exact-integer?\ } {\cf exact?\ }         {\cf expt}
{\cf floor}           {\cf for-each}        {\cf gcd}
{\cf guard}           {\cf if}              {\cf import}
{\cf inexact->exact}  {\cf inexact?\ }       {\cf integer->char}
{\cf integer?\ }       {\cf lambda}          {\cf lcm}
{\cf length}          {\cf let*}            {\cf let-syntax}
{\cf letrec*}         {\cf letrec-syntax}   {\cf let-values}
{\cf let*-values}     {\cf letrec}          {\cf let}
{\cf list-copy}       {\cf list->string}    {\cf list->vector}
{\cf list-ref}        {\cf list-set!}       {\cf list-tail}
{\cf list?\ }          {\cf list}            {\cf make-bytevector}
{\cf make-list}       {\cf make-parameter}  {\cf make-string}
{\cf make-vector}     {\cf map}             {\cf max}
{\cf member}          {\cf memq}            {\cf memv}
{\cf min}             {\cf modulo}          {\cf negative?}
{\cf not}             {\cf null?\ }          {\cf number->string}
{\cf number?\ }        {\cf numerator}       {\cf odd?}
{\cf or}              {\cf pair?\ }          {\cf parameterize}
{\cf partial-bytevector}
{\cf bytevector-copy-partial!}         {\cf positive?}
{\cf procedure?\ }     {\cf quasiquote}      {\cf quote}
{\cf quotient}        {\cf raise-continuable}
{\cf raise}           {\cf rational?\ }      {\cf rationalize}
{\cf real?\ }          {\cf remainder}       {\cf reverse}
{\cf round}           {\cf set!}            {\cf set-car!}
{\cf set-cdr!}        {\cf string->list}    {\cf string->number}
{\cf string->symbol}  {\cf string->vector}  {\cf string-append}
{\cf string-copy}     {\cf string-fill!}    {\cf string-for-each}
{\cf string-length}   {\cf string-map}      {\cf string-ref}
{\cf string-set!}     {\cf string<=?\ }      {\cf string<?}
{\cf string=?\ }       {\cf string>=?\ }      {\cf string>?}
{\cf string?\ }        {\cf string}          {\cf substring}
{\cf symbol->string}  {\cf symbol?\ }        {\cf syntax-error}
{\cf syntax-rules}    {\cf truncate}        {\cf values}
{\cf unquote}         {\cf unquote-splicing}
{\cf vector-copy}     {\cf vector->list}    {\cf vector->string}
{\cf vector-fill!}    {\cf vector-for-each} {\cf vector-length}
{\cf vector-map}      {\cf vector-ref}      {\cf vector-set!}
{\cf vector?\ }        {\cf vector}          {\cf zero?}
{\cf when}            {\cf with-exception-handler}
{\cf unless}
\end{scheme}

\textbf{Inexact Module}

The \texttt{(scheme inexact)} module exports procedures which are
typically only useful with inexact values.

\begin{scheme}
{\cf exp}             {\cf log}             {\cf sqrt}
{\cf sin}             {\cf cos}             {\cf tan}
{\cf asin}            {\cf acos}            {\cf atan}
{\cf finite?\ }        {\cf nan?}
\end{scheme}

\textbf{Complex Module}

The \texttt{(scheme complex)} module exports procedures which are
typically only useful with complex values.

\begin{scheme}
{\cf angle}           {\cf magnitude}       {\cf imag-part}
{\cf real-part}       {\cf make-polar}
{\cf make-rectangular}
\end{scheme}

\textbf{Division Module}

The \texttt{(scheme division)} module exports procedures for integer
division.

\begin{scheme}
{\cf floor/}          {\cf floor-quotient}  {\cf floor-remainder}
{\cf ceiling/}        {\cf ceiling-quotient}
{\cf ceiling-remainder}                {\cf truncate/}
{\cf truncate-quotient}
{\cf truncate-remainder}               {\cf round/}
{\cf round-quotient}  {\cf round-remainder} {\cf euclidean/}
{\cf euclidean-quotient}
{\cf euclidean-remainder}
\end{scheme}

\textbf{Lazy Module}

The \texttt{(scheme lazy)} module exports forms for lazy evaluation.

\begin{scheme}
{\cf delay}           {\cf force}           {\cf lazy}
\end{scheme}

\textbf{Eval Module}

The \texttt{(scheme eval)} module exports procedures for evaluating Scheme
data as programs.

\begin{scheme}
{\cf eval}            {\cf environment}
{\cf null-environment}
{\cf scheme-report-environment}
\end{scheme}

\textbf{Repl Module}

The \texttt{(scheme repl)} module exports the {\cf
  interaction-environment} procedure.

\begin{scheme}
{\cf interaction-environment}
\end{scheme}

\textbf{Process Context Module}

The \texttt{(scheme process-context)} module exports procedures for
accessing with the program's calling context.

\begin{scheme}
{\cf environment-variable}
{\cf environment-variables}            {\cf command-line}
{\cf exit}
\end{scheme}

\textbf{Load Module}

The \texttt{(scheme load)} module exports forms for loading and
including Scheme expressions from files.

\begin{scheme}
{\cf load}            {\cf include}         {\cf include-ci}
\end{scheme}

\textbf{I/O Module}

The \texttt{(scheme io)} module exports procedures for general input
and output on ports.

\begin{scheme}
{\cf binary-port?\ }   {\cf char-ready?\ }    {\cf character-port?}
{\cf close-port}      {\cf close-input-port}
{\cf close-output-port}
{\cf current-error-port}
{\cf current-input-port}
{\cf current-output-port}              {\cf eof-object?}
{\cf flush-output-port}
{\cf get-output-string}
{\cf get-output-bytevector}            {\cf input-port?}
{\cf newline}         {\cf open-input-string}
{\cf open-output-string}
{\cf open-input-bytevector}
{\cf open-output-bytevector}           {\cf output-port?}
{\cf peek-char}       {\cf peek-u8?\ }       {\cf port?}
{\cf port-open?\ }     {\cf read-char}       {\cf read-line}
{\cf read-u8}         {\cf u8-ready?\ }      {\cf write-char}
{\cf write-u8}
\end{scheme}

\textbf{File Module}

The \texttt{(scheme file)} module provides procedures for accessing
files.

\begin{scheme}
{\cf call-with-input-file}
{\cf call-with-output-file}            {\cf delete-file}
{\cf file-exists?\ }   {\cf open-input-file}
{\cf open-output-file}
{\cf open-binary-input-file}
{\cf open-binary-output-file}
{\cf with-input-from-file}
{\cf with-output-to-file}
\end{scheme}

\textbf{Read Module}

The \texttt{(scheme read)} module provides procedures for reading
Scheme objects.

\begin{scheme}
{\cf read}
\end{scheme}

\textbf{Write Module}

The \texttt{(scheme write)} module provides procedures for writing
Scheme objects.

\begin{scheme}
{\cf write}           {\cf display}
\end{scheme}

\textbf{Char Module}

The \texttt{(scheme char)} module provides procedures for dealing
with Unicode character operations.

\begin{scheme}
{\cf char-alphabetic?\ }                {\cf char-ci=?}
{\cf char-ci<?\ }      {\cf char-ci>?\ }      {\cf char-ci<=?}
{\cf char-ci>=?\ }     {\cf char-upcase}     {\cf char-downcase}
{\cf char-foldcase}   {\cf char-lower-case?}
{\cf char-numeric?\ }  {\cf char-upper-case?}
{\cf char-whitespace?\ }                {\cf string-ci=?}
{\cf string-ci<?\ }    {\cf string-ci>?\ }    {\cf string-ci<=?}
{\cf string-ci>=?\ }   {\cf string-upcase}   {\cf string-downcase}
{\cf string-foldcase}
\end{scheme}

\textbf{Char Normalization Module}

The \texttt{(scheme char normalization)} module provides procedures
for dealing with Unicode normalization operations.

\begin{scheme}
{\cf string-ni=?\ }    {\cf string-ni<?\ }    {\cf string-ni>?}
{\cf string-ni<=?\ }   {\cf string-ni>=?}
\end{scheme}

\textbf{Time}

The \texttt{(scheme time)} module provides access to the system time.

\begin{scheme}
{\cf current-second}  {\cf current-jiffy}
{\cf jiffies-per-second}
\end{scheme}
