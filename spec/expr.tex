%\vfill\eject
\chapter{Expressions}
\label{expressionchapter}

\newcommand{\syntax}{{\em Syntax: }}
\newcommand{\semantics}{{\em Semantics: }}

%[Deleted for R5RS because of multiple-value returns. -RK]
%A Scheme expression is a construct that returns a value, such as a
%variable reference, literal, procedure call, or conditional.

Expression types are categorized as {\em primitive} or {\em derived}.
Primitive expression types include variables and procedure calls.
Derived expression types are not semantically primitive, but can instead
be defined as macros.
With the exception of {\cf quasiquote}, whose macro definition is complex,
suitable definitions of the derived expressions are 
given in section~\ref{derivedsection}.

\section{Primitive expression types}
\label{primitivexps}

\subsection{Variable references}\unsection

\begin{entry}{%
\pproto{\hyper{variable}}{\exprtype}}

An expression consisting of a variable\index{variable}
(section~\ref{variablesection}) is a variable reference.  The value of
the variable reference is the value stored in the location to which the
variable is bound.  It is an error to reference an
unbound\index{unbound} variable.

\begin{scheme}
(define x 28)
x   \ev  28%
\end{scheme}
\end{entry}

\subsection{Literal expressions}\unsection
\label{literalsection}

\begin{entry}{%
\proto{quote}{ \hyper{datum}}{\exprtype}
\pproto{\singlequote\hyper{datum}}{\exprtype}
\pproto{\hyper{constant}}{\exprtype}}

{\cf (quote \hyper{datum})} evaluates to \hyper{datum}.\mainschindex{'}
\hyper{Datum}
may be any external representation of a Scheme object (see
section~\ref{externalreps}).  This notation is used to include literal
constants in Scheme code.

\begin{scheme}%
(quote a)                     \ev  a
(quote \sharpsign(a b c))     \ev  \#(a b c)
(quote (+ 1 2))               \ev  (+ 1 2)%
\end{scheme}

{\cf (quote \hyper{datum})} may be abbreviated as
\singlequote\hyper{datum}.  The two notations are equivalent in all
respects.

\begin{scheme}
'a                   \ev  a
'\#(a b c)           \ev  \#(a b c)
'()                  \ev  ()
'(+ 1 2)             \ev  (+ 1 2)
'(quote a)           \ev  (quote a)
''a                  \ev  (quote a)%
\end{scheme}

Numerical constants, string constants, character constants, and boolean
constants evaluate ``to themselves''; they need not be quoted.

\begin{scheme}
'"abc"     \ev  "abc"
"abc"      \ev  "abc"
'145932    \ev  145932
145932     \ev  145932
'\schtrue  \ev  \schtrue
\schtrue   \ev  \schtrue%
\end{scheme}

As noted in section~\ref{storagemodel}, it is an error to alter a constant
(i.e.~the value of a literal expression) using a mutation procedure like
{\cf set-car!}\ or {\cf string-set!}.

\end{entry}

\subsection{Procedure calls}\unsection

\begin{entry}{%
\pproto{(\hyper{operator} \hyperi{operand} \dotsfoo)}{\exprtype}}

A procedure call is written by simply enclosing in parentheses
expressions for the procedure to be called and the arguments to be
passed to it.  The operator and operand expressions are evaluated (in an
unspecified order) and the resulting procedure is passed the resulting
arguments.\mainindex{call}\mainindex{procedure call}
\begin{scheme}%
(+ 3 4)                          \ev  7
((if \schfalse + *) 3 4)         \ev  12%
\end{scheme}

A number of procedures are available as the values of variables in the
initial environment; for example, the addition and multiplication
procedures in the above examples are the values of the variables {\cf +}
and {\cf *}.  New procedures are created by evaluating \lambdaexp{}s
(see section~\ref{lambda}).
\todo{At Friedman's request, flushed mention of other ways.}
% or definitions (see section~\ref{define}).

Procedure calls may return any number of values (see \ide{values} in
section~\ref{proceduresection}).  With the exception of {\cf values}
the procedures available in the initial environment return one
value or, for procedures such as {\cf apply}, pass on the values returned
by a call to one of their arguments.

Procedure calls are also called {\em combinations}.
\mainindex{combination}

\begin{note} In contrast to other dialects of Lisp, the order of
evaluation is unspecified, and the operator expression and the operand
expressions are always evaluated with the same evaluation rules.
\end{note}

\begin{note}
Although the order of evaluation is otherwise unspecified, the effect of
any concurrent evaluation of the operator and operand expressions is
constrained to be consistent with some sequential order of evaluation.
The order of evaluation may be chosen differently for each procedure call.
\end{note}

\begin{note} In many dialects of Lisp, the empty combination, {\tt
()}, is a legitimate expression.  In Scheme, combinations must have at
least one subexpression, so {\tt ()} is not a syntactically valid
expression.  \todo{Dybvig: ``it should be obvious from the syntax.''}
\end{note}

\todo{Freeman:
I think an explanation as to why evaluation order is not specified
should be included.  It should not include any reference to parallel
evaluation.  Does any existing compiler generate better code because
the evaluation order is unspecified?  Clinger: yes: T3, MacScheme v2,
probably MIT Scheme and Chez Scheme.  But that's not the main reason
for leaving the order unspecified.}

\end{entry}


\subsection{Procedures}\unsection
\label{lamba}

\begin{entry}{%
\proto{lambda}{ \hyper{formals} \hyper{body}}{\exprtype}}

\syntax
\hyper{Formals} should be a formal arguments list as described below,
and \hyper{body} should be a sequence of one or more expressions.

\semantics
\vest A \lambdaexp{} evaluates to a procedure.  The environment in
effect when the \lambdaexp{} was evaluated is remembered as part of the
procedure.  When the procedure is later called with some actual
arguments, the environment in which the \lambdaexp{} was evaluated will
be extended by binding the variables in the formal argument list to
fresh locations, the corresponding actual argument values will be stored
in those locations, and the expressions in the body of the \lambdaexp{}
will be evaluated sequentially in the extended environment.
The result(s) of the last expression in the body will be returned as
the result(s) of the procedure call.

\begin{scheme}
(lambda (x) (+ x x))      \ev  {\em{}a procedure}
((lambda (x) (+ x x)) 4)  \ev  8

(define reverse-subtract
  (lambda (x y) (- y x)))
(reverse-subtract 7 10)         \ev  3

(define add4
  (let ((x 4))
    (lambda (y) (+ x y))))
(add4 6)                        \ev  10%
\end{scheme}

\hyper{Formals} should have one of the following forms:

\begin{itemize}
\item {\tt(\hyperi{variable} \dotsfoo)}:
The procedure takes a fixed number of arguments; when the procedure is
called, the arguments will be stored in the bindings of the
corresponding variables.

\item \hyper{variable}:
The procedure takes any number of arguments; when the procedure is
called, the sequence of actual arguments is converted into a newly
allocated list, and the list is stored in the binding of the
\hyper{variable}.

\item {\tt(\hyperi{variable} \dotsfoo{} \hyper{variable$_{n}$}\ {\bf.}\
\hyper{variable$_{n+1}$})}:
If a space-delimited period precedes the last variable, then
the procedure takes $n$ or more arguments, where $n$ is the
number of formal arguments before the period (there must
be at least one).
The value stored in the binding of the last variable will be a
newly allocated
list of the actual arguments left over after all the other actual
arguments have been matched up against the other formal arguments.
\end{itemize}

It is an error for a \hyper{variable} to appear more than once in
\hyper{formals}.

\begin{scheme}
((lambda x x) 3 4 5 6)          \ev  (3 4 5 6)
((lambda (x y . z) z)
 3 4 5 6)                       \ev  (5 6)%
\end{scheme}

Each procedure created as the result of evaluating a \lambdaexp{} is
(conceptually) tagged
with a storage location, in order to make \ide{eqv?} and
\ide{eq?} work on procedures (see section~\ref{equivalencesection}).

\end{entry}


\subsection{Conditionals}\unsection

\begin{entry}{%
\proto{if}{ \hyper{test} \hyper{consequent} \hyper{alternate}}{\exprtype}
\rproto{if}{ \hyper{test} \hyper{consequent}}{\exprtype}}  %\/ if hyper = italic

\syntax
\hyper{Test}, \hyper{consequent}, and \hyper{alternate} may be arbitrary
expressions.

\semantics
An {\cf if} expression is evaluated as follows: first,
\hyper{test} is evaluated.  If it yields a true value\index{true} (see
section~\ref{booleansection}), then \hyper{consequent} is evaluated and
its value(s) is(are) returned.  Otherwise \hyper{alternate} is evaluated and its
value(s) is(are) returned.  If \hyper{test} yields a false value and no
\hyper{alternate} is specified, then the result of the expression is
unspecified.

\begin{scheme}
(if (> 3 2) 'yes 'no)           \ev  yes
(if (> 2 3) 'yes 'no)           \ev  no
(if (> 3 2)
    (- 3 2)
    (+ 3 2))                    \ev  1%
\end{scheme}

\end{entry}


\subsection{Assignments}\unsection
\label{assignment}

\begin{entry}{%
\proto{set!}{ \hyper{variable} \hyper{expression}}{\exprtype}}

\hyper{Expression} is evaluated, and the resulting value is stored in
the location to which \hyper{variable} is bound.  \hyper{Variable} must
be bound either in some region\index{region} enclosing the {\cf set!}\ expression
or at top level.  The result of the {\cf set!} expression is
unspecified.

\begin{scheme}
(define x 2)
(+ x 1)                 \ev  3
(set! x 4)              \ev  \unspecified
(+ x 1)                 \ev  5%
\end{scheme}

\end{entry}


\section{Derived expression types}
\label{derivedexps}

The constructs in this section are hygienic, as discussed in
section~\ref{macrosection}.
For reference purposes, section~\ref{derivedsection} gives macro definitions
that will convert most of the constructs described in this section 
into the primitive constructs described in the previous section.

\todo{Mention that no definition of backquote is provided?}

\subsection{Conditionals}\unsection

\begin{entry}{%
\proto{cond}{ \hyperi{clause} \hyperii{clause} \dotsfoo}{\exprtype}}

\syntax
Each \hyper{clause} should be of the form
\begin{scheme}
(\hyper{test} \hyperi{expression} \dotsfoo)%
\end{scheme}
where \hyper{test} is any expression.  Alternatively, a \hyper{clause} may be
of the form
\begin{scheme}
(\hyper{test} => \hyper{expression})%
\end{scheme}
The last \hyper{clause} may be
an ``else clause,'' which has the form
\begin{scheme}
(else \hyperi{expression} \hyperii{expression} \dotsfoo)\rm.%
\end{scheme}
\mainschindex{else}
\mainschindex{=>}

\semantics
A {\cf cond} expression is evaluated by evaluating the \hyper{test}
expressions of successive \hyper{clause}s in order until one of them
evaluates to a true value\index{true} (see
section~\ref{booleansection}).  When a \hyper{test} evaluates to a true
value, then the remaining \hyper{expression}s in its \hyper{clause} are
evaluated in order, and the result(s) of the last \hyper{expression} in the
\hyper{clause} is(are) returned as the result(s) of the entire {\cf cond}
expression.

If the selected \hyper{clause} contains only the
\hyper{test} and no \hyper{expression}s, then the value of the
\hyper{test} is returned as the result.  If the selected \hyper{clause} uses the
\ide{=>} alternate form, then the \hyper{expression} is evaluated.
Its value must be a procedure that accepts one argument; this procedure is then
called on the value of the \hyper{test} and the value(s) returned by this
procedure is(are) returned by the {\cf cond} expression.

If all \hyper{test}s evaluate
to false values, and there is no else clause, then the result of
the conditional expression is unspecified; if there is an else
clause, then its \hyper{expression}s are evaluated, and the value(s) of
the last one is(are) returned.

\begin{scheme}
(cond ((> 3 2) 'greater)
      ((< 3 2) 'less))         \ev  greater%

(cond ((> 3 3) 'greater)
      ((< 3 3) 'less)
      (else 'equal))            \ev  equal%

(cond ((assv 'b '((a 1) (b 2))) => cadr)
      (else \schfalse{}))         \ev  2%
\end{scheme}


\end{entry}


\begin{entry}{%
\proto{case}{ \hyper{key} \hyperi{clause} \hyperii{clause} \dotsfoo}{\exprtype}}

\syntax
\hyper{Key} may be any expression.  Each \hyper{clause} should have
the form
\begin{scheme}
((\hyperi{datum} \dotsfoo) \hyperi{expression} \hyperii{expression} \dotsfoo)\rm,%
\end{scheme}
where each \hyper{datum} is an external representation of some object.
All the \hyper{datum}s must be distinct.
Alternatively, a \hyper{clause} may be of the form
\begin{scheme}
((\hyperi{datum} \dotsfoo) => \hyper{expression})%
\end{scheme}
The last \hyper{clause} may be an ``else clause,'' which has one of the forms
\begin{scheme}
(else \hyperi{expression} \hyperii{expression} \dotsfoo)
\end{scheme}
or
\begin{scheme}
(else => \hyper{expression})\rm.%
\end{scheme}
\schindex{else}

\semantics
A {\cf case} expression is evaluated as follows.  \hyper{Key} is
evaluated and its result is compared against each \hyper{datum}.  If the
result of evaluating \hyper{key} is equivalent (in the sense of
{\cf eqv?}; see section~\ref{eqv?}) to a \hyper{datum}, then the
expressions in the corresponding \hyper{clause} are evaluated from left
to right and the result(s) of the last expression in the \hyper{clause} is(are)
returned as the result(s) of the {\cf case} expression.

If the result of
evaluating \hyper{key} is different from every \hyper{datum}, then if
there is an else clause its expressions are evaluated and the
result(s) of the last is(are) the result(s) of the {\cf case} expression;
otherwise the result of the {\cf case} expression is unspecified.

If the selected \hyper{clause} or else clause uses the
\ide{=>} alternate form, then the \hyper{expression} is evaluated.
Its value must be a procedure that accepts one argument; this procedure is then
called on the value of the \hyper{key} and the value(s) returned by this
procedure is(are) returned by the {\cf case} expression.

\begin{scheme}
(case (* 2 3)
  ((2 3 5 7) 'prime)
  ((1 4 6 8 9) 'composite))     \ev  composite
(case (car '(c d))
  ((a) 'a)
  ((b) 'b))                     \ev  \unspecified
(case (car '(c d))
  ((a e i o u) 'vowel)
  ((w y) 'semivowel)
  (else => (lambda (x) x)))     \ev  c%
\end{scheme}

\end{entry}


\begin{entry}{%
\proto{and}{ \hyperi{test} \dotsfoo}{\exprtype}}

The \hyper{test} expressions are evaluated from left to right, and if
any expression evaluates to a false value (see
section~\ref{booleansection}), \schfalse{} is returned.  Any remaining expressions
are not evaluated.  If all the expressions evaluate to true values, the
value of the last expression is returned.  If there are no expressions
then \schtrue{} is returned.

\begin{scheme}
(and (= 2 2) (> 2 1))           \ev  \schtrue
(and (= 2 2) (< 2 1))           \ev  \schfalse
(and 1 2 'c '(f g))             \ev  (f g)
(and)                           \ev  \schtrue%
\end{scheme}

\end{entry}


\begin{entry}{%
\proto{or}{ \hyperi{test} \dotsfoo}{\exprtype}}

The \hyper{test} expressions are evaluated from left to right, and the value of the
first expression that evaluates to a true value (see
section~\ref{booleansection}) is returned.  Any remaining expressions
are not evaluated.  If all expressions evaluate to false values, 
or if there are no expressions, \schfalse{} is returned.  

\begin{scheme}
(or (= 2 2) (> 2 1))            \ev  \schtrue
(or (= 2 2) (< 2 1))            \ev  \schtrue
(or \schfalse \schfalse \schfalse) \ev  \schfalse
(or (memq 'b '(a b c)) 
    (/ 3 0))                    \ev  (b c)%
\end{scheme}

\end{entry}


\subsection{Binding constructs}

The four binding constructs {\cf let}, {\cf let*}, {\cf letrec}, and {\cf letrec*}
give Scheme a block structure, like Algol 60.  The syntax of the three
constructs is identical, but they differ in the regions\index{region} they establish
for their variable bindings.  In a {\cf let} expression, the initial
values are computed before any of the variables become bound; in a
{\cf let*} expression, the bindings and evaluations are performed
sequentially; while in {\cf letrec} and {\cf letrec*} expressions,
all the bindings are in
effect while their initial values are being computed, thus allowing
mutually recursive definitions.

\begin{entry}{%
\proto{let}{ \hyper{bindings} \hyper{body}}{\exprtype}}

\syntax
\hyper{Bindings} should have the form
\begin{scheme}
((\hyperi{variable} \hyperi{init}) \dotsfoo)\rm,%
\end{scheme}
where each \hyper{init} is an expression, and \hyper{body} should be a
sequence of zero or more definitions followed by a
sequence of one or more expressions.  It is
an error for a \hyper{variable} to appear more than once in the list of variables
being bound.

\semantics
The \hyper{init}s are evaluated in the current environment (in some
unspecified order), the \hyper{variable}s are bound to fresh locations
holding the results, the \hyper{body} is evaluated in the extended
environment, and the value(s) of the last expression of \hyper{body}
is(are) returned.  Each binding of a \hyper{variable} has \hyper{body} as its
region.\index{region}

\begin{scheme}
(let ((x 2) (y 3))
  (* x y))                      \ev  6

(let ((x 2) (y 3))
  (let ((x 7)
        (z (+ x y)))
    (* z x)))                   \ev  35%
\end{scheme}

See also named {\cf let}, section \ref{namedlet}.

\end{entry}


\begin{entry}{%
\proto{let*}{ \hyper{bindings} \hyper{body}}{\exprtype}}\nobreak

\nobreak
\syntax
\hyper{Bindings} should have the form
\begin{scheme}
((\hyperi{variable} \hyperi{init}) \dotsfoo)\rm,%
\end{scheme}
and \hyper{body} should be a sequence of
sequence of zero or more definitions followed by a
one or more expressions.

\semantics
{\cf Let*} is similar to {\cf let}, but the bindings are performed
sequentially from left to right, and the region\index{region} of a binding indicated
by {\cf(\hyper{variable} \hyper{init})} is that part of the {\cf let*}
expression to the right of the binding.  Thus the second binding is done
in an environment in which the first binding is visible, and so on.

\begin{scheme}
(let ((x 2) (y 3))
  (let* ((x 7)
         (z (+ x y)))
    (* z x)))             \ev  70%
\end{scheme}

\end{entry}


\begin{entry}{%
\proto{letrec}{ \hyper{bindings} \hyper{body}}{\exprtype}}

\syntax
\hyper{Bindings} should have the form
\begin{scheme}
((\hyperi{variable} \hyperi{init}) \dotsfoo)\rm,%
\end{scheme}
and \hyper{body} should be a sequence of
sequence of zero or more definitions followed by a
one or more expressions. It is an error for a \hyper{variable} to appear more
than once in the list of variables being bound.

\semantics
The \hyper{variable}s are bound to fresh locations holding undefined
values, the \hyper{init}s are evaluated in the resulting environment (in
some unspecified order), each \hyper{variable} is assigned to the result
of the corresponding \hyper{init}, the \hyper{body} is evaluated in the
resulting environment, and the value(s) of the last expression in
\hyper{body} is(are) returned.  Each binding of a \hyper{variable} has the
entire {\cf letrec} expression as its region\index{region}, making it possible to
define mutually recursive procedures.

\begin{scheme}
%(letrec ((x 2) (y 3))
%  (letrec ((foo (lambda (z) (+ x y z))) (x 7))
%    (foo 4)))                   \ev  14
%
(letrec ((even?
          (lambda (n)
            (if (zero? n)
                \schtrue
                (odd? (- n 1)))))
         (odd?
          (lambda (n)
            (if (zero? n)
                \schfalse
                (even? (- n 1))))))
  (even? 88))   
		\ev  \schtrue%
\end{scheme}

One restriction on {\cf letrec} is very important: it must be possible
to evaluate each \hyper{init} without assigning or referring to the value of any
\hyper{variable}.  If this restriction is violated, then it is an error.  The
restriction is necessary because Scheme passes arguments by value rather than by
name.  In the most common uses of {\cf letrec}, all the \hyper{init}s are
\lambdaexp{}s and the restriction is satisfied automatically.
Another restriction is that the continuation of each \hyper{init}
should not be invoked more than once.

% \todo{use or uses?  --- Jinx.}

\end{entry}


\begin{entry}{%
\proto{letrec*}{ \hyper{bindings} \hyper{body}}{\exprtype}}

\syntax
\hyper{Bindings} should have the form
\begin{scheme}
((\hyperi{variable} \hyperi{init}) \dotsfoo)\rm,%
\end{scheme}
and \hyper{body} should be a sequence of
sequence of zero or more definitions followed by a
one or more expressions. It is an error for a \hyper{variable} to appear more
than once in the list of variables being bound.

\semantics
The \hyper{variable}s are bound to fresh locations,
each \hyper{variable} is assigned in left-to-right order to the
result of evaluating the corresponding \hyper{init}, the \hyper{body} is
evaluated in the resulting environment, and the values of the last
expression in \hyper{body} are returned. 
Despite the left-to-right evaluation and assignment order, each binding of
a \hyper{variable} has the entire {\cf letrec*} expression as its
region\index{region}, making it possible to define mutually recursive
procedures.

In all four constructs, the continuation of each expression
used to compute initial values must not be invoked more than once.

\begin{scheme}
(letrec* ((p
           (lambda (x)
             (+ 1 (q (- x 1)))))
          (q
           (lambda (y)
             (if (zero? y)
                 0
                 (+ 1 (p (- y 1))))))
          (x (p 5))
          (y x))
  y)
                \ev  5%
\end{scheme}

It must be possible to evaluate each \hyper{init} without assigning or
referring to the value of the corresponding \hyper{variable} or the
\hyper{variable} of any of the bindings that follow it in
\hyper{bindings}.

\end{entry}


\subsection{Sequencing}\unsection

\begin{entry}{%
\proto{begin}{ \hyperi{expression} \hyperii{expression} \dotsfoo}{\exprtype}}

The \hyper{expression}s are evaluated sequentially from left to right,
and the value(s) of the last \hyper{expression} is(are) returned.  This
expression type is used to sequence side effects such as input and
output.

\begin{scheme}
(define x 0)

(begin (set! x 5)
       (+ x 1))                  \ev  6

(begin (display "4 plus 1 equals ")
       (display (+ 4 1)))      \ev  \unspecified
 \>{\em and prints}  4 plus 1 equals 5%
\end{scheme}

\end{entry}


\subsection{Iteration}%\unsection

\noindent%
\pproto{(do ((\hyperi{variable} \hyperi{init} \hyperi{step})}{\exprtype}
\mainschindex{do}{\tt\obeyspaces%
     \dotsfoo)\\
    (\hyper{test} \hyper{expression} \dotsfoo)\\
  \hyper{command} \dotsfoo)}

{\cf Do} is an iteration construct.  It specifies a set of variables to
be bound, how they are to be initialized at the start, and how they are
to be updated on each iteration.  When a termination condition is met,
the loop exits after evaluating the \hyper{expression}s.

{\cf Do} expressions are evaluated as follows:
The \hyper{init} expressions are evaluated (in some unspecified order),
the \hyper{variable}s are bound to fresh locations, the results of the
\hyper{init} expressions are stored in the bindings of the
\hyper{variable}s, and then the iteration phase begins.

\vest Each iteration begins by evaluating \hyper{test}; if the result is
false (see section~\ref{booleansection}), then the \hyper{command}
expressions are evaluated in order for effect, the \hyper{step}
expressions are evaluated in some unspecified order, the
\hyper{variable}s are bound to fresh locations, the results of the
\hyper{step}s are stored in the bindings of the
\hyper{variable}s, and the next iteration begins.

\vest If \hyper{test} evaluates to a true value, then the
\hyper{expression}s are evaluated from left to right and the value(s) of
the last \hyper{expression} is(are) returned.  If no \hyper{expression}s
are present, then the value of the {\cf do} expression is unspecified.

\vest The region\index{region} of the binding of a \hyper{variable}
consists of the entire {\cf do} expression except for the \hyper{init}s.
It is an error for a \hyper{variable} to appear more than once in the
list of {\cf do} variables.

\vest A \hyper{step} may be omitted, in which case the effect is the
same as if {\cf(\hyper{variable} \hyper{init} \hyper{variable})} had
been written instead of {\cf(\hyper{variable} \hyper{init})}.

\begin{scheme}
(do ((vec (make-vector 5))
     (i 0 (+ i 1)))
    ((= i 5) vec)
  (vector-set! vec i i))          \ev  \#(0 1 2 3 4)

(let ((x '(1 3 5 7 9)))
  (do ((x x (cdr x))
       (sum 0 (+ sum (car x))))
      ((null? x) sum)))             \ev  25%
\end{scheme}

%\end{entry}


\begin{entry}{%
\rproto{let}{ \hyper{variable} \hyper{bindings} \hyper{body}}{\exprtype}}

\label{namedlet}
``Named {\cf let}'' is a variant on the syntax of \ide{let} which provides
a more general looping construct than {\cf do} and may also be used to express
recursions.
It has the same syntax and semantics as ordinary {\cf let}
except that \hyper{variable} is bound within \hyper{body} to a procedure
whose formal arguments are the bound variables and whose body is
\hyper{body}.  Thus the execution of \hyper{body} may be repeated by
invoking the procedure named by \hyper{variable}.

%                                              |  <-- right margin
\begin{scheme}
(let loop ((numbers '(3 -2 1 6 -5))
           (nonneg '())
           (neg '()))
  (cond ((null? numbers) (list nonneg neg))
        ((>= (car numbers) 0)
         (loop (cdr numbers)
               (cons (car numbers) nonneg)
               neg))
        ((< (car numbers) 0)
         (loop (cdr numbers)
               nonneg
               (cons (car numbers) neg))))) %
  \lev  ((6 1 3) (-5 -2))%
\end{scheme}

\end{entry}


\subsection{Delayed evaluation}\unsection

\begin{entry}{%
\proto{delay}{ \hyper{expression}}{lazy module syntax}}

\todo{Fix.}

The {\cf delay} construct is used together with the procedure \ide{force} to
implement \defining{lazy evaluation} or \defining{call by need}.
{\tt(delay~\hyper{expression})} returns an object called a
\defining{promise} which at some point in the future may be asked (by
the {\cf force} procedure) \todo{Bartley's white lie; OK?} to evaluate
\hyper{expression}, and deliver the resulting value.
The effect of \hyper{expression} returning multiple values
is unspecified.

See the description of {\cf force} (section~\ref{force}) for a
more complete description of {\cf delay}.

\end{entry}

\subsection{Dynamic Bindings}\unsection

\begin{entry}{%
\pproto{(parameterize ((param value) \dotsfoo)}{syntax}
{\tt\obeyspaces%
\hspace*{1em}\hyperi{body}}}

\domain{The value of the \var{param} expressions must be parameter objects.}
The {\cf parameterize} form is used to change the values returned by
parameter objects for the dynamic extent of the body.
The expressions \var{param} and \texttt{(\var{converter} \var{value})}
are evaluated in an unspecified order.  The \hyper{expr}s are
evaluated in order in a dynamic extent during which calls to the
\var{param} parameter objects return the result of the corresponding
\texttt{(\var{converter} \var{value})}.  The result(s) of the last
\hyper{expr} is(are) returned as the result(s) of the entire {\cf
  parameterize} form.

If an implementation supports multiple threads of execution, then {\cf
  parameterize} must not change the associated values of any
parameters in any thread created before or after the {\cf
  parameterize} form.

See the description of {\cf make-parameter}
(section~\ref{make-parameter}) for a more complete description of {\cf
  parameterize}.
\end{entry}


\subsection{Exception Handling}\unsection
\label{guard}

\begin{entry}{%
\pproto{(guard (\hyper{variable}}{\exprtype}
{\tt\obeyspaces%
\hspace*{4em}\hyperi{cond clause} \hyperii{cond clause} \dotsfoo)\\
\hspace*{2em}\hyper{body})}\\
}

\syntax
Each \hyper{cond clause} is as in the specification of {\cf cond}.
%(See report section~\ref{derivedexps}.)

\semantics
Evaluating a {\cf guard} form evaluates \hyper{body} with an exception
handler that binds the raised object to \hyper{variable} and within the scope of
that binding evaluates the clauses as if they were the clauses of a
{\cf cond} expression. That implicit {\cf cond} expression is evaluated with the
continuation and dynamic extent of the {\cf guard} expression. If every
\hyper{cond clause}'s \hyper{test} evaluates to \schfalse{} and there
is no {\cf else} clause, then
{\cf raise} is re-invoked on the raised object within the dynamic
extent of the original call to {\cf raise} except that the current
exception handler is that of the {\cf guard} expression.

The final expression in a \hyper{cond} clause is in a tail context if
the {\cf guard} expression itself is.

See section~\ref{exceptionsection} for a more complete discussion of
exceptions.
\end{entry}


\subsection{Quasiquotation}\unsection
\label{quasiquotesection}

\begin{entry}{%
\proto{quasiquote}{ \hyper{qq template}}{\exprtype} \nopagebreak
\pproto{\backquote\hyper{qq template}}{\exprtype}}

``Backquote'' or ``quasiquote''\index{backquote} expressions are useful
for constructing a list or vector structure when most but not all of the
desired structure is known in advance.  If no
commas\index{comma} appear within the \hyper{qq template}, the result of
evaluating
\backquote\hyper{qq template} is equivalent to the result of evaluating
\singlequote\hyper{qq template}.  If a comma\mainschindex{,} appears within the
\hyper{qq template}, however, the expression following the comma is
evaluated (``unquoted'') and its result is inserted into the structure
instead of the comma and the expression.  If a comma appears followed
immediately by an at-sign (\atsign),\mainschindex{,@} then the following
expression must evaluate to a list; the opening and closing parentheses
of the list are then ``stripped away'' and the elements of the list are
inserted in place of the comma at-sign expression sequence.  A comma
at-sign should only appear within a list or vector \hyper{qq template}.

% struck: "(in the sense of {\cf equal?})" after "equivalent"

\begin{scheme}
`(list ,(+ 1 2) 4)  \ev  (list 3 4)
(let ((name 'a)) `(list ,name ',name)) %
          \lev  (list a (quote a))
`(a ,(+ 1 2) ,@(map abs '(4 -5 6)) b) %
          \lev  (a 3 4 5 6 b)
`(({\cf foo} ,(- 10 3)) ,@(cdr '(c)) . ,(car '(cons))) %
          \lev  ((foo 7) . cons)
`\#(10 5 ,(sqrt 4) ,@(map sqrt '(16 9)) 8) %
          \lev  \#(10 5 2 4 3 8)%
\end{scheme}

Quasiquote forms may be nested.  Substitutions are made only for
unquoted components appearing at the same nesting level
as the outermost backquote.  The nesting level increases by one inside
each successive quasiquotation, and decreases by one inside each
unquotation.

\begin{scheme}
`(a `(b ,(+ 1 2) ,(foo ,(+ 1 3) d) e) f) %
          \lev  (a `(b ,(+ 1 2) ,(foo 4 d) e) f)
(let ((name1 'x)
      (name2 'y))
  `(a `(b ,,name1 ,',name2 d) e)) %
          \lev  (a `(b ,x ,'y d) e)%
\end{scheme}

The two notations
 \backquote\hyper{qq template} and {\tt (quasiquote \hyper{qq template})}
 are identical in all respects.
 {\cf,\hyper{expression}} is identical to {\cf (unquote \hyper{expression})},
 and
 {\cf,@\hyper{expression}} is identical to {\cf (unquote-splicing \hyper{expression})}.
The external syntax generated by \ide{write} for two-element lists whose
car is one of these symbols may vary between implementations.
\mainschindex{`}

\begin{scheme}
(quasiquote (list (unquote (+ 1 2)) 4)) %
          \lev  (list 3 4)
'(quasiquote (list (unquote (+ 1 2)) 4)) %
          \lev  `(list ,(+ 1 2) 4)
     {\em{}i.e.,} (quasiquote (list (unquote (+ 1 2)) 4))%
\end{scheme}

Unpredictable behavior can result if any of the symbols
\ide{quasiquote}, \ide{unquote}, or \ide{unquote-splicing} appear in
positions within a \hyper{qq template} otherwise than as described above.

\end{entry}

\subsection{Case-lambda}\unsection
\label{caselambdasection}
\begin{entry}{%
\proto{case-lambda}{ \hyperi{clause} \hyperii{clause} \dotsfoo}{\exprtype}}

\syntax
Each \hyper{clause} should be of the form
(\hyper{formals} \hyper{body})%
where \hyper{formals} and \hyper{body} have the same syntax
as in a \lambdaexp.

\semantics
A {\cf case-lambda} expression evaluates to a procedure that accepts
a variable number of arguments and is lexically scoped in the same
manner as procedures resulting from a \lambdaexp. When the procedure
is called, then the first \hyper{clause} for which the arguments agree
with \hyper{formals} is selected, where agreement is specified as for
the \hyper{formals} of a \lambdaexp. The variables of \hyper{formals} are
bound to fresh locations, the values of the arguments are stored in those
locations, the \hyper{body} is evaluated in the extended environment,
and the results of \hyper{body} are returned as the results of the
procedure call.

It is an error for the arguments not to agree with
the \hyper{formals} of any \hyper{clause}.

\begin{scheme}
(define plus
  (case-lambda 
    (() 0)
    ((x) x)
    ((x y) (+ x y))
    ((x y z) (+ (+ x y) z))
    (args (apply + args))))

(plus) \ev 0
(plus 1) \ev 1
(plus 1 2 3) \ev 6
\end{scheme}

\end{entry}

\subsection{Reader Labels}\unsection
\label{labelsection}

\begin{entry}{%
\pproto{\#\hyper{n}=\hyper{datum}}{\exprtype}
\pproto{\#\hyper{n}\#}{\exprtype}}

\hyper{N} must be an exact unsigned decimal integer.  The syntax
\texttt{\#\hyper{n}=\hyper{datum}} reads as \hyper{datum}, except
that within the syntax of \hyper{datum} the \hyper{datum} is labelled
by \hyper{n}.

The syntax \texttt{\#\hyper{n}\#} serves as a reference to some
object labelled by \texttt{\#\hyper{n}=}; the result is the same
object as the \texttt{\#\hyper{n}}= as compared with {\cf eqv?}
(see section~\ref{equivalencesection}). This permits notation of
structures with shared or circular substructure.

\begin{scheme}
(let ((x (list 'a 'b 'c)))
  (set-cdr! (cddr x) x)
  x)                       \ev \#0=(a b c . \#0\#)
\end{scheme}

A reference \texttt{\#\hyper{n}\#} may occur only after a label
\texttt{\#\hyper{n}=}; forward references are not permitted. In
addition, the reference may not appear as the labelled object itself
(that is, one may not write \texttt{\#\hyper{n}= \#\hyper{n}\#}),
because the object labelled by \texttt{\#\hyper{n}=} is not well
defined in this case.

\end{entry}

\section{Macros}
\label{macrosection}

Scheme programs can define and use new derived expression types,
 called {\em macros}.\mainindex{macro}
Program-defined expression types have the syntax
\begin{scheme}
(\hyper{keyword} {\hyper{datum}} ...)%
\end{scheme}%
where \hyper{keyword} is an identifier that uniquely determines the
expression type.  This identifier is called the {\em syntactic
keyword}\index{syntactic keyword}, or simply {\em
keyword}\index{keyword}, of the macro\index{macro keyword}.  The
number of the \hyper{datum}s, and their syntax, depends on the
expression type.

Each instance of a macro is called a {\em use}\index{macro use}
of the macro.
The set of rules that specifies
how a use of a macro is transcribed into a more primitive expression
is called the {\em transformer}\index{macro transformer}
of the macro.

The macro definition facility consists of two parts:

\begin{itemize}
\item A set of expressions used to establish that certain identifiers
are macro keywords, associate them with macro transformers, and control
the scope within which a macro is defined, and

\item a pattern language for specifying macro transformers.
\end{itemize}

The syntactic keyword of a macro may shadow variable bindings, and local
variable bindings may shadow keyword bindings.  \index{keyword}  All macros
defined using the pattern language  are ``hygienic'' and ``referentially
transparent'' and thus preserve Scheme's lexical scoping~\cite{Kohlbecker86,
hygienic,Bawden88,macrosthatwork,syntacticabstraction}:
\mainindex{hygienic}
\mainindex{referentially transparent}


\begin{itemize}

\item If a macro transformer inserts a binding for an identifier
(variable or keyword), the identifier will in effect be renamed
throughout its scope to avoid conflicts with other identifiers.
Note that a \ide{define} at top level may or may not introduce a binding;
see section~\ref{defines}.

\item If a macro transformer inserts a free reference to an
identifier, the reference refers to the binding that was visible
where the transformer was specified, regardless of any local
bindings that may surround the use of the macro.

\end{itemize}

%The low-level facility permits non-hygienic macros to be written,
%and may be used to implement the high-level pattern language.

% The fourth section describes some features that would make the
% low-level macro facility easier to use directly.

\subsection{Binding constructs for syntactic keywords}
\label{bindsyntax}

{\cf Let-syntax} and {\cf letrec-syntax} are
analogous to {\cf let} and {\cf letrec}, but they bind
syntactic keywords to macro transformers instead of binding variables
to locations that contain values.  Syntactic keywords may also be
bound at top level; see section~\ref{define-syntax}.

\begin{entry}{%
\proto{let-syntax}{ \hyper{bindings} \hyper{body}}{\exprtype}}

\syntax
\hyper{Bindings} should have the form
\begin{scheme}
((\hyper{keyword} \hyper{transformer spec}) \dotsfoo)%
\end{scheme}
Each \hyper{keyword} is an identifier,
each \hyper{transformer spec} is an instance of {\cf syntax-rules}, and
\hyper{body} should be a sequence of one or more expressions.  It is an error
for a \hyper{keyword} to appear more than once in the list of keywords
being bound.

\semantics
The \hyper{body} is expanded in the syntactic environment
obtained by extending the syntactic environment of the
{\cf let-syntax} expression with macros whose keywords are
the \hyper{keyword}s, bound to the specified transformers.
Each binding of a \hyper{keyword} has \hyper{body} as its region.

\begin{scheme}
(let-syntax ((when (syntax-rules ()
                     ((when test stmt1 stmt2 ...)
                      (if test
                          (begin stmt1
                                 stmt2 ...))))))
  (let ((if \schtrue))
    (when if (set! if 'now))
    if))                           \ev  now

(let ((x 'outer))
  (let-syntax ((m (syntax-rules () ((m) x))))
    (let ((x 'inner))
      (m))))                       \ev  outer%
\end{scheme}

\end{entry}

\begin{entry}{%
\proto{letrec-syntax}{ \hyper{bindings} \hyper{body}}{\exprtype}}

\syntax
Same as for {\cf let-syntax}.

\semantics
 The \hyper{body} is expanded in the syntactic environment obtained by
extending the syntactic environment of the {\cf letrec-syntax}
expression with macros whose keywords are the
\hyper{keyword}s, bound to the specified transformers.
Each binding of a \hyper{keyword} has the \hyper{bindings}
as well as the \hyper{body} within its region,
so the transformers can
transcribe expressions into uses of the macros
introduced by the {\cf letrec-syntax} expression.

\begin{scheme}
(letrec-syntax
  ((my-or (syntax-rules ()
            ((my-or) \schfalse)
            ((my-or e) e)
            ((my-or e1 e2 ...)
             (let ((temp e1))
               (if temp
                   temp
                   (my-or e2 ...)))))))
  (let ((x \schfalse)
        (y 7)
        (temp 8)
        (let odd?)
        (if even?))
    (my-or x
           (let temp)
           (if y)
           y)))        \ev  7%
\end{scheme}

\end{entry}

\subsection{Pattern language}
\label{patternlanguage}

A \hyper{transformer spec} has one of the following form:

\begin{entry}{%
\pproto{(syntax-rules (\hyper{literal} \dotsfoo)}{\exprtype}
{\tt\obeyspaces%
\hspace*{1em}\hyper{syntax rule} \dotsfoo)\\
}
\pproto{(syntax-rules \hyper{ellipsis} (\hyper{literal} \dotsfoo)}{\exprtype}
{\tt\obeyspaces%
\hspace*{1em}\hyper{syntax rule} \dotsfoo)}
}

\syntax
Each \hyper{literal}, as well as the \hyper{ellipsis} in the second form must
be an identifier, and each
\hyper{syntax rule} should be of the form
\begin{scheme}
(\hyper{pattern} \hyper{template})%
\end{scheme}
The \hyper{pattern} in a \hyper{syntax rule} is a list \hyper{pattern}
whose first subform is an identifier.

A \hyper{pattern} is either an identifier, a constant, or one of the
following
\begin{scheme}
(\hyper{pattern} \ldots)
(\hyper{pattern} \hyper{pattern} \ldots . \hyper{pattern})
(\hyper{pattern} \ldots \hyper{pattern} \hyper{ellipsis} \hyper{pattern} \ldots)
(\hyper{pattern} \ldots \hyper{pattern} \hyper{ellipsis} \hyper{pattern} \ldots
  . \hyper{pattern})
\#(\hyper{pattern} \ldots)
\#(\hyper{pattern} \ldots \hyper{pattern} \hyper{ellipsis} \hyper{pattern} \ldots)%
\end{scheme}
and a template is either an identifier, a constant, or one of the following
\begin{scheme}
(\hyper{element} \ldots)
(\hyper{element} \hyper{element} \ldots . \hyper{template})
(\hyper{ellipsis} \hyper{template})
\#(\hyper{element} \ldots)%
\end{scheme}
where an \hyper{element} is a \hyper{template} optionally
followed by an \hyper{ellipsis}.
An \hyper{ellipsis} is the identifier specified in the second form
of {\cf syntax-rules}, or the default identifier {\cf ...}
(three consecutive periods) otherwise.\schindex{...}

\semantics An instance of {\cf syntax-rules} produces a new macro
transformer by specifying a sequence of hygienic rewrite rules.  A use
of a macro whose keyword is associated with a transformer specified by
{\cf syntax-rules} is matched against the patterns contained in the
\hyper{syntax rule}s, beginning with the leftmost \hyper{syntax rule}.
When a match is found, the macro use is transcribed hygienically
according to the template.

An identifier appearing within a \hyper{pattern} may be an underscore
(``{\cf \_}''), a literal identifier listed in the list of \hyper{literal}s,
or the \hyper{ellipsis}.
All other identifiers appearing within a \hyper{pattern} are
{\em pattern variables}.

The keyword at the beginning of the pattern in a
\hyper{syntax rule} is not involved in the matching and
is not considered a pattern variable or literal identifier.

%% \begin{rationale}
%% The scope of the keyword is determined by the expression or syntax
%% definition that binds it to the associated macro transformer.
%% If the keyword were a pattern variable or literal
%% identifier, then
%% the template that follows the pattern would be within its scope
%% regardless of whether the keyword were bound by {\cf let-syntax}
%% or by {\cf letrec-syntax}.
%% \end{rationale}

Pattern variables match arbitrary input elements and
are used to refer to elements of the input in the template.  
It is an error for the same pattern variable to appear more than once in a
\hyper{pattern}.

Underscores also match arbitrary input subforms but are not pattern variables
and so cannot be used to refer to those elements.  If an underscore appears
in the \hyper{literal}s list, then that takes precedence and
underscores in the \hyper{pattern} match as literals.
Multiple underscores may appear in a \hyper{pattern}.

Identifiers that appear in \texttt{(\hyper{literal} \dotsfoo)} are
interpreted as literal
identifiers to be matched against corresponding subforms of the input.
A subform in the input matches a literal identifier if and only if it is an
identifier and either both its occurrence in the macro expression and its
occurrence in the macro definition have the same lexical binding, or
the two identifiers are equal and both have no lexical binding.

% [Bill Rozas suggested the term "noise word" for these literal
% identifiers, but in their most interesting uses, such as a setf
% macro, they aren't noise words at all. -- Will]

A subpattern followed by \hyper{ellipsis} can match zero or more elements of
the input, unless \hyper{ellipsis} appears in the \hyper{literal}s in which
case it is matched as a literal.

More formally, an input form $F$ matches a pattern $P$ if and only if:

\begin{itemize}
\item $P$ is an underscore (``{\cf \_}'').

\item $P$ is a non-literal identifier; or

\item $P$ is a literal identifier and $F$ is an identifier with the same
      binding; or

\item $P$ is a list {\cf ($P_1$ $\dots$ $P_n$)} and $F$ is a
      list of $n$
      forms that match $P_1$ through $P_n$, respectively; or

\item $P$ is an improper list
      {\cf ($P_1$ $P_2$ $\dots$ $P_n$ . $P_{n+1}$)}
      and $F$ is a list or
      improper list of $n$ or more forms that match $P_1$ through $P_n$,
      respectively, and whose $n$th ``cdr'' matches $P_{n+1}$; or

\item $P$ is of the form
      {\cf ($P_1$ $\dots$ $P_{e-1}$ $P_{e}$ \meta{ellipsis} $P_{m+1}$ \dotsfoo{} $P_n$)}
      where $F$ is
      a proper list of $n$ forms, the first $e-1$ of which match
      $P_1$ through $P_{e-1}$, respectively,
      whose next $m-k$ forms each match $P_e$,
      whose remaining $n-m$ forms match $P_{m+1}$ through $P_n$; or

\item $P$ is of the form
      {\cf ($P_1$ $\dots$ $P_{e-1}$ $P_{e}$ \meta{ellipsis} $P_{m+1}$ \dotsfoo{} $P_n$ . $P_x$)}
      where $F$ is
      an list or improper list of $n$ forms, the first $e-1$ of which match
      $P_1$ through $P_{e-1}$,
      whose next $m-k$ forms each match $P_e$,
      whose remaining $n-m$ forms match $P_{m+1}$ through $P_n$,
      and whose $n$th and final cdr matches $P_x$; or

\item $P$ is a vector of the form {\cf \#($P_1$ $\dots$ $P_n$)}
      and $F$ is a vector
      of $n$ forms that match $P_1$ through $P_n$; or

\item $P$ is of the form
      {\cf \#($P_1$ $\dots$ $P_{e-1}$ $P_{e}$ \meta{ellipsis} $P_{m+1}$ \dotsfoo $P_n$)}
      where $F$ is a vector of $n$
      forms the first $e-1$ of which match $P_1$ through $P_{e-1}$,
      whose next $m-k$ forms each match $P_e$,
      and whose remaining $n-m$ forms matche $P_{m+1}$ through $P_n$; or

\item $P$ is a datum and $F$ is equal to $P$ in the sense of
      the {\cf equal?} procedure.
\end{itemize}

It is an error to use a macro keyword, within the scope of its
binding, in an expression that does not match any of the patterns.

When a macro use is transcribed according to the template of the
matching \hyper{syntax rule}, pattern variables that occur in the
template are replaced by the subforms they match in the input.
Pattern variables that occur in subpatterns followed by one or more
instances of the identifier
\hyper{ellipsis} are allowed only in subtemplates that are
followed by as many instances of \hyper{ellipsis}.
They are replaced in the
output by all of the subforms they match in the input, distributed as
indicated.  It is an error if the output cannot be built up as
specified.

%%% This description of output construction is very vague.  It should
%%% probably be formalized, but that is not easy...

Identifiers that appear in the template but are not pattern variables
or the identifier
\hyper{ellipsis} are inserted into the output as literal identifiers.  If a
literal identifier is inserted as a free identifier then it refers to the
binding of that identifier within whose scope the instance of
{\cf syntax-rules} appears.
If a literal identifier is inserted as a bound identifier then it is
in effect renamed to prevent inadvertent captures of free identifiers.

A template of the form
{\cf (\hyper{ellipsis} \hyper{template})} is identical to \hyper{template},
except that
ellipses within the template have no special meaning.
That is, any ellipses contained within \hyper{template} are
treated as ordinary identifiers.
In particular, the template {\cf (\hyper{ellipsis} \hyper{ellipsis})} produces
a single \hyper{ellipsis}.
This allows syntactic abstractions to expand into forms containing
ellipses.

\begin{scheme}
(define-syntax be-like-begin
  (syntax-rules ()
    ((be-like-begin name)
     (define-syntax name
       (syntax-rules ()
         ((name expr (... ...))
          (begin expr (... ...))))))))

(be-like-begin sequence)
(sequence 1 2 3 4) \ev 4%
\end{scheme}

As an example, if \ide{let} and \ide{cond} are defined as in
section~\ref{derivedsection} then they are hygienic (as required) and
the following is not an error.

\begin{scheme}
(let ((=> \schfalse))
  (cond (\schtrue => 'ok)))           \ev ok%
\end{scheme}

The macro transformer for {\cf cond} recognizes {\cf =>}
as a local variable, and hence an expression, and not as the
top-level identifier {\cf =>}, which the macro transformer treats
as a syntactic keyword.  Thus the example expands into

\begin{scheme}
(let ((=> \schfalse))
  (if \schtrue (begin => 'ok)))%
\end{scheme}

instead of

\begin{scheme}
(let ((=> \schfalse))
  (let ((temp \schtrue))
    (if temp ('ok temp))))%
\end{scheme}

which would result in an invalid procedure call.

\end{entry}

\subsection{Signalling errors in macros}
\label{syntax-error}

% Is it better to describe this as part of the template language?

\begin{entry}{%
\pproto{(syntax-error \hyper{message} \hyper{args} \dotsfoo)}{\exprtype}}

{\cf syntax-error} is similar to {\cf error} except that implementations
with an expansion pass separate from evaluation should signal an error
as soon as the {\cf syntax-error} form is expanded.  This can be used as
a {\cf syntax-rules} \hyper{template} for a \hyper{pattern} that is
an invalid use of the macro, which can provide more descriptive error
messages.  \hyper{message} should be a string literal, and \hyper{args}
arbitrary forms providing additional information.

\todo{Shinn: This doesn't check all non-identifier cases, think of a better example.}

\begin{scheme}
(define-syntax simple-let
  (syntax-rules ()
    ((\_ (head ... ((x . y) val) . tail)
        body1 body2 ...)
     (syntax-error
      "expected an identifier but got"
      (x . y)))
    ((\_ ((name val) ...) body1 body2 ...)
     ((lambda (name ...) body1 body2 ...)
       val ...))))
\end{scheme}

\end{entry}
