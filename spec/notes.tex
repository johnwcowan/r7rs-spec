\extrapart{Notes}


\subsection*{Language changes}
\label{differences}

This section enumerates the changes that have been made to Scheme since
the ``Revised$^5$ report''~\cite{R5RS} was published.

{\em The list is incomplete and subject to change while this report is in draft status.}

\begin{itemize}

\item Various minor ambiguities and unclarities in R5RS have been cleaned up.

\item Libraries have been added as a new program structure to improve
encapsulation and sharing of code.  Some existing and new identifiers
have been factored out into separate libraries.
Libraries can be imported into other libraries or main programs, with
controlled exposure and renaming of identifiers.
The contents of a library can be made conditional on the features of
the implementation it is to be used on.

\item Exceptions can now be signalled explicitly with {\cf raise},
{\cf raise-continuable} or {\cf error}, and can be handled with {\cf
with-exception-handler} and the {\cf guard} syntax.
Any object can specify an error condition; the implementation-defined
conditions signalled by {\cf error} have accessor functions to
retrieve the arguments passed to {\cf error}.

\item New disjoint types supporting access to multiple fields can be
generated with SRFI 9's {\cf define-record-type}.

\item Parameter objects can be created with {\cf make-parameter}, and
dynamically rebound with {\cf parameterize}.

\item {\em Bytevectors}, homogeneous vectors of integers in the range
$[0..255]$, have been added as a new disjoint type.
A subset of the procedures available for vectors is provided.  Bytevectors
can be converted to and from strings in accordance with the UTF-8 encoding.
Bytevectors have a datum representation which evaluates to itself.

\item {\cf Read-line} is provided to make line-oriented textual input
simpler.

\item {\em Ports} can now be designated as {\em textual} or {\em
binary} ports, with new procedures for reading and writing binary
data.
The new predicate {\cf port-open?} returns whether a port is open or closed.

\item {\em String ports} have been added as a way to read and write
characters to and from strings, and {\em bytevector ports} to read
and write bytes to and from bytevectors.

\item {\cf Current-input-port} and {\cf current-output-port} are now
parameter objects, along with the newly introduced {\cf
current-error-port}.

\item {\cf Syntax-rules} now recognizes {\em \_} as a wildcard, allows
the ellipsis symbol to be specified explicitly instead of the default
{\cf ...}, allows template escapes with an ellipsis-prefixed list, and
allows tail patterns to follow an ellipsis pattern.

\item {\cf Syntax-error} has been added as a way to signal immediate
and more informative errors when a macro is expanded.

\item Internal {\cf define-syntax} definitions are now allowed wherever
internal {\cf define}s are.

\item {\cf Letrec*} has been added, and internal {\cf define} specified in
terms of it.

\item Support for capturing multiple values has been enhanced with {\cf
define-values}, {\cf let-values}, and {\cf let*-values}.
Programs are now explicitly permitted to pass zero or more than one
value to continuations which discard them.

\item {\cf Case} now supports a {\tt =>} syntax analogous to {\cf cond}.

\item {\cf Case-lambda} has been added in its own library as a way to
dispatch on the number of arguments passed to a procedure.

\item {\cf When} and {\cf unless} have been added as convenience
conditionals.

\item Positive, negative infinity and a NaN object, and negative inexact zero have been added
to the numeric tower as inexact values with the written
representations {\tt +inf.0}, {\tt -inf.0}, {\tt +nan.0}, and {\cf -0.0}
respectively.

\item {\cf Map} and {\cf for-each} are now required to terminate on
the shortest list when inputs have different length.

\item {\cf Member} and {\cf assoc} now take an optional third argument
for the equality predicate to use.

\item {\cf Exact-integer?}\  and {\cf exact-integer-sqrt} have been added.

\item The procedures {\cf make-list}, {\cf list-copy}, {\cf list-set!}, {\cf
string-map}, {\cf string-for-each}, {\cf string->vector}, {\cf
vector-copy}, {\cf vector-map}, {\cf vector-for-each}, and {\cf
vector->string} have been added to round out the sequence operations.

\item The set of characters used is required to be consistent with the
Unicode Standard, 
but implementations are not required to support the whole character repertoire.
Various character and string procedures have been extended accordingly.
String comparison remains implementation-dependent, and is no longer
required to be consistent with character comparison, which is based
on Unicode code points.
The new {\cf digit-value} procedure is added to obtain the numerical
value of a numeric character.

\item {\cf String-ni=?} and related procedures have been added to
compare strings as though they had gone through an
implementation-defined normalization, without exposing the
normalization.

\item The case-folding behavior of the reader can now be explicitly
controlled, with no folding as the default.

\item The reader now recognizes the comment syntaxes {\tt \#;} to
skip the next datum, and {\tt \#| ... |\#}
for nestable block comments.

\item Data prefixed with reader labels {\tt \#<n>=} can be referenced
with {\tt \#<n>\#} allowing for reading and writing of data with
shared structure.

\item Strings and symbols now allow mnemonic and numeric escape
sequences, and the list of named characters has been extended.

\item {\cf File-exists?}\ and {\cf delete-file} are available in the
{\tt (scheme file)} library.

\item An interface to the system environment and command line is
available in the {\tt (scheme process-context)} library.

\item Procedures for accessing the current time are available in the
{\tt (scheme time)} library.

\item A complete set of integer division operators is available in the
{\tt (scheme division)} library.

\item {\cf Load} now accepts a second argument specifying the environment to
load into.

\item {\cf Transcript-on} and {\cf transcript-off} have been removed.

\item The semantics of read-eval-print loops are now partly prescribed,
allowing the retroactive redefinition of procedures but not syntax forms.

\end{itemize}

