\extrapart{Language changes}


\subsection*{Incompatibilities with \rfivers}
\label{incompatibilities}

This section enumerates the incompatibilities between this report and
the ``Revised$^5$ report''~\cite{R5RS}.

{\em This list is not authoritative, but is believed to be correct and complete.}


\begin{itemize}

\item Case sensitivity is now the default in symbols and character names.
This means that code written under the assumption that symbols could be
written {\cf FOO} or {\cf Foo} in some contexts and {\cf foo} in other contexts
can either be changed, be marked with the new {\cf \#!fold-case} directive,
or be included in a library using the {\cf include-ci} library declaration.
All standard identifiers are entirely in lower case.

\item The {\cf syntax-rules} construct now recognizes {\em \_} (underscore) 
as a wildcard, which means it cannot be used as a syntax variable.
It can still be used as a literal.

\item The \rfivers\ procedures {\cf exact->inexact} and {\cf inexact->exact}
have been renamed to their \rsixrs\ names, {\cf inexact} and {\cf exact},
respectively, as these names are shorter and more correct.
The former names are still available in the \rfivers\ library.

\item The guarantee that string comparison (with {\cf string<?} and the
related predicates) is a lexicographical extension of character comparison
(with {\cf char<?} and the related predicates) has been removed.  

\item Support for the \# character in numeric literals is no longer required.

\item Support for the letters {\cf s}, {\cf f}, {\cf d}, and {\cf l}
as exponent markers is no longer required.

\item Implementations of {\cf string\coerce{}number} are no longer permitted
to return \schfalse{} when the argument contains an explicit radix prefix.

\item The procedures {\cf transcript-on} and {\cf transcript-off} have been removed.

\end{itemize}

\subsection*{Other language changes since \rfivers}
\label{differences}
This section enumerates the additional differences between this report and
the ``Revised$^5$ report''~\cite{R5RS}.

{\em This list is not authoritative, but is believed to be correct and complete.}

\begin{itemize}

\item Various minor ambiguities and unclarities in \rfivers\ have been cleaned up.

\item Libraries have been added as a new program structure to improve
encapsulation and sharing of code.  Some existing and new identifiers
have been factored out into separate libraries.
Libraries can be imported into other libraries or main programs, with
controlled exposure and renaming of identifiers.
The contents of a library can be made conditional on the features of
the implementation on which it is to be used.
There is an \rfivers\ compatibility library.

\item Exceptions can now be signaled explicitly with {\cf raise},
{\cf raise-continuable} or {\cf error}, and can be handled with {\cf
with-exception-handler} and the {\cf guard} syntax.
Any object can specify an error condition; the implementation-defined
conditions signaled by {\cf error} have a predicate to detect them and accessor functions to
retrieve the arguments passed to {\cf error}.
Conditions signaled by {\cf read} and by file-related procedures
also have predicates to detect them.

\item New disjoint types supporting access to multiple fields can be
generated with SRFI 9's {\cf define-record-type}.

\item Parameter objects can be created with {\cf make-parameter}, and
dynamically rebound with {\cf parameterize}.
The procedures {\cf current-input-port} and {\cf current-output-port} are now
parameter objects, as is the newly introduced {\cf current-error-port}.

\item {\em Bytevectors}, vectors of exact integers in the range
$[0, 255]$, have been added as a new disjoint type.
A subset of the vector procedures is provided.  Bytevectors
can be converted to and from strings in accordance with the UTF-8 character encoding.
Bytevectors have a datum representation and evaluate to themselves.

\item Vector constants evaluate to themselves.

\item The procedure {\cf read-line} is provided to make line-oriented textual input
simpler.

\item The procedure {\cf flush-output-port} is provided to allow minimal
control of output port buffering.

\item {\em Ports} can now be designated as {\em textual} or {\em
binary} ports, with new procedures for reading and writing binary
data.
The new predicates {\cf input-port-open?} and {\cf output-port-open?} return whether a port is open or closed.
The new procedure {\cf close-port} now closes a port; if the port
has both input and output sides, both are closed.

\item {\em String ports} have been added as a way to read and write
characters to and from strings, and {\em bytevector ports} to read
and write bytes to and from bytevectors.

\item There are now I/O procedures specific to strings and bytevectors.

\item The {\cf write} procedure now generates datum labels when applied to
circular objects.  The new procedure {\cf write-simple} never generates
labels; {\cf write-shared} generates labels for all shared and circular
structure.

\item The \rsixrs\ procedure {\cf eof-object} has been added.

\item Syntax definitions are now allowed wherever variable definitions are.

\item The {\cf syntax-rules} construct now allows
the ellipsis symbol to be specified explicitly instead of the default
{\cf ...}, allows template escapes with an ellipsis-prefixed list, and
allows tail patterns to follow an ellipsis pattern.

\item The {\cf syntax-error} syntax has been added as a way to signal immediate
and more informative errors when a macro is expanded.

\item The {\cf letrec*} binding construct has been added, and internal {\cf define} 
is specified in terms of it.

\item Support for capturing multiple values has been enhanced with {\cf
define-values}, {\cf let-values}, and {\cf let*-values}.
Standard forms which introduce a body now permit passing zero or more than
one value to the continuations of all non-final forms of the body.

\item The {\cf case} conditional now supports {\tt =>} syntax
analogous to {\cf cond} not only in regular clauses but in the {\cf
else} clause as well.

\item To support dispatching on the number of arguments passed to a
procedure, {\cf case-lambda} has been added in its own library.

\item The convenience conditionals {\cf when} and {\cf unless} have been added.

\item The behavior of both {\cf eq?} and {\cf eqv?} on procedures is now unspecified.

\item The behavior of {\cf eqv?} on inexact numbers now conforms to the
\rsixrs\ definition.

\item The \rsixrs\ procedures {\cf boolean=?} and {\cf symbol=?} have been added.

\item Positive infinity, negative infinity, NaN, and negative inexact zero have been added
to the numeric tower as inexact values with the written
representations {\tt +inf.0}, {\tt -inf.0}, {\tt +nan.0}, and {\cf -0.0}
respectively.  Support for them is not required.
The representation {\tt -nan.0} is synonymous with {\tt +nan.0}.

\item The {\cf log} procedure now accepts a second argument specifying
the logarithm base.

\item The procedures {\cf map} and {\cf for-each} are now required to terminate on
the shortest argument list.

\item The procedures {\cf member} and {\cf assoc} now take an optional third argument
specifying the equality predicate to be used.

\item The numeric procedures {\cf finite?}, {\cf infinite?}, {\cf nan?},
{\cf exact-integer?}, {\cf square}, and {\cf exact-integer-sqrt}
have been added.

\item The procedures {\cf -}, {\cf /}, and the character and string comparison
procedures are now required to support an arbitrary number of arguments.

\item The procedures {\cf make-list}, {\cf list-copy}, {\cf list-set!},
{\cf string-map}, {\cf string-for-each}, {\cf string->vector}, 
{\cf vector-append},
{\cf vector-copy}, {\cf vector-map}, {\cf vector-for-each}, 
{\cf vector->string}, {\cf vector-copy!}, and {\cf string-copy!}
have been added to round out the sequence operations.

\item Some string and vector procedures support processing of part of a string or vector using
optional \var{start} and \var{end} arguments.

\item Some list procedures are now defined on circular lists.

\item Implementations may provide any subset of the full Unicode
repertoire that includes ASCII, but implementations must support any
such subset in a way consistent with Unicode.
Various character and string procedures have been extended accordingly.
String comparison is no longer
required to be consistent with character comparison, which is based
solely on Unicode scalar values.
The new {\cf digit-value} procedure has been added to obtain the numerical
value of a numeric character.

\item There are now two additional comment syntaxes: {\tt \#;} to
skip the next datum, and {\tt \#| ... |\#}
for nestable block comments.

\item Data prefixed with datum labels {\tt \#<n>=} can be referenced
with {\tt \#<n>\#}, allowing for reading and writing of data with
shared structure.

\item Strings and symbols now allow mnemonic and numeric escape
sequences, and the list of named characters has been extended.

\item The procedures {\cf file-exists?}\ and {\cf delete-file} are available in the
{\tt (scheme file)} library.

\item An interface to the system environment, command line, and process exit status is
available in the {\tt (scheme process-context)} library.

\item Procedures for accessing time-related values are available in the
{\tt (scheme time)} library.

\item A less irregular set of integer division operators is provided
with new and clearer names.

\item The {\cf load} procedure now accepts a second argument specifying the environment to
load into.

\item The {\cf call-with-current-continuation} procedure now has the synonym
{\cf call/cc}.

\item The semantics of read-eval-print loops are now partly prescribed,
requiring the redefinition of procedures, but not syntax keywords, to have retroactive effect.

\end{itemize}

\subsection*{Incompatibilities with \rsixrs}
This section enumerates the incompatibilities between \rsevenrs~and
the ``Revised$^6$ report''~\cite{R6RS} and its accompanying Standard Libraries document.

{\em This list is not authoritative, and is possibly incomplete.}

\begin{itemize}
\item \rsevenrs\ libraries begin with the keyword {\cf define-library}
rather than {\cf library} in order to make them syntactically
distinguishable from \rsixrs\ libraries.
In \rsevenrs\ terms, the body of an \rsixrs\ library consists
of a single export declaration followed by a single import declaration,
followed by commands and definitions.  In \rsevenrs, commands and
definitions are not permitted directly within the body: they have to be be wrapped in a {\cf begin}
library declaration.

\item There is no direct \rsixrs\ equivalent of the {\cf include}, {\cf include-ci},
{\cf include-library-declarations}, or {\cf cond-expand} library declarations.
On the other hand, the \rsevenrs\ library syntax does not support phase or version specifications.

\item The grouping of standardized identifiers into libraries is different from the \rsixrs\
approach. In particular, procedures which are optional in \rfivers\, either expressly
or by implication, have been removed from the base library.
Only the base library itself is an absolute requirement.

\item No form of identifier syntax is provided. 

\item Internal syntax definitions are allowed, but uses of a syntax form
cannot appear before its definition; the {\cf even}/{\cf odd} example given in
\rsixrs\ is not allowed.

\item The \rsixrs\ exception system was incorporated as-is, but the condition
types have been left unspecified.  In particular, where \rsixrs\ requires
a condition of a specified type to be signaled, \rsevenrs\ says only
``it is an error'', leaving the question of signaling open.

\item Full Unicode support is not required.
Normalization is not provided.
Character comparisons are
defined by Unicode, but string comparisons are implementation-dependent.
Non-Unicode characters are permitted.

\item The full numeric tower is optional as in \rfivers, but optional support for IEEE
infinities, NaN, and {\mbox -0.0} was adopted from \rsixrs. Most clarifications on
numeric results were also adopted, but the \rsixrs\ procedures {\cf real-valued?},
{\cf rational-valued?}, and {\cf integer-valued}? were not. 
The \rsixrs\ division operators {\cf div}, {\cf mod}, {\cf div-and-mod}, {\cf
div0}, {\cf mod0} and {\cf div0-and-mod0} are not provided.

\item When a result is unspecified, it is still required to be a single value.
However, non-final expressions
in a body can return any number of values.

\item The semantics of {\cf map} and {\cf for-each} have been changed to use
the SRFI 1 early termination behavior. Likewise
{\cf assoc} and {\cf member} take an optional {\cf equal?} argument as in SRFI 1,
instead of the separate {\cf assp} and {\cf memp} procedures of \rsixrs.

\item The \rsixrs~{\cf quasiquote} clarifications have been adopted, with the
exception of multiple-argument {\cf unquote} and
{\cf unquote-splicing}.

\item The \rsixrs~method of specifying mantissa widths was not adopted.

\item String ports are compatible with SRFI 6 rather than \rsixrs.

\item \rsixrs{}-style bytevectors are included, but 
only the unsigned byte ({\cf u8}) procedures have been provided.
The lexical syntax uses {\cf \#u8} for compatibility
with SRFI 4, rather than the \rsixrs~{\cf \#vu8} style.

\item The utility macros {\cf when} and {\cf unless} are provided, but 
their result is left unspecified.

\item The remaining features of the Standard Libraries document were
left to future standardization efforts.

\end{itemize}

