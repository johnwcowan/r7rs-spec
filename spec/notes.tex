\extrapart{Notes}


\subsection*{Language changes since R5RS}
\label{differences}
This section enumerates the differences between this report and
the ``Revised$^5$ report''~\cite{R5RS}.

{\em The list is incomplete and subject to change while this report is in draft status.}


\begin{itemize}

\item Various minor ambiguities and unclarities in R5RS have been cleaned up.

\item Libraries have been added as a new program structure to improve
encapsulation and sharing of code.  Some existing and new identifiers
have been factored out into separate libraries.
Libraries can be imported into other libraries or main programs, with
controlled exposure and renaming of identifiers.
The contents of a library can be made conditional on the features of
the implementation it is to be used on.

\item Exceptions can now be signalled explicitly with {\cf raise},
{\cf raise-continuable} or {\cf error}, and can be handled with {\cf
with-exception-handler} and the {\cf guard} syntax.
Any object can specify an error condition; the implementation-defined
conditions signalled by {\cf error} have accessor functions to
retrieve the arguments passed to {\cf error}.

\item New disjoint types supporting access to multiple fields can be
generated with SRFI 9's {\cf define-record-type}.

\item Parameter objects can be created with {\cf make-parameter}, and
dynamically rebound with {\cf parameterize}.

\item {\em Bytevectors}, homogeneous vectors of integers in the range
$[0..255]$, have been added as a new disjoint type.
A subset of the procedures available for vectors is provided.  Bytevectors
can be converted to and from strings in accordance with the UTF-8 encoding.
Bytevectors have a datum representation which evaluates to itself.

\item {\cf Read-line} is provided to make line-oriented textual input
simpler.

\item {\em Ports} can now be designated as {\em textual} or {\em
binary} ports, with new procedures for reading and writing binary
data.
The new predicate {\cf port-open?} returns whether a port is open or closed.

\item {\em String ports} have been added as a way to read and write
characters to and from strings, and {\em bytevector ports} to read
and write bytes to and from bytevectors.

\item {\cf Current-input-port} and {\cf current-output-port} are now
parameter objects, along with the newly introduced {\cf
current-error-port}.

\item {\cf Syntax-rules} now recognizes {\em \_} as a wildcard, allows
the ellipsis symbol to be specified explicitly instead of the default
{\cf ...}, allows template escapes with an ellipsis-prefixed list, and
allows tail patterns to follow an ellipsis pattern.

\item {\cf Syntax-error} has been added as a way to signal immediate
and more informative errors when a macro is expanded.

\item Internal {\cf define-syntax} definitions are now allowed wherever
internal {\cf define}s are.

\item {\cf Letrec*} has been added, and internal {\cf define} specified in
terms of it.

\item Support for capturing multiple values has been enhanced with {\cf
define-values}, {\cf let-values}, and {\cf let*-values}.
Programs are now explicitly permitted to pass zero or more than one
value to continuations which discard them.

\item {\cf Case} now supports a {\tt =>} syntax analogous to {\cf cond}.

\item {\cf Case-lambda} has been added in its own library as a way to
dispatch on the number of arguments passed to a procedure.

\item {\cf When} and {\cf unless} have been added as convenience
conditionals.

\item Positive, negative infinity and a NaN object, and negative inexact zero have been added
to the numeric tower as inexact values with the written
representations {\tt +inf.0}, {\tt -inf.0}, {\tt +nan.0}, and {\cf -0.0}
respectively.

\item {\cf Map} and {\cf for-each} are now required to terminate on
the shortest list when inputs have different length.

\item {\cf Member} and {\cf assoc} now take an optional third argument
for the equality predicate to use.

\item {\cf Exact-integer?}\  and {\cf exact-integer-sqrt} have been added.

\item The procedures {\cf make-list}, {\cf list-copy}, {\cf list-set!}, {\cf
string-map}, {\cf string-for-each}, {\cf string->vector}, {\cf
vector-copy}, {\cf vector-map}, {\cf vector-for-each}, and {\cf
vector->string} have been added to round out the sequence operations.

\item The set of characters used is required to be consistent with the
Unicode Standard, 
but implementations are not required to support the whole character repertoire.
Various character and string procedures have been extended accordingly.
String comparison remains implementation-dependent, and is no longer
required to be consistent with character comparison, which is based
on Unicode code points.
The new {\cf digit-value} procedure is added to obtain the numerical
value of a numeric character.

\item {\cf String-ni=?} and related procedures have been added to
compare strings as though they had gone through an
implementation-defined normalization, without exposing the
normalization.

\item The case-folding behavior of the reader can now be explicitly
controlled, with no folding as the default.

\item The reader now recognizes the comment syntaxes {\tt \#;} to
skip the next datum, and {\tt \#| ... |\#}
for nestable block comments.

\item Data prefixed with reader labels {\tt \#<n>=} can be referenced
with {\tt \#<n>\#} allowing for reading and writing of data with
shared structure.

\item Strings and symbols now allow mnemonic and numeric escape
sequences, and the list of named characters has been extended.

\item {\cf File-exists?}\ and {\cf delete-file} are available in the
{\tt (scheme file)} library.

\item An interface to the system environment and command line is
available in the {\tt (scheme process-context)} library.

\item Procedures for accessing the current time are available in the
{\tt (scheme time)} library.

\item A complete set of integer division operators is available in the
{\tt (scheme division)} library.

\item {\cf Load} now accepts a second argument specifying the environment to
load into.

\item {\cf Transcript-on} and {\cf transcript-off} have been removed.

\item The current system time is provided using TAI time.

\item The semantics of read-eval-print loops are now partly prescribed,
allowing the retroactive redefinition of procedures but not syntax forms.

\end{itemize}

\subsection*{Incompatibilities with the main R6RS report}
This section enumerates the incompatibilities between R7RS and
the ``Revised$^6$ report''~\cite{R6RS} was published.

{\em The list is incomplete and subject to change while this report is in draft status.}

\begin{itemize}
\item The syntax of the module system was deliberately chosen to be
syntactically different from R6RS, using {\cf define-library} instead of
{\cf library} in order to allow easy disambiguation between R6RS and R7RS libraries.

\item The module system does not support phase distinctions, which
are unnecessary in the absence of low-level macros (see below),
nor versioning, which is an important feature but deserves more
experimentation before standardizing.

\item Putting an extra level of indirection around the library body
allows room for extensibility. The R6RS syntax provides two positional
forms which must be present and must have the correct keywords,
{\cf export} and {\cf import}, which does not allow for unambiguous
extensions. The working group considers extensibility to be important,
and so chose a syntax which provides
a clear separation between the module declarations and the Scheme code
which makes up the body.

\item The {\cf include} library form
makes it easier to include separate files, 
and the {\cf include-ci} variant allows legacy 
case-insensitive code to be incorporated.

\item The {\cf cond-expand} form from SRFI 0 allows for a more
deterministic alternative to the R6RS {\cf .impl.sls} file naming
convention.

\item Since the R7RS module system is straightforward, it is expected
that R6RS implementations will be able to support the {\cf define-library}
syntax in addition to their {\cf library} syntax.

\item The modularization of standardized identifiers is different from the R6RS
approach. In particular, procedures which are optional either expressly
or by implication in R5RS have been removed from the base module.
Only the base module is an absolute requirement.

\item Identifier syntax is not provided. This is a useful feature in
some situations, but the existence of such macros means that neither
programmers nor other macros can look at an identifier in an evaluated
position and know it is a reference --- this in a sense makes all macros
slightly weaker. Individual implementations are encouraged to continue
experimenting with this and other extensions before further standardization is done.

\item Internal syntax definitions are allowed, but all references to syntax
must follow the definition --- the {\cf even}/{\cf odd} example given is
R6RS is not allowed.

\item The R6RS exception system was incorporated as-is, but the condition
types have been left unspecified. Specific errors that must be signalled
in R6RS remain ``an error'' in R7RS, allowing implementations to provide
their own extensions.  The condition system and stricter semantics may
reappear in the large language or a later SRFI or standard. There is no
discussion of safety.

\item Full Unicode support is not required, but implementations are required to
implement whatever characters they do support in a manner
consistent with Unicode.  Case
conversions are described in terms of the Unicode locale-independent
mappings, and instead of explicit normalization forms we provide 
normalization-insensitive string comparisons that use
an implementation-defined normalization form
(which may be the identity transformation). Character comparisons are
defined by Unicode, but string comparisons are implementation-dependent,
and therefore need not be the lexicographic mapping of the corresponding
character comparisons (an incompatibility with R5RS). Non-Unicode
characters are permitted.

\item The full numeric tower is optional as in R5RS, but optional support for IEEE
infinities, NaN, and -0.0 was adopted from R6RS. Most clarifications on
numeric results were also adopted, but the R6RS procedures {\cf real-valued?},
{\cf rational-valued?}, and {\cf integer-valued}? were not. The R5RS names
{\cf inexact->exact} for {\cf exact} and {\cf exact->inexact} for {\cf inexact} were retained,
with a note indicating that their names are historical.
The R6RS division operators {\cf div}, {\cf mod}, {\cf div-and-mod}, {\cf
div0}, {\cf mod0} and {\cf div0-and-mod0} have been replaced with a full
set of 18 operators describing 6 different rounding semantics.

\item When a result is unspecified, it is still required to be a single value,
in the interests of R5RS compatibility. However, non-final expressions
in a body may return any number of values.

\item Because of widespread SRFI 1 support and extensive code
that uses it, the semantics of {\cf map} and {\cf for-each} have been changed to use
the SRFI 1 early termination behavior. Likewise
{\cf assoc} and {\cf member} take an optional {\cf equal?} argument as in SRFI 1,
replacing the separate {\cf assp} and {\cf memp} procedures from R6RS.

\item The R6RS {\cf quasiquote} clarifications have been adopted, but the working group has not seen
convincing enough examples to allow multiple-argument {\cf unquote} and
{\cf unquote-splicing}.

\item The R6RS method of specifying mantissa widths was not adopted.

\end{itemize}

\subsection*{Incompatibilities with the R6RS library report}

This section enumerates the incompatibilities between R7RS and
the ``Revised$^6$ library report''~\cite{R6RS-lib} was published.

{\em The list is incomplete and subject to change while this report is in draft status.}

\begin{itemize}

\item The low-level macro system and {\cf syntax-case} were not adopted. There
are two general families of macro systems in widespread use --- the
{\cf {\cf syntax-case} family and the syntactic-closures} family --- and they have
neither been shown to be equivalent nor capable of implementing each
other. Given this situation,
low-level macros have been left to the large language.

\item The new I/O system from R6RS was not adopted. We feel a completely new
system deserves a period of usage before being standardized, and were
unhappy with the backwards-compatibility ``simple I/O'' which introduced
a redundant API and relegated R5RS code to being a second-class
citien.  Instead, binary ports were made at least potentially disjoint from
character ports, using their own procedures.

\item String ports are compatible with SRFI 6 rather than R6RS; analogous
bytevector ports are also provided.

\item The working group felt that the R6RS records system was overly complex, and the two layers
poorly integrated. We spent a lot of time debating this, but in the
end decided to simply use a generative version of SRFI 9, which has
near-universal support among implementations. We hope to provide a more
powerful records system in the large language.

\item Enumerations are not included in the small language.

\item R6RS bytevectors are included, but provide only the "u8" procedures in the small
language.  The lexical syntax uses {\cf \#u8} for compatibility
with SRFI 4, rather than the R6RS {\cf \#vu8} style.
With a module system it's easier to change names than reader syntax.

\item The utility macros {\cf when} and {\cf unless} are provided, but since it would be
meaningless to try to use their result, it is left unspecified.

\item The working group could not agree on many issues with hash tables and have left them for
the large language. We've also left sorting and bitwise arithmetic to WG2.

\item Pair and string mutation are too well-established to be relegated to
separate modules.

\end{itemize}

