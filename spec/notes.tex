\extrapart{Notes}


\subsection*{Language changes since \rfivers}
\label{differences}
This section enumerates the differences between this report and
the ``Revised$^5$ report''~\cite{R5RS}.

{\em The list is incomplete and subject to change while this report is in draft status.}


\begin{itemize}

\item Various minor ambiguities and unclarities in \rfivers\ have been cleaned up.

\item Libraries have been added as a new program structure to improve
encapsulation and sharing of code.  Some existing and new identifiers
have been factored out into separate libraries.
Libraries can be imported into other libraries or main programs, with
controlled exposure and renaming of identifiers.
The contents of a library can be made conditional on the features of
the implementation on which it is to be used.

\item Exceptions can now be signalled explicitly with {\cf raise},
{\cf raise-continuable} or {\cf error}, and can be handled with {\cf
with-exception-handler} and the {\cf guard} syntax.
Any object can specify an error condition; the implementation-defined
conditions signalled by {\cf error} have accessor functions to
retrieve the arguments passed to {\cf error}.

\item New disjoint types supporting access to multiple fields can be
generated with SRFI 9's {\cf define-record-type}.

\item Parameter objects can be created with {\cf make-parameter}, and
dynamically rebound with {\cf parameterize}.

\item {\em Bytevectors}, homogeneous vectors of integers in the range
$[0..255]$, have been added as a new disjoint type.
A subset of the procedures available for vectors is provided.  Bytevectors
can be converted to and from strings in accordance with the UTF-8 character encoding.
Bytevectors have a datum representation and evaluate to themselves.

\item The procedure {\cf read-line} is provided to make line-oriented textual input
simpler.

\item {\em Ports} can now be designated as {\em textual} or {\em
binary} ports, with new procedures for reading and writing binary
data.
The new predicate {\cf port-open?} returns whether a port is open or closed.

\item {\em String ports} have been added as a way to read and write
characters to and from strings, and {\em bytevector ports} to read
and write bytes to and from bytevectors.

\item The procedures {\cf current-input-port} and {\cf current-output-port} are now
parameter objects, as is the newly introduced {\cf
current-error-port}.

\item The {\cf syntax-rules} special form now recognizes {\em \_} (underscore) as a wildcard, allows
the ellipsis symbol to be specified explicitly instead of the default
{\cf ...}, allows template escapes with an ellipsis-prefixed list, and
allows tail patterns to follow an ellipsis pattern.

\item The {\cf syntax-error} syntax has been added as a way to signal immediate
and more informative errors when a macro is expanded.

\item Internal {\cf define-syntax} definitions are now allowed wherever
internal {\cf define}s are.

\item The {\cf letrec*} special form has been added, and internal {\cf define} specified in
terms of it.

\item Support for capturing multiple values has been enhanced with {\cf
define-values}, {\cf let-values}, and {\cf let*-values}.
Programs are now explicitly permitted to pass zero or more than one
value to continuations which discard them.

\item The {\cf case} special form now supports a {\tt =>} syntax analogous to {\cf cond}.

\item The special form {\cf case-lambda} has been added in its own library as a way to
dispatch on the number of arguments passed to a procedure.

\item The special forms {\cf when} and {\cf unless} have been added as convenience
conditionals.

\item Positive infinity, negative infinity, NaN, and negative inexact zero have been added
to the numeric tower as inexact values with the written
representations {\tt +inf.0}, {\tt -inf.0}, {\tt +nan.0}, and {\cf -0.0}
respectively.

\item The procedures {\cf map} and {\cf for-each} are now required to terminate on
the shortest list when inputs have different length.

\item The procedures {\cf member} and {\cf assoc} now take an optional third argument
specifying the equality predicate to be used.

\item The procedures {\cf exact-integer?}\  and {\cf exact-integer-sqrt} have been added.

\item The procedures {\cf make-list}, {\cf list-copy}, {\cf list-set!}, {\cf
string-map}, {\cf string-for-each}, {\cf string->vector}, {\cf
vector-copy}, {\cf vector-map}, {\cf vector-for-each}, and {\cf
vector->string} have been added to round out the sequence operations.

\item Implementations may provide any subset of the full Unicode
repertoire that includes ASCII, but implementations must support any
such subset in a way consistent with Unicode.
Various character and string procedures have been extended accordingly.
String comparison remains implementation-dependent, and is no longer
required to be consistent with character comparison, which is based
on Unicode code points.
The new {\cf digit-value} procedure is added to obtain the numerical
value of a numeric character.

\item The procedures {\cf string-ni=?} and related procedures have been added to
compare strings as though they had gone through an
implementation-defined normalization, without exposing the
normalization.

\item The case-folding behavior of {\cf read} can now be explicitly
controlled, with no folding as the default.

\item There are now two additional comment syntaxes: {\tt \#;} to
skip the next datum, and {\tt \#| ... |\#}
for nestable block comments.

\item Data prefixed with datum labels {\tt \#<n>=} can be referenced
with {\tt \#<n>\#} allowing for reading and writing of data with
shared structure.

\item Strings and symbols now allow mnemonic and numeric escape
sequences, and the list of named characters has been extended.

\item The procedures {\cf file-exists?}\ and {\cf delete-file} are available in the
{\tt (scheme file)} library.

\item An interface to the system environment and command line is
available in the {\tt (scheme process-context)} library.

\item Procedures for accessing the current time are available in the
{\tt (scheme time)} library.

\item A complete set of integer division operators is available in the
{\tt (scheme division)} library.

\item The {\cf load} procedure now accepts a second argument specifying the environment to
load into.

\item The procedures {\cf transcript-on} and {\cf transcript-off} have been removed.

\item The semantics of read-eval-print loops are now partly prescribed,
allowing the retroactive redefinition of procedures but not syntax forms.

\end{itemize}

\subsection*{Incompatibilities with the main \rsixrs\ document}
This section enumerates the incompatibilities between \rsevenrs~and
the ``Revised$^6$ report''~\cite{R6RS}.

{\em The list is incomplete and subject to change while this report is in draft status.}

\begin{itemize}
\item The syntax of the library system was deliberately chosen to be
syntactically different from \rsixrs, using {\cf define-library} instead of
{\cf library} in order to allow easy disambiguation between \rsixrs\
and \rsevenrs\ libraries.

\item The library system does not support phase distinctions, which
are unnecessary in the absence of low-level macros (see below),
nor does it support versioning, which is an important feature but deserves more
experimentation before being standardized.

\item Putting an extra level of indirection around the library body
allows room for extensibility. The \rsixrs\ syntax provides two positional
forms which must be present and must have the correct keywords,
{\cf export} and {\cf import}, which does not allow for unambiguous
extensions. The Working Group considers extensibility to be important,
and so chose a syntax which provides
a clear separation between the library declarations and the Scheme code
which makes up the body.

\item The {\cf include} library form
makes it easier to include separate files, 
and the {\cf include-ci} variant allows legacy 
case-insensitive code to be incorporated.

\item The {\cf cond-expand} form from SRFI 0 allows for a more
deterministic alternative to the \rsixrs\ {\cf .impl.sls} file naming
convention.

\item Since the \rsevenrs\ library system is straightforward, we expect
that \rsixrs\ implementations will be able to support the {\cf define-library}
syntax in addition to their {\cf library} syntax.

\item The grouping of standardized identifiers into libraries is different from the \rsixrs\
approach. In particular, procedures which are optional either expressly
or by implication in \rfivers\ have been removed from the base library.
Only the base library is an absolute requirement.

\item Identifier syntax is not provided. This is a useful feature in
some situations, but the existence of such macros means that neither
programmers nor other macros can look at an identifier in an evaluated
position and know it is a reference --- this in a sense makes all macros
slightly weaker. Individual implementations are encouraged to continue
experimenting with this and other extensions before further standardization is done.

\item Internal syntax definitions are allowed, but all references to syntax
must follow the definition; the {\cf even}/{\cf odd} example given in
\rsixrs\ is not allowed.

\item The \rsixrs\ exception system was incorporated as is, but the condition
types have been left unspecified.  Specific errors that must be signalled
in \rsixrs\ remain errors in \rsevenrs, allowing implementations to provide
their own extensions.  There is no discussion of safety.

\item Full Unicode support is not required.
Instead of explicit normalization forms this report provides
normalization-insensitive string comparisons that use
an implementation-defined normalization form
(which may be the identity transformation). Character comparisons are
defined by Unicode, but string comparisons are implementation-dependent,
and therefore need not be the lexicographic mapping of the corresponding
character comparisons (an incompatibility with \rfivers). Non-Unicode
characters are permitted.

\item The full numeric tower is optional as in \rfivers, but optional support for IEEE
infinities, NaN, and {\mbox -0.0} was adopted from \rsixrs. Most clarifications on
numeric results were also adopted, but the \rsixrs\ procedures {\cf real-valued?},
{\cf rational-valued?}, and {\cf integer-valued}? were not. The \rfivers\ names
{\cf inexact->exact} for {\cf exact} and {\cf exact->inexact} for {\cf inexact} were retained,
with a note indicating that their names are historical.
The \rsixrs\ division operators {\cf div}, {\cf mod}, {\cf div-and-mod}, {\cf
div0}, {\cf mod0} and {\cf div0-and-mod0} have been replaced with a full
set of 18 operators describing 6 different rounding semantics.

\item When a result is unspecified, it is still required to be a single value,
in the interests of \rfivers\ compatibility. However, non-final expressions
in a body may return any number of values.

\item Because of widespread SRFI 1 support and extensive code
that uses it, the semantics of {\cf map} and {\cf for-each} have been changed to use
the SRFI 1 early termination behavior. Likewise
{\cf assoc} and {\cf member} take an optional {\cf equal?} argument as in SRFI 1,
instead of the separate {\cf assp} and {\cf memp} procedures from \rsixrs.

\item The \rsixrs~{\cf quasiquote} clarifications have been adopted, but the Working Group has not seen
convincing enough examples to allow multiple-argument {\cf unquote} and
{\cf unquote-splicing}.

\item The \rsixrs~method of specifying mantissa widths was not adopted.

\end{itemize}

\subsection*{Incompatibilities with the \rsixrs\ Standard Libraries document}

This section enumerates the incompatibilities between \rsevenrs\ and
the \rsixrs~\cite{R6RS} Standard Libraries.

{\em The list is incomplete and subject to change while this report is in draft status.}

\begin{itemize}

\item The low-level macro system and {\cf syntax-case} were not adopted. There
are two general families of macro systems in widespread use --- the
{\cf syntax-case} family and the {\cf syntactic-closures} family --- and they have
neither been shown to be equivalent nor capable of implementing each
other. Given this situation,
low-level macros have been left to the large language.

\item The new I/O system from \rsixrs was not adopted. The Working Group believes that a completely new
system deserves a period of usage before being standardized, and were
unhappy with the redundant
provision of both the new system and the \rfivers{}-compatible ``simple I/O''
system, which
relegated \rfivers\ code to being a second-class
citizen.  Instead, binary ports were made at least potentially disjoint from
textual ports, using their own parallel set of procedures.

\item String ports are compatible with SRFI 6 rather than \rsixrs; analogous
bytevector ports are also provided.

\item The Working Group felt that the \rsixrs{} records system was overly complex, and the two layers
poorly integrated. The Working Group spent a lot of time debating this, but in the
end decided to simply use a generative version of SRFI 9, which has
near-universal support among implementations. The Working Group hopes to provide a more
powerful records system in the large language.

\item Enumerations are not included in the small language.

\item \rsixrs{}-style bytevectors are included, but provide only the ``u8'' procedures in the small
language.  The lexical syntax uses {\cf \#u8} for compatibility
with SRFI 4, rather than the \rsixrs~{\cf \#vu8} style.
With a library system, it's easier to change names than reader syntax.

\item The utility macros {\cf when} and {\cf unless} are provided, but since it would be
meaningless to try to use their result, it is left unspecified.

\item The Working Group could not agree on a single design for hash tables and left them for
the large language. 

\item Sorting, bitwise arithmetic, and enumerations were not considered to be
sufficiently useful to include in the small language.  They will probably be
included in the large language.

\item Pair and string mutation are too well-established to be relegated to
separate libraries.

\end{itemize}

