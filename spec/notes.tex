\extrapart{Notes}


\subsection*{Language changes}
\label{differences}

This section enumerates the changes that have been made to Scheme since
the ``Revised$^5$ report''~\cite{R5RS} was published.

{\em While this report is in draft status the list should be considered incomplete and subject to change.}

\begin{itemize}

\item Modules have been added as a new program structure to improve
encapsulation and sharing of code.  Some existing and new identifiers
have been factored out into separate modules.

\item Exceptions can now be signalled explicitly with {\cf raise},
{\cf raise-continuable} or {\cf error}, and can be handled with {\cf
with-exception-handler} and the {\cf guard} syntax.

\item New disjoint types supporting access to multiple fields can be
generated with SRFI 9's {\cf define-record-type}.

\item Parameter objects can be created with {\cf make-parameter}, and
dynamically rebound with {\cf parameterize}.

\item {\em Bytevectors}, homogeneous vectors of integers in the range
$[0..255]$, have been added as a new disjoint type.

\item {\em Ports} can now be designated as {\em binary} or {\em
character} ports, with new procedures for reading and writing binary
data.

\item {\em String ports} have been added as a way to read and write
characters to and from strings, and {\em bytevector ports} to read
and write bytes to and from bytevectors.

\item {\cf Current-input-port} and {\cf current-output-port} are now
parameter objects, along with the newly introduced {\cf
current-error-port}.

\item {\cf Syntax-rules} now recognizes {\em \_} as a wildcard, allows
the ellipsis symbol to be specified explicitly instead of the default
{\cf ...}, allows template escapes with an ellipsis-prefixed list, and
allows tail patterns to follow an ellipsis pattern.

\item {\cf Syntax-error} has been added as a way to signal immediate
and more informative errors when a macro is expanded.

\item Internal {\cf define-syntax} definitions are now allowed wherever
internal {\cf define}s are.

\item {\cf Letrec*} has been added, and internal define specified in
terms of it.

\item {\cf Case} now supports a {\tt =>} syntax analogous to {\cf cond}.

\item {\cf Case-lambda} has been added to the base module as a way to
dispatch on the number of arguments passed to a procedure.

\item {\cf When} and {\cf unless} have been added as convenience
conditionals.

\item Positive and negative infinity and a NaN object have been added
to the numeric tower as inexact values with the written
representations {\tt +inf.0}, {\tt -inf.0} and {\tt +nan.0},
respectively.

\item {\cf Map} and {\cf for-each} are now required to terminate on
the shortest list when inputs have different length.

\item {\cf Member} and {\cf assoc} now take an optional third argument
for the equality predicate to use.

\item {\cf Exact-integer?}\  and {\cf exact-integer-sqrt} have been added.

\item {\cf Make-list}, {\cf list-copy}, {\cf list-set!}, {\cf
string-map}, {\cf string-for-each}, {\cf string->vector}, {\cf
vector-copy}, {\cf vector-map}, {\cf vector-for-each}, and {\cf
vector->string} have been added to round out the sequence operations.

\item The set of characters used is required to be consistent with the
Unicode Standard, 
but are not required to support the whole character repertoire.

\item {\cf string-ni=?} and related procedures have been added to
compare strings as though they had gone through an
implementation-defined normalization, without exposing the
normalization.

\item The case-folding behavior of the reader can now be explicitly
controlled, with no folding as the default.

\item The reader now recognizes the comment syntaxes {\tt \#;} to
skip the next datum, and {\tt \#| ... |\#}
for nestable block comments.

\item Data prefixed with reader labels {\tt \#<n>=} can be referenced
with {\tt \#<n>\#} allowing for reading and writing of data with
shared structure.

\item Strings and symbols now allow mnemonic and numeric escape
sequences, and the list of named characters has been extended.

\item {\cf File-exists?}\ and {\cf delete-file} are available in the
{\tt (scheme file)} module.

\item An interface to the system environment and command line is
available in the {\tt (scheme process-context)} module.

\item Procedures for accessing the current time are available in the
{\tt (scheme time)} module.

\item A complete set of integer division operators is available in the
{\tt (scheme division)} module.

\item {\cf Transcript-on} and {\cf transcript-off} have been removed.

\end{itemize}

