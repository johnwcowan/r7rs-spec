\chapter{Standard Feature Identifiers}
\label{stdfeatures}

An implementation may provide any or all of the feature identifiers
listed in table~\ref{standard_features}, as well as any others that it
chooses, but must not provide a feature identifier if it does not
provide the corresponding feature.

\begin{table*}
\begin{tabular}{|l|l|}
\hline
\textbf{Feature identifier} & \textbf{Feature description} \\ \hline
r7rs & All R7RS Scheme implementations have this feature. \\ \hline
exact-closed & All rational operations except {\cf /} produce exact values given exact inputs. \\ \hline
ratios & {\cf /} with exact arguments produces an exact result. \\ \hline
ieee-float & Inexact numbers are IEEE 754 floating point values. \\ \hline
full-unicode & All Unicode characters are supported. \\ \hline
windows & This Scheme implementation is running on Windows. \\ \hline
posix & This Scheme implementation is running on a Posix system. \\ \hline
unix, darwin, linux, bsd, freebsd, solaris, ... & Operating system flags (more than one may apply). \\ \hline
i386, x86-64, ppc, sparc, jvm, clr, llvm, ... & CPU architecture flags. \\ \hline
ilp32, lp64, ilp64, ... & C memory model flags \\ \hline
big-endian, little-endian & Byte order flags. \\ \hline
\hyper{name} & The name of this implementation. \\ \hline
\hyper{name-version} & The name and version of this implementation. \\ \hline
\end{tabular}
\caption{Standard Feature Identifiers}
\label{standard_features}
\end{table*}
