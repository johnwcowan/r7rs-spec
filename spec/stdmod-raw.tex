\chapter{Standard Libraries}
\label{stdlibraries}

%% Note, this is used to generate stdmod.tex.  The bindings could be
%% extracted automatically from the document, but this lets us choose
%% the ordering and optionally format manually where needed.

This section lists the exports provided by the standard libraries.  The
libraries are factored so as to separate features which might not be
supported by all implementations, or which might be expensive to load.

The {\cf scheme} library prefix is used for all standard libraries, and
is reserved for use by future standards.

\textbf{Base Library}

The \texttt{(scheme base)} library exports many of the procedures and
syntax bindings that are traditionally associated with Scheme.

\begin{scheme}
._               ...
.*                +                -
./                <=               <
.=>               =                >=
.>                abs              and
.append           apply            assoc
.assq             assv             begin
.boolean?         boolean=?
.bytevector-copy  bytevector-append  bytevector-copy!
.bytevector-length bytevector-u8-ref bytevector-u8-set!
.bytevector?            caar             cadr
.call-with-current-continuation     call-with-values
.call-with-port
.call/cc          case
.car              cdr
.caar    cadr     cdar    cddr
.caaar caadr cadar caddr cdaar cdadr cddar cdddr
.ceiling          char->integer    char<=?
.char<?
.char=?           char>=?          char>?
.char?            complex?         cond
.cond-expand
.cons             define-syntax    define
.define-values
.define-record-type                 denominator
.do               dynamic-wind     else
.eq?
.equal?           eqv?             error
.error-object?    error-object-message  error-object-irritants
.even?            exact            exact-integer-sqrt
.exact-integer?   exact?           expt
.floor            for-each         gcd
.floor/     floor-quotient     floor-remainder
.truncate/  truncate-quotient  truncate-remainder
.features
.guard            if               import
.inexact          inexact?         infinite?
.integer->char    integer?         lambda
.lcm              length           let*
.let-syntax       letrec*          letrec-syntax
.let-values       let*-values
.letrec           let              list-copy
.list->string     list->vector     list-ref
.list-set!        list-tail        list?
.list             make-bytevector  make-list
.make-parameter   make-string      make-vector
.map              max              member
.memq             memv             min
.modulo           negative?        not
.null?            number->string   number?
.numerator        odd?             or
.pair?            parameterize
.positive?
.procedure?       quasiquote       quote
.quotient         raise-continuable
.raise            rational?        rationalize
.real?            remainder        reverse
.round            set!             set-car!
.set-cdr!         square
.string->list     string->number
.string->symbol   string->vector   string-append
.string-copy      string-copy!
.string-fill!     string-for-each
.string-length    string-map       string-ref
.string-set!      string<=?        string<?
.string=?         string>=?        string>?
.string?          string           substring
.symbol->string   symbol=?
.symbol?          syntax-error
.syntax-rules     truncate         values
.unquote          unquote-splicing
.vector-copy      vector-copy!
.vector->list     vector->string   vector-fill!
.vector-for-each  vector-length    vector-map
.vector-ref       vector-set!      vector?
.vector           zero?            when
.with-exception-handler            unless
.binary-port?             char-ready?
.textual-port?            close-port
.close-input-port
.close-output-port        current-error-port
.current-input-port       current-output-port
.eof-object
.eof-object?              flush-output-port
.get-output-string        get-output-bytevector
.input-port?              newline
.open-input-string        open-output-string
.open-input-bytevector    open-output-bytevector
.output-port?             peek-char
.peek-u8                  port?
.port-open?               read-char
.read-line
.read-u8                  u8-ready?
.write-char               write-u8
.write-bytevector
.read-bytevector          read-bytevector!
.string->utf8             utf8->string
\end{scheme}

\textbf{Inexact Library}

The \texttt{(scheme inexact)} library exports procedures which are
typically only useful with inexact values.

\begin{scheme}
.exp     log      sqrt
.sin     cos      tan
.asin    acos     atan
.finite? nan?
\end{scheme}

\textbf{Complex Library}

The \texttt{(scheme complex)} library exports procedures which are
typically only useful with non-real numbers.

\begin{scheme}
.angle   magnitude   imag-part   real-part
.make-polar           make-rectangular
\end{scheme}

\textbf{Lazy Library}

The \texttt{(scheme lazy)} library exports procedures and syntax keywords for lazy evaluation.

\begin{scheme}
.delay   delay-force   force   make-promise
\end{scheme}

\textbf{Case-Lambda Library}

The \texttt{(scheme case-lambda)} library exports the {\cf case-lambda}
syntax.

\begin{scheme}
.case-lambda
\end{scheme}

\textbf{CxR Library}

The \texttt{(scheme cxr)} library exports twenty-four procedures which
are the compositions of from three to four {\cf car} and {\cf cdr}
operations.  For example {\cf caddar} could be defined by

\begin{scheme}
(define caddar
  (lambda (x) (car (cdr (cdr (car x)))))){\rm.}%
\end{scheme}

The procedures {\cf car} and {\cf cdr} themselves and the four
two-level compositions are included in the base library.  See
section~\ref{listsection}.

\begin{scheme}
.caaar
.caadr
\ldots
.cdddar
.cddddr
\end{scheme}

\textbf{Eval Library}

The \texttt{(scheme eval)} library exports procedures for evaluating Scheme
data as programs.

\begin{scheme}
.eval
.environment
\end{scheme}

\textbf{Repl Library}

The \texttt{(scheme repl)} library exports the {\cf
  interaction-environment} procedure.

\begin{scheme}
.interaction-environment
\end{scheme}

\textbf{Process-Context Library}

The \texttt{(scheme process-context)} library exports procedures for
accessing with the program's calling context.

\begin{scheme}
.get-environment-variable
.get-environment-variables
.command-line
.emergency-exit
.exit
\end{scheme}

\textbf{Load Library}

The \texttt{(scheme load)} library exports procedures for loading
Scheme expressions from files.

\begin{scheme}
.load
\end{scheme}

\textbf{File Library}

The \texttt{(scheme file)} library provides procedures for accessing
files.

\begin{scheme}
.call-with-input-file    call-with-output-file
.delete-file             file-exists?
.open-input-file         open-output-file
.open-binary-input-file  open-binary-output-file
.with-input-from-file    with-output-to-file
\end{scheme}

\textbf{Read Library}

The \texttt{(scheme read)} library provides procedures for reading
Scheme objects.

\begin{scheme}
.read
\end{scheme}

\textbf{Write Library}

The \texttt{(scheme write)} library provides procedures for writing
Scheme objects.

\begin{scheme}
.write  write-simple  display
\end{scheme}

\textbf{Char Library}

The \texttt{(scheme char)} library provides procedures for dealing
with Unicode character operations.

\begin{scheme}
.char-alphabetic?
.char-ci=?       char-ci<?       char-ci>?
.char-ci<=?      char-ci>=?      char-upcase
.char-downcase   char-foldcase   char-lower-case?
.char-numeric?   char-upper-case?
.char-whitespace?                 string-ci=?
.string-ci<?     string-ci>?     string-ci<=?
.string-ci>=?    string-upcase   string-downcase
.string-foldcase
.digit-value
\end{scheme}

\textbf{Time Library}

The \texttt{(scheme time)} library provides access to time-related values.

\begin{scheme}
.current-second
.current-jiffy
.jiffies-per-second
\end{scheme}

\textbf{R5RS Library}

The \texttt{(scheme r5rs)} library provides the identifiers present in
both this report and
the ``Revised$^5$ report''~\cite{R5RS}, with the following deviations:

\begin{itemize}

\item{The \rfivers\ procedures {\cf transcript-on} and {\cf transcript-off} are not present.}

\item{The {\cf exact} and {\cf inexact} procedures appear under their \rfivers\ names
{\cf inexact->exact} and {\cf exact->inexact} respectively.}

\item{If an implementation does not provide a particular library such as the
complex library, the corresponding identifiers will not appear in this
library either.}

\item{The \rfivers\ procedures {\cf scheme-report-environment}
and {\cf null-environment} appear only in this library and not in any other.}

\end{itemize}

\begin{scheme}
.- * / + < <= = > >= abs acos and angle append apply asin assoc assq
.assv atan begin boolean? call-with-current-continuation
.call-with-values car case cdr ceiling char->integer char-alphabetic?
.char-ci<? char-ci<=? char-ci=? char-ci>? char-ci>=? char-downcase
.char-lower-case? char-numeric? char-ready? char-upcase
.char-upper-case? char-whitespace? char? char<? char<=? char=? char>?
.char>=? close-input-port close-output-port complex? cond cons cos
.current-input-port current-output-port define define-syntax delay
.denominator display do dynamic-wind eof-object? eq? equal? eqv? eval
.even? exact->inexact exact? exp expt floor for-each force gcd if
.imag-part inexact->exact inexact? input-port? integer->char integer?
.interaction-environment lambda lcm length let let-syntax let* letrec
.letrec-syntax list list->string list->vector list-ref list-tail list?
.load log magnitude make-polar make-rectangular make-string make-vector
.map max member memq memv min modulo negative? newline not
.null-environment null? number->string number? numerator odd?
.open-input-file open-output-file or output-port? pair? peek-char
.positive? procedure? quasiquote quote quotient rational? rationalize
.read read-char real-part real? remainder reverse round
.scheme-report-environment set-car! set-cdr! set! sin sqrt string
.string->list string->number string->symbol string-append string-ci<?
.string-ci<=? string-ci=? string-ci>? string-ci>=? string-copy
.string-fill! string-length string-ref string-set! string? string<?
.string<=? string=? string>? string>=? substring symbol->string symbol?
.tan truncate values vector vector-append vector->list vector-fill! vector-length
.vector-ref vector-set! vector? with-input-from-file
.with-output-to-file write write-char zero?
\end{scheme}
