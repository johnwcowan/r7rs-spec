\chapter{Standard Libraries}
\label{stdlibraries}

%% Note, this is used to generate stdmod.tex.  The bindings could be
%% extracted automatically from the document, but this lets us choose
%% the ordering and optionally format manually where needed.

This section lists the exports provided by the standard libraries.  The
libraries are factored so as to separate features which may not be
supported by all implementations, or which may be expensive to load.

The {\cf scheme} library prefix is used for all standard libraries, and
is reserved for use by future standards.

\textbf{Base Library}

The \texttt{(scheme base)} library exports many of the procedures and
syntax bindings that are traditionally associated with Scheme.

\begin{scheme}
._               ...
.*                +                -
./                <=               <
.=>               =                >=
.>                abs              and
.append           apply            assoc
.assq             assv             begin
.boolean?         bytevector-copy  bytevector-copy!
.bytevector-copy-partial
.bytevector-length bytevector-u8-ref bytevector-u8-set!
.bytevector?            caar             cadr
.call-with-current-continuation     call-with-values
.call-with-port
.call/cc
.car              case
.cdddar           cddddr           cdr
.ceiling          char->integer    char<=?
.char<?
.char=?           char>=?          char>?
.char?            complex?         cond
.cond-expand
.cons             define-syntax    define
.define-record-type                 denominator
.do               dynamic-wind     else
.eq?
.equal?           eqv?             error
.error-object?    error-object-message  error-object-irritants
.even?            exact->inexact   exact-integer-sqrt
.exact-integer?   exact?           expt
.floor            for-each         gcd
.guard            if               import
.inexact->exact   inexact?
.integer->char    integer?         lambda
.lcm              length           let*
.let-syntax       letrec*          letrec-syntax
.let-values       let*-values
.letrec           let              list-copy
.list->string     list->vector     list-ref
.list-set!        list-tail        list?
.list             make-bytevector  make-list
.make-parameter   make-string      make-vector
.map              max              member
.memq             memv             min
.modulo           negative?        not
.null?            number->string   number?
.numerator        odd?             or
.pair?            parameterize
.bytevector-copy-partial!           positive?
.procedure?       quasiquote       quote
.quotient         raise-continuable
.raise            rational?        rationalize
.real?            remainder        reverse
.round            set!             set-car!
.set-cdr!         string->list     string->number
.string->symbol   string->vector   string-append
.string-copy      string-fill!     string-for-each
.string-length    string-map       string-ref
.string-set!      string<=?        string<?
.string=?         string>=?        string>?
.string?          string           substring
.symbol->string   symbol?          syntax-error
.syntax-rules     truncate         values
.unquote          unquote-splicing
.vector-copy
.vector->list     vector->string   vector-fill!
.vector-for-each  vector-length    vector-map
.vector-ref       vector-set!      vector?
.vector           zero?            when
.with-exception-handler            unless
.binary-port?             char-ready?
.textual-port?            close-port
.close-input-port
.close-output-port        current-error-port
.current-input-port       current-output-port
.eof-object?              flush-output-port
.get-output-string        get-output-bytevector
.input-port?              newline
.open-input-string        open-output-string
.open-input-bytevector    open-output-bytevector
.output-port?             peek-char
.peek-u8                  port?
.port-open?               read-char
.read-line
.read-u8                  u8-ready?
.write-char               write-u8
.write-bytevector         write-partial-bytevector
.read-bytevector          read-bytevector!
.string->utf8             utf8->string
\end{scheme}

\textbf{Inexact Library}

The \texttt{(scheme inexact)} library exports procedures which are
typically only useful with inexact values.

\begin{scheme}
.exp     log      sqrt
.sin     cos      tan
.asin    acos     atan
.finite? nan?
\end{scheme}

\textbf{Complex Library}

The \texttt{(scheme complex)} library exports procedures which are
typically only useful with complex values.

\begin{scheme}
.angle   magnitude   imag-part   real-part
.make-polar           make-rectangular
\end{scheme}

\textbf{Division Library}

The \texttt{(scheme division)} library exports procedures for integer
division.

\begin{scheme}
.floor/     floor-quotient     floor-remainder
.ceiling/   ceiling-quotient   ceiling-remainder
.centered/  centered-quotient  centered-remainder
.truncate/  truncate-quotient  truncate-remainder
.round/     round-quotient     round-remainder
.euclidean/ euclidean-quotient euclidean-remainder
\end{scheme}

\textbf{Lazy Library}

The \texttt{(scheme lazy)} library exports forms for lazy evaluation.

\begin{scheme}
.delay   eager  force   lazy
\end{scheme}

\textbf{Case-Lambda Library}

The \texttt{(scheme case-lambda)} library exports the {\cf case-lambda}
syntax.

\begin{scheme}
.case-lambda
\end{scheme}

\textbf{Eval Library}

The \texttt{(scheme eval)} library exports procedures for evaluating Scheme
data as programs.

\begin{scheme}
.eval
.environment
.null-environment
.scheme-report-environment
\end{scheme}

\textbf{Repl Library}

The \texttt{(scheme repl)} library exports the {\cf
  interaction-environment} procedure.

\begin{scheme}
.interaction-environment
\end{scheme}

\textbf{Process Context Library}

The \texttt{(scheme process-context)} library exports procedures for
accessing with the program's calling context.

\begin{scheme}
.get-environment-variable
.get-environment-variables
.command-line
.exit
\end{scheme}

\textbf{Load Library}

The \texttt{(scheme load)} library exports forms for loading and
including Scheme expressions from files.

\begin{scheme}
.load
\end{scheme}

\textbf{File Library}

The \texttt{(scheme file)} library provides procedures for accessing
files.

\begin{scheme}
.call-with-input-file    call-with-output-file
.delete-file             file-exists?
.open-input-file         open-output-file
.open-binary-input-file  open-binary-output-file
.with-input-from-file    with-output-to-file
\end{scheme}

\textbf{Read Library}

The \texttt{(scheme read)} library provides procedures for reading
Scheme objects.

\begin{scheme}
.read
\end{scheme}

\textbf{Write Library}

The \texttt{(scheme write)} library provides procedures for writing
Scheme objects.

\begin{scheme}
.write  write-simple  display
\end{scheme}

\textbf{Char Library}

The \texttt{(scheme char)} library provides procedures for dealing
with Unicode character operations.

\begin{scheme}
.char-alphabetic?
.char-ci=?       char-ci<?       char-ci>?
.char-ci<=?      char-ci>=?      char-upcase
.char-downcase   char-foldcase   char-lower-case?
.char-numeric?   char-upper-case?
.char-whitespace?                 string-ci=?
.string-ci<?     string-ci>?     string-ci<=?
.string-ci>=?    string-upcase   string-downcase
.string-foldcase
.numeric-digit
\end{scheme}

\textbf{Char Normalization Library}

The \texttt{(scheme char normalization)} library provides procedures
for dealing with Unicode normalization operations.

\begin{scheme}
.string-ni=?     string-ni<?     string-ni>?
.string-ni<=?    string-ni>=?
\end{scheme}

\textbf{Time}

The \texttt{(scheme time)} library provides access to the system time.

\begin{scheme}
.current-second
.current-jiffy
.jiffies-per-second
\end{scheme}
