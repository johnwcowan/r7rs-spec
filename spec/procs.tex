% Initial environment

%\vfill\eject
\chapter{Standard procedures}
\label{initialenv}
\label{builtinchapter}

\mainindex{initial environment}
\mainindex{top level environment}
\mainindex{procedure}

This chapter describes Scheme's built-in procedures.  The initial (or
``top level'') Scheme environment is empty except for {\cf import},
so further bindings can only be introduced with {\cf import}.

The procedures {\cf force} and {\cf eager} are intimately associated
with the expression types {\cf delay} and {\cf lazy}, and are described
with them in section~\ref{force}.  In the same way, the procedure
{\cf make-parameter} is intimately assocaited with the expression type
{\cf parameterize}, and is described with it in section~\ref{make-parameter}.

\todo{Gleckler: We need to make it clearer what ``top level'' really
  means here.  For example, how we shoudl explain how it differs from
  the REPL.}

\todo{consider removing unspecified effect}
A program may use a top-level definition to bind any variable.  It may
subsequently alter any such binding by an assignment (see
section~\ref{assignment}).  These operations do not modify the behavior of
Scheme's built-in procedures, or any procedure defined in a library
(see section~\ref{libraries}).  Altering any top-level binding that has
not been introduced by a definition has an unspecified effect on the
behavior of the built-in procedures.

\section{Equivalence predicates}
\label{equivalencesection}

A \defining{predicate} is a procedure that always returns a boolean
value (\schtrue{} or \schfalse).  An \defining{equivalence predicate} is
the computational analogue of a mathematical equivalence relation (it is
symmetric, reflexive, and transitive).  Of the equivalence predicates
described in this section, {\cf eq?}\ is the finest or most
discriminating, and {\cf equal?}\ is the coarsest.  {\cf Eqv?}\ is
slightly less discriminating than {\cf eq?}.  


\begin{entry}{%
\proto{eqv?}{ \vari{obj} \varii{obj}}{procedure}}

The {\cf eqv?} procedure defines a useful equivalence relation on objects.
Briefly, it returns \schtrue{} if \vari{obj} and \varii{obj} are
normally regarded as the same object.  This relation is left slightly
open to interpretation, but the following partial specification of
{\cf eqv?} holds for all implementations of Scheme.

The {\cf eqv?} procedure returns \schtrue{} if:

\begin{itemize}
\item \vari{obj} and \varii{obj} are both \schtrue{} or both \schfalse.

\item \vari{obj} and \varii{obj} are both symbols and

\begin{scheme}
(string=? (symbol->string obj1)
          (symbol->string obj2))
    \ev  \schtrue%
\end{scheme}

\begin{note} 
This assumes that neither \vari{obj} nor \varii{obj} is an ``uninterned
symbol'' as alluded to in section~\ref{symbolsection}.  This report does
not specify the behavior of {\cf eqv?} on implementation-dependent
extensions.
\end{note}

\item \vari{obj} and \varii{obj} are both numbers, and {\cf nan?}
returns true on both of them.

\item \vari{obj} and \varii{obj} are both numbers, are numerically
equal (see {\cf =}, section~\ref{numbersection}), and are either both
exact\index{exact} or both inexact\index{inexact}.

\item \vari{obj} and \varii{obj} are both characters and are the same
character according to the {\cf char=?} procedure
(section~\ref{charactersection}).

\item \vari{obj} and \varii{obj} are both the empty list.

\item \vari{obj} and \varii{obj} are pairs, vectors, bytevectors, records,
or strings that denote the same location in the store
(section~\ref{storagemodel}).
\end{itemize}

The {\cf eqv?} procedure returns \schfalse{} if:

\begin{itemize}
\item \vari{obj} and \varii{obj} are of different types
(section~\ref{disjointness}).

\item one of \vari{obj} and \varii{obj} is \schtrue{} but the other is
\schfalse{}.

\item \vari{obj} and \varii{obj} are symbols but

\begin{scheme}
(string=? (symbol->string \vari{obj})
          (symbol->string \varii{obj}))
    \ev  \schfalse%
\end{scheme}

\item \vari{obj} and \varii{obj} are numbers, and {\cf nan?} returns true
on one of them but not the other.

\item one of \vari{obj} and \varii{obj} is an exact number but the other
is an inexact number.

\item \vari{obj} and \varii{obj} are numbers for which the {\cf =}
procedure returns \schfalse{}.

\item \vari{obj} and \varii{obj} are characters for which the {\cf char=?}
procedure returns \schfalse{}.

\item one of \vari{obj} and \varii{obj} is the empty list but the other
is not.

\item \vari{obj} and \varii{obj} are pairs, vectors, bytevectors, records,
or strings that denote distinct locations.

\item \vari{obj} and \varii{obj} are procedures that would behave differently
(return different values or have different side effects) for some arguments.

\end{itemize}

\begin{scheme}
(eqv? 'a 'a)                     \ev  \schtrue
(eqv? 'a 'b)                     \ev  \schfalse
(eqv? 2 2)                       \ev  \schtrue
(eqv? '() '())                   \ev  \schtrue
(eqv? 100000000 100000000)       \ev  \schtrue
(eqv? (cons 1 2) (cons 1 2))     \ev  \schfalse
(eqv? (lambda () 1)
      (lambda () 2))             \ev  \schfalse
(eqv? \#f 'nil)                  \ev  \schfalse
\end{scheme}

The following examples illustrate cases in which the above rules do
not fully specify the behavior of {\cf eqv?}.  All that can be said
about such cases is that the value returned by {\cf eqv?} must be a
boolean.

\begin{scheme}
(eqv? "" "")             \ev  \unspecified
(eqv? '\#() '\#())         \ev  \unspecified
(eqv? (lambda (x) x)
      (lambda (x) x))    \ev  \unspecified
(let ((p (lambda (x) x)))
  (eqv? p p))                    \ev  \unspecified
(eqv? (lambda (x) x)
      (lambda (y) y))    \ev  \unspecified%
\end{scheme}

The next set of examples shows the use of {\cf eqv?}\ with procedures
that have local state.  {\cf Gen-counter} must return a distinct
procedure every time, since each procedure has its own internal counter.
{\cf Gen-loser}, however, returns equivalent procedures each time, since
the local state does not affect the value or side effects of the
procedures.  However, {\cf eqv?} may or may not detect this equivalence.

\begin{scheme}
(define gen-counter
  (lambda ()
    (let ((n 0))
      (lambda () (set! n (+ n 1)) n))))
(let ((g (gen-counter)))
  (eqv? g g))           \ev  \schtrue
(eqv? (gen-counter) (gen-counter))
                        \ev  \schfalse
(define gen-loser
  (lambda ()
    (let ((n 0))
      (lambda () (set! n (+ n 1)) 27))))
(let ((g (gen-loser)))
  (eqv? g g))           \ev  \schtrue
(eqv? (gen-loser) (gen-loser))
                        \ev  \unspecified

(letrec ((f (lambda () (if (eqv? f g) 'both 'f)))
         (g (lambda () (if (eqv? f g) 'both 'g))))
  (eqv? f g))
                        \ev  \unspecified

(letrec ((f (lambda () (if (eqv? f g) 'f 'both)))
         (g (lambda () (if (eqv? f g) 'g 'both))))
  (eqv? f g))
                        \ev  \schfalse%
\end{scheme}

Since it is an error to modify constant objects (those returned by
literal expressions), implementations are permitted, though not
required, to share structure between constants where appropriate.  Thus
the value of {\cf eqv?} on constants is sometimes
implementation-dependent.

\begin{scheme}
(eqv? '(a) '(a))                 \ev  \unspecified
(eqv? "a" "a")                   \ev  \unspecified
(eqv? '(b) (cdr '(a b)))	 \ev  \unspecified
(let ((x '(a)))
  (eqv? x x))                    \ev  \schtrue%
\end{scheme}

\begin{rationale} 
The above definition of {\cf eqv?} allows implementations latitude in
their treatment of procedures and literals:  implementations are free
either to detect or to fail to detect that two procedures or two literals
are equivalent to each other, and can decide whether or not to
merge representations of equivalent objects by using the same pointer or
bit pattern to represent both.
\end{rationale}

\end{entry}


\begin{entry}{%
\proto{eq?}{ \vari{obj} \varii{obj}}{procedure}}

{\cf Eq?}\ is similar to {\cf eqv?}\ except that in some cases it is
capable of discerning distinctions finer than those detectable by
{\cf eqv?}.

\vest {\cf Eq?}\ and {\cf eqv?}\ are guaranteed to have the same
behavior on symbols, booleans, the empty list, pairs, procedures,
and non-empty
strings, vectors, bytevectors, and records.  {\cf Eq?}'s behavior
on numbers and characters is
implementation-dependent, but it will always return either true or
false, and will return true only when {\cf eqv?}\ would also return
true.  {\cf Eq?} may also behave differently from {\cf eqv?} on empty
vectors and empty strings.

\begin{scheme}
(eq? 'a 'a)                     \ev  \schtrue
(eq? '(a) '(a))                 \ev  \unspecified
(eq? (list 'a) (list 'a))       \ev  \schfalse
(eq? "a" "a")                   \ev  \unspecified
(eq? "" "")                     \ev  \unspecified
(eq? '() '())                   \ev  \schtrue
(eq? 2 2)                       \ev  \unspecified
(eq? \#\backwhack{}A \#\backwhack{}A) \ev  \unspecified
(eq? car car)                   \ev  \unspecified
(let ((n (+ 2 3)))
  (eq? n n))      \ev  \unspecified
(let ((x '(a)))
  (eq? x x))      \ev  \schtrue
(let ((x '\#()))
  (eq? x x))      \ev  \schtrue
(let ((p (lambda (x) x)))
  (eq? p p))      \ev  \schtrue%
\end{scheme}


\begin{rationale} It will usually be possible to implement {\cf eq?}\ much
more efficiently than {\cf eqv?}, for example, as a simple pointer
comparison instead of as some more complicated operation.  One reason is
that it is not always possible to compute {\cf eqv?}\ of two numbers in
constant time, whereas {\cf eq?}\ implemented as pointer comparison will
always finish in constant time.  {\cf Eq?}\ may be used like {\cf eqv?}\
in applications using procedures to implement objects with state since
it obeys the same constraints as {\cf eqv?}.
\end{rationale}

\end{entry}


\begin{entry}{%
\proto{equal?}{ \vari{obj} \varii{obj}}{procedure}}

{\cf Equal?} recursively compares the contents of pairs, vectors, 
strings, bytevectors, and records,
applying {\cf eqv?} on other objects such as numbers and symbols.
A rule of thumb is that objects are generally {\cf equal?} if they print
the same.  {\cf Equal?}\ must always terminate, even if its arguments are
circular data structures.
If two objects are {cf eqv?}, they must be {cf equal?} as well.

\begin{scheme}
(equal? 'a 'a)                  \ev  \schtrue
(equal? '(a) '(a))              \ev  \schtrue
(equal? '(a (b) c)
        '(a (b) c))             \ev  \schtrue
(equal? "abc" "abc")            \ev  \schtrue
(equal? 2 2)                    \ev  \schtrue
(equal? (make-vector 5 'a)
        (make-vector 5 'a))     \ev  \schtrue
(equal? (lambda (x) x)
        (lambda (y) y))  \ev  \unspecified%
\end{scheme}


\end{entry}


\section{Numbers}
\label{numbersection}
\index{number}

\newcommand{\type}[1]{{\it#1}}
\newcommand{\tupe}[1]{{#1}}

It is important to distinguish between mathematical numbers, the
Scheme numbers that attempt to model them, the machine representations
used to implement the Scheme numbers, and notations used to write numbers.
This report uses the types \type{number}, \type{complex}, \type{real},
\type{rational}, and \type{integer} to refer to both mathematical numbers
and Scheme numbers.  

\subsection{Numerical types}
\label{numericaltypes}
\index{numerical types}

\vest Mathematically, numbers are arranged into a tower of subtypes
in which each level is a subset of the level above it:
\begin{tabbing}
\ \ \ \ \ \ \ \ \ \=\tupe{number} \\
\> \tupe{complex} \\
\> \tupe{real} \\
\> \tupe{rational} \\
\> \tupe{integer} 
\end{tabbing}

For example, 3 is an integer.  Therefore 3 is also a rational,
a real, and a complex.  The same is true of the Scheme numbers
that model 3.  For Scheme numbers, these types are defined by the
predicates \ide{number?}, \ide{complex?}, \ide{real?}, \ide{rational?},
and \ide{integer?}.

There is no simple relationship between a number's type and its
representation inside a computer.  Although most implementations of
Scheme will offer at least two different representations of 3, these
different representations denote the same integer.

Scheme's numerical operations treat numbers as abstract data, as
independent of their representation as possible.  Although an implementation
of Scheme may use multiple internal representations of
numbers, this should not be apparent to a casual programmer writing
simple programs.

It is necessary, however, to distinguish between numbers that are
represented exactly and those that might not be.  For example, indexes
into data structures must be known exactly, as must some polynomial
coefficients in a symbolic algebra system.  On the other hand, the
results of measurements are inherently inexact, and irrational numbers
may be approximated by rational and therefore inexact approximations.
In order to catch uses of inexact numbers where exact numbers are
required, Scheme explicitly distinguishes exact from inexact numbers.
This distinction is orthogonal to the dimension of type.

\subsection{Exactness}

\mainindex{exactness} \label{exactly}
Scheme numbers are either \type{exact} or \type{inexact}.  A number is
\tupe{exact} if it was written as an exact constant or was derived from
\tupe{exact} numbers using only \tupe{exact} operations.  A number is
\tupe{inexact} if it was written as an inexact constant,
if it was
derived using \tupe{inexact} ingredients, or if it was derived using
\tupe{inexact} operations. Thus \tupe{inexact}ness is a contagious
property of a number.

\vest If two implementations produce \tupe{exact} results for a
computation that did not involve \tupe{inexact} intermediate results,
the two ultimate results will be mathematically equivalent.  This is
generally not true of computations involving \tupe{inexact} numbers
since approximate methods such as floating-point arithmetic may be used,
but it is the duty of each implementation to make the result as close as
practical to the mathematically ideal result.

\vest Rational operations such as {\cf +} should always produce
\tupe{exact} results when given \tupe{exact} arguments.
If the operation is unable to produce an \tupe{exact} result,
then it may either report the violation of an implementation restriction
or it may silently coerce its
result to an \tupe{inexact} value.
See section~\ref{restrictions}.

\vest Except for \ide{inexact->exact}, the operations described in
this section must generally return inexact results when given any inexact
arguments.  An operation may, however, return an \tupe{exact} result if it can
prove that the value of the result is unaffected by the inexactness of its
arguments.  For example, multiplication of any number by an \tupe{exact} zero
may produce an \tupe{exact} zero result, even if the other argument is
\tupe{inexact}.

\subsection{Implementation restrictions}

\index{implementation restriction}\label{restrictions}

\vest Implementations of Scheme are not required to implement the whole
tower of subtypes given in section~\ref{numericaltypes},
but they must implement a coherent subset consistent with both the
purposes of the implementation and the spirit of the Scheme language.
For example, implementations in which all numbers are \tupe{real},
or in which non-\tupe{real} numbers are always \tupe{inexact},
or in which \tupe{exact} numbers are always \tupe{integer},
are still quite useful.

\vest Implementations may also support only a limited range of numbers of
any type, subject to the requirements of this section.  The supported
range for \tupe{exact} numbers of any type may be different from the
supported range for \tupe{inexact} numbers of that type.  For example,
an implementation that uses IEEE double-precision floating-point numbers to represent all its
\tupe{inexact} \tupe{real} numbers may also
support a practically unbounded range of \tupe{exact} \tupe{integer}s
and \tupe{rational}s
while limiting the range of \tupe{inexact} \tupe{real}s (and therefore
the range of \tupe{inexact} \tupe{integer}s and \tupe{rational}s)
to the dynamic range of the IEEE double format.
Furthermore,
the gaps between the representable \tupe{inexact} \tupe{integer}s and
\tupe{rational}s are
likely to be very large in such an implementation as the limits of this
range are approached.

\vest An implementation of Scheme must support exact integers
throughout the range of numbers permitted as indexes of
lists, vectors, and strings or that result from computing the length of a
list, vector, or string.  The \ide{length}, \ide{vector-length},
\ide{bytevector-length}, and \ide{string-length} procedures must return an exact
integer, and it is an error to use anything but an exact integer as an
index.  Furthermore, any integer constant within the index range, if
expressed by an exact integer syntax, will indeed be read as an exact
integer, regardless of any implementation restrictions that apply
outside this range.  Finally, the procedures listed below will always
return exact integer results provided all their arguments are exact integers
and the mathematically expected results are representable as exact integers
within the implementation:

\begin{scheme}
+                  -                   *
quotient           remainder           modulo
max                min                 abs
numerator          denominator         gcd
lcm                floor               ceiling
truncate           round               rationalize
expt               exact-integer-sqrt
floor/             ceiling/            centered/
truncate/          round/              euclidean/
floor-quotient     floor-remainder
ceiling-quotient   ceiling-remainder
centered-quotient  centered-remainder
truncate-quotient  truncate-remainder
round-quotient     round-remainder
euclidean-quotient euclidean-remainder
\end{scheme}

\vest Implementations are encouraged, but not required, to support
\tupe{exact} \tupe{integer}s and \tupe{exact} \tupe{rational}s of
practically unlimited size and precision, and to implement the
above procedures and the {\cf /} procedure in
such a way that they always return \tupe{exact} results when given \tupe{exact}
arguments.  If one of these procedures is unable to deliver an \tupe{exact}
result when given \tupe{exact} arguments, then it may either report a
violation of an
implementation restriction or it may silently coerce its result to an
\tupe{inexact} number; such a coercion can cause an error later.
Nevertheless, implementations that do not provide \tupe{exact} rational
numbers should return \tupe{inexact} rational numbers rather than
reporting an implementation restriction.

\vest An implementation may use floating-point and other approximate 
representation strategies for \tupe{inexact} numbers.
This report recommends, but does not require, that the IEEE 754
standard be followed by implementations that use
floating-point representations, and that implementations using
other representations should match or exceed the precision achievable
using these floating-point standards~\cite{IEEE}.
In particular, the description of transcendental functions in IEEE 754-2008
should be followed by such implementations, particularly with respect
to infinities and NaNs.

Although Scheme allows a variety of written
notations for
numbers, any particular implementation may support only some of them.
For example, an implementation in which all numbers are \tupe{real}
need not support the rectangular and polar notations for complex
numbers.  If an implementation encounters an \tupe{exact} numerical constant that
it cannot represent as an \tupe{exact} number, then it may either report a
violation of an implementation restriction or it may silently represent the
constant by an \tupe{inexact} number.

\subsection{Implementation extensions}
\index{implementation extension}

\vest Implementations may provide more than one representation of
floating-point numbers with differing precisions.  In an implementation
which does so, an inexact result must be represented with at least
as much precision as is used to express any of the inexact arguments
to that operation.  Although it is desirable for potentially inexact
operations such as {\cf sqrt} to produce \tupe{exact} answers when
applied to \tupe{exact} arguments, if an \tupe{exact} number is operated
upon so as to produce an \tupe{inexact} result, then the most precise
representation available must be used.  For example, the value of {\cf
(sqrt 4)} should be {\cf 2}, but in an implementation that provides both
single and double precision floating point numbers it may be {\cf 2.0d0}
but must not be {\cf 2.0f0} (the standard representation of single precision).

In addition, implementations may
distinguish special numbers called \tupe{positive infinity},
\tupe{negative infinity}, \tupe{NaN}, and \tupe{negative zero}.

Positive infinity is regarded as an inexact real (but not rational)
number that represents an indeterminate value greater than the
numbers represented by all rational numbers. Negative infinity
is regarded as an inexact real (but not rational) number that
represents an indeterminate value less than the numbers represented
by all rational numbers.

A NaN is regarded as an inexact real (but not rational) number
so indeterminate that it might represent any real value, including
positive or negative infinity, and might even be greater than positive
infinity or less than negative infinity.
It might even represent no number at all, as in the case of 
{\cf (asin 2.0)}.

Note that the real and the imaginary parts of a complex number
can be infinities or NaNs.

Negative zero is an inexact real value written {\cf -0.0} which is distinct
(in the sense of {\cf eqv?}) from {\cf 0.0}.  A Scheme implementation
is not required to distinguish negative zero.  If it does, however, the
behavior of the transcendental functions is sensitive to the distinction
in accordance with IEEE 754.

Furthermore, the negation of negative zero is ordinary zero and vice
versa.  This implies that the sum of two negative zeros is negative,
and the result of subtracting (positive) zero from a negative zero is
likewise negative.  However, numerical comparisons treat negative zero
as equal to zero.

\subsection{Syntax of numerical constants}
\label{numbernotations}

The syntax of the written representations for numbers is described formally in
section~\ref{numbersyntax}.  Note that case is not significant in numerical
constants.

A number can be written in binary, octal, decimal, or
hexa\-decimal by the use of a radix prefix.  The radix prefixes are {\cf
\#b}\sharpindex{b} (binary), {\cf \#o}\sharpindex{o} (octal), {\cf
\#d}\sharpindex{d} (decimal), and {\cf \#x}\sharpindex{x} (hexa\-decimal).  With
no radix prefix, a number is assumed to be expressed in decimal.

A
numerical constant can be specified to be either \tupe{exact} or
\tupe{inexact} by a prefix.  The prefixes are {\cf \#e}\sharpindex{e}
for \tupe{exact}, and {\cf \#i}\sharpindex{i} for \tupe{inexact}.  An exactness
prefix may appear before or after any radix prefix that is used.  If
the written representation of a number has no exactness prefix, the
constant is
\tupe{inexact} if it contains a decimal point or an
exponent.
Otherwise, it is \tupe{exact}.

In systems with \tupe{inexact} numbers
of varying precisions it can be useful to specify
the precision of a constant.  For this purpose,
implementations may accept numerical constants
written with an exponent marker that indicates the
desired precision of the \tupe{inexact}
representation.  The letters {\cf s}, {\cf f},
{\cf d}, and {\cf l}, meaning \var{short}, \var{single},
\var{double}, and \var{long} precision respectively,
are acceptable in place of {\cf e}.
The default precision has at least as much precision
as \var{double}, but
implementations may allow this default to be set by the user.

\begin{scheme}
3.14159265358979F0
       {\rm Round to single ---} 3.141593
0.6L0
       {\rm Extend to long ---} .600000000000000%
\end{scheme}

The numbers positive infinity, negative infinity and NaN are written
{\cf +inf.0}, {\cf -inf.0} and {\cf +nan.0} respectively.
Implementations are not required to support them, but if they do,
they must be in conformance with IEEE 754.  However, implementations
are not required to support signaling NaNs, or provide a way to distinguish
between different NaNs.


\subsection{Numerical operations}

The reader is referred to section~\ref{typeconventions} for a summary
of the naming conventions used to specify restrictions on the types of
arguments to numerical routines.
The examples used in this section assume that any numerical constant written
using an \tupe{exact} notation is indeed represented as an \tupe{exact}
number.  Some examples also assume that certain numerical constants written
using an \tupe{inexact} notation can be represented without loss of
accuracy; the \tupe{inexact} constants were chosen so that this is
likely to be true in implementations that use IEEE doubles to represent
inexact numbers.

\todo{Scheme provides the usual set of operations for manipulating
numbers, etc.}

\begin{entry}{%
\proto{number?}{ obj}{procedure}
\proto{complex?}{ obj}{procedure}
\proto{real?}{ obj}{procedure}
\proto{rational?}{ obj}{procedure}
\proto{integer?}{ obj}{procedure}}

These numerical type predicates can be applied to any kind of
argument, including non-numbers.  They return \schtrue{} if the object is
of the named type, and otherwise they return \schfalse{}.
In general, if a type predicate is true of a number then all higher
type predicates are also true of that number.  Consequently, if a type
predicate is false of a number, then all lower type predicates are
also false of that number.

If \vr{z} is a complex number, then {\cf (real? \vr{z})} is true if
and only if {\cf (zero? (imag-part \vr{z}))} and {\cf (exact? (imag-part
\vr{z}))} are both true.  If \vr{x} is an inexact real number, then {\cf
(integer? \vr{x})} is true if and only if {\cf (= \vr{x} (round \vr{x}))}.

The numbers {\cf +inf.0}, {\cf -inf.0}, and {\cf +nan.0} are real but
not rational.

\begin{scheme}
(complex? 3+4i)         \ev  \schtrue
(complex? 3)            \ev  \schtrue
(real? 3)               \ev  \schtrue
(real? -2.5+0i)         \ev  \schtrue
(real? -2.5+0.0i)       \ev  \schfalse
(real? \#e1e10)          \ev  \schtrue
(real? +inf.0)           \ev  \schtrue
(rational? -inf.0)       \ev  \schfalse
(rational? 6/10)        \ev  \schtrue
(rational? 6/3)         \ev  \schtrue
(integer? 3+0i)         \ev  \schtrue
(integer? 3.0)          \ev  \schtrue
(integer? 8/4)          \ev  \schtrue%
\end{scheme}

\begin{note}
The behavior of these type predicates on \tupe{inexact} numbers
is unreliable, since any inaccuracy might affect the result.
\end{note}

\begin{note}
In many implementations the \ide{complex?} procedure will be the same as
\ide{number?}, but unusual implementations may be able to represent
some irrational numbers exactly or may extend the number system to
support some kind of non-complex numbers.
\end{note}

\end{entry}

\begin{entry}{%
\proto{exact?}{ \vr{z}}{procedure}
\proto{inexact?}{ \vr{z}}{procedure}}

These numerical predicates provide tests for the exactness of a
quantity.  For any Scheme number, precisely one of these predicates
is true.

\begin{scheme}
(exact? 3.0)           \ev  \schfalse
(exact? \#e3.0)         \ev  \schtrue
(inexact? 3.)          \ev  \schtrue
\end{scheme}

\end{entry}


\begin{entry}{%
\proto{exact-integer?}{ \vr{z}}{procedure}}

The conjunction of {\cf exact?} and {\cf integer?}, it returns \schtrue{}
if \vr{z} is both \tupe{exact} and an \tupe{integer}.
Otherwise, it returns \schfalse{}.
\begin{scheme}
(exact-integer? 32) \ev \schtrue{}
(exact-integer? 32.0) \ev \schfalse{}
(exact-integer? 32/5) \ev \schfalse{}
\end{scheme}
\end{entry}


\begin{entry}{%
\proto{finite?}{ \vr{z}}{inexact library procedure}}

{\cf Finite?} returns \schtrue{} on all real numbers except
{\cf +inf.0}, {\cf -inf.0}, and {\cf +nan.0}, and on complex
numbers if their real and imaginary parts are both finite.
Otherwise it returns \schfalse{}.

\begin{scheme}
(finite? 3)         \ev  \schtrue
(finite? +inf.0)       \ev  \schfalse
(finite? 3.0+inf.0i)   \ev  \schfalse
\end{scheme}
\end{entry}

\begin{entry}{%
\proto{nan?}{ \vr{z}}{inexact library procedure}}

{\cf Nan} returns \schtrue{} on {\cf +nan.0}, and on any complex number
if its real part or its imaginary part or both are {\cf +nan.0}.
Otherwise it returns \schfalse{}.

\begin{scheme}
(nan? +nan.0)          \ev  \schtrue
(nan? +nan.0+5.0i)     \ev  \schtrue
\end{scheme}
\end{entry}


\begin{entry}{%
\proto{=}{ \vri{z} \vrii{z} \vriii{z} \dotsfoo}{procedure}
\proto{<}{ \vri{x} \vrii{x} \vriii{x} \dotsfoo}{procedure}
\proto{>}{ \vri{x} \vrii{x} \vriii{x} \dotsfoo}{procedure}
\proto{<=}{ \vri{x} \vrii{x} \vriii{x} \dotsfoo}{procedure}
\proto{>=}{ \vri{x} \vrii{x} \vriii{x} \dotsfoo}{procedure}}

These procedures return \schtrue{} if their arguments are (respectively):
equal, monotonically increasing, monotonically decreasing,
monotonically nondecreasing, or monotonically nonincreasing,
and \schfalse{} otherwise.
If any of the arguments are +nan.0, all the predicates return \schfalse{}.

These predicates are required to be transitive.

\begin{note}
The traditional implementations of these predicates in Lisp-like
languages are not transitive.
\end{note}

\begin{note}
While it is not an error to compare \tupe{inexact} numbers using these
predicates, the results are unreliable because a small inaccuracy
can affect the result; this is especially true of \ide{=} and \ide{zero?}.
When in doubt, consult a numerical analyst.
\end{note}

\end{entry}

\begin{entry}{%
\proto{zero?}{ \vr{z}}{procedure}
\proto{positive?}{ \vr{x}}{procedure}
\proto{negative?}{ \vr{x}}{procedure}
\proto{odd?}{ \vr{n}}{procedure}
\proto{even?}{ \vr{n}}{procedure}}

These numerical predicates test a number for a particular property,
returning \schtrue{} or \schfalse.  See note above.

\end{entry}

\begin{entry}{%
\proto{max}{ \vri{x} \vrii{x} \dotsfoo}{procedure}
\proto{min}{ \vri{x} \vrii{x} \dotsfoo}{procedure}}

These procedures return the maximum or minimum of their arguments.

\begin{scheme}
(max 3 4)              \ev  4    ; exact
(max 3.9 4)            \ev  4.0  ; inexact%
\end{scheme}

\begin{note}
If any argument is inexact, then the result will also be inexact (unless
the procedure can prove that the inaccuracy is not large enough to affect the
result, which is possible only in unusual implementations).  If {\cf min} or
{\cf max} is used to compare numbers of mixed exactness, and the numerical
value of the result cannot be represented as an inexact number without loss of
accuracy, then the procedure may report a violation of an implementation
restriction.
\end{note}

\end{entry}


\begin{entry}{%
\proto{+}{ \vri{z} \dotsfoo}{procedure}
\proto{*}{ \vri{z} \dotsfoo}{procedure}}

These procedures return the sum or product of their arguments.

\begin{scheme}
(+ 3 4)                 \ev  7
(+ 3)                   \ev  3
(+)                     \ev  0
(* 4)                   \ev  4
(*)                     \ev  1%
\end{scheme} 
 
\end{entry}


\begin{entry}{%
\proto{-}{ \vri{z} \vrii{z}}{procedure}
\rproto{-}{ \vr{z}}{procedure}
\rproto{-}{ \vri{z} \vrii{z} \dotsfoo}{procedure}
\proto{/}{ \vri{z} \vrii{z}}{procedure}
\rproto{/}{ \vr{z}}{procedure}
\rproto{/}{ \vri{z} \vrii{z} \dotsfoo}{procedure}}

With two or more arguments, these procedures return the difference or
quotient of their arguments, associating to the left.  With one argument,
however, they return the additive or multiplicative inverse of their argument.

\begin{scheme}
(- 3 4)                 \ev  -1
(- 3 4 5)               \ev  -6
(- 3)                   \ev  -3
(/ 3 4 5)               \ev  3/20
(/ 3)                   \ev  1/3%
\end{scheme}

\end{entry}


\begin{entry}{%
\proto{abs}{ x}{procedure}}

{\cf Abs} returns the absolute value of its argument.  
\begin{scheme}
(abs -7)                \ev  7
\end{scheme}
\end{entry}


\begin{entry}{%
\proto{floor/}{ \vri{n} \vrii{n}}{procedure}
\proto{floor-quotient}{ \vri{n} \vrii{n}}{procedure}
\proto{floor-remainder}{ \vri{n} \vrii{n}}{procedure}
\proto{ceiling/}{ \vri{n} \vrii{n}}{procedure}
\proto{ceiling-quotient}{ \vri{n} \vrii{n}}{procedure}
\proto{ceiling-remainder}{ \vri{n} \vrii{n}}{procedure}
\proto{centered/}{ \vri{n} \vrii{n}}{procedure}
\proto{centered-quotient}{ \vri{n} \vrii{n}}{procedure}
\proto{centered-remainder}{ \vri{n} \vrii{n}}{procedure}
\proto{truncate/}{ \vri{n} \vrii{n}}{procedure}
\proto{truncate-quotient}{ \vri{n} \vrii{n}}{procedure}
\proto{truncate-remainder}{ \vri{n} \vrii{n}}{procedure}
\proto{round/}{ \vri{n} \vrii{n}}{procedure}
\proto{round-quotient}{ \vri{n} \vrii{n}}{procedure}
\proto{round-remainder}{ \vri{n} \vrii{n}}{procedure}
\proto{euclidean/}{ \vri{n} \vrii{n}}{procedure}
\proto{euclidean-quotient}{ \vri{n} \vrii{n}}{procedure}
\proto{euclidean-remainder}{ \vri{n} \vrii{n}}{procedure}
\proto{centered/}{ \vri{n} \vrii{n}}{procedure}
\proto{centered-quotient}{ \vri{n} \vrii{n}}{procedure}
\proto{centered-remainder}{ \vri{n} \vrii{n}}{procedure}}

These procedures, all in the division library, implement
number-theoretic (integer) division.  It is an error if \vrii{n} is zero.
The procedures ending in {\cf /} return two integers; the other
procedures return an integer.  All the procedures compute a
quotient \vr{n_q} and remainder \vr{n_r} such that
$\vri{n} = \vrii{n} \vr{n_q} + \vr{n_r}$.  For each of the six
division operators, there are three procedures defined as follows:

\begin{scheme}
(\hyper{operator}/ \vri{n} \vrii{n})             \ev \vr{n_q} \vr{n_r}
(\hyper{operator}-quotient \vri{n} \vrii{n})     \ev \vr{n_q}
(\hyper{operator}-remainder \vri{n} \vrii{n})    \ev \vr{n_r}
\end{scheme}

The remainder \vr{n_r} is determined by the choice of integer
\vr{n_q}: $\vr{n_r} = \vri{n} - \vrii{n} \vr{n_q}$.  Each set of
operators uses a different choice of \vr{n_q}:

\begin{tabular}{l l}
\texttt{ceiling}   & $\vr{n_q} = \lceil\vri{n} / \vrii{n}\rceil$ \\
\texttt{floor}     & $\vr{n_q} = \lfloor\vri{n} / \vrii{n}\rfloor$ \\
\texttt{truncate}  & $\vr{n_q} = \text{truncate}(\vri{n} / \vrii{n})$ \\
\texttt{round}     & $\vr{n_q} = [\vri{n} / \vrii{n}]$ \\
\texttt{euclidean} & $\text{if } \vrii{n} > 0, \vr{n_q} = \lfloor\vri{n} / \vrii{n}\rfloor; \text{ if } \vrii{n} < 0, \vr{n_q} = \lceil\vri{n} / \vrii{n}\rceil$ \\
\texttt{centered}  &  \text{ choose } $\vr{n_q}$ \text{ such that } $-|d/2| <= \vr{n_r} < |d/2|$ \\
\end{tabular}

For any of the operators, and for integers \vri{n} and \vrii{n}
with \vrii{n} not equal to 0,
\begin{scheme}
     (= \vri{n} (+ (* \vrii{n} (\hyper{operator}-quotient \vri{n} \vrii{n}))
           (\hyper{operator}-remainder \vri{n} \vrii{n})))
                                 \ev  \schtrue%
\end{scheme}
provided all numbers involved in that computation are exact.

\end{entry}


\begin{entry}{%
\proto{quotient}{ \vri{n} \vrii{n}}{procedure}
\proto{remainder}{ \vri{n} \vrii{n}}{procedure}
\proto{modulo}{ \vri{n} \vrii{n}}{procedure}}

{\cf Quotient} and {\cf remainder} are equivalent to {\cf
  truncate-quotient} and {\cf truncate-remainder}, respectively.  {\cf
  Modulo} is equivalent to {\cf floor-remainder}.

\begin{scheme}
(modulo 13 4)           \ev  1
(remainder 13 4)        \ev  1

(modulo -13 4)          \ev  3
(remainder -13 4)       \ev  -1

(modulo 13 -4)          \ev  -3
(remainder 13 -4)       \ev  1

(modulo -13 -4)         \ev  -1
(remainder -13 -4)      \ev  -1

(remainder -13 -4.0)    \ev  -1.0  ; inexact%
\end{scheme}

\begin{note}
These procedures are provided for backward compatibility with earlier
versions of this report.
\end{note}
\end{entry}

\begin{entry}{%
\proto{gcd}{ \vri{n} \dotsfoo}{procedure}
\proto{lcm}{ \vri{n} \dotsfoo}{procedure}}

These procedures return the greatest common divisor or least common
multiple of their arguments.  The result is always non-negative.

\begin{scheme}
(gcd 32 -36)            \ev  4
(gcd)                   \ev  0
(lcm 32 -36)            \ev  288
(lcm 32.0 -36)          \ev  288.0  ; inexact
(lcm)                   \ev  1%
\end{scheme}

\end{entry}


\begin{entry}{%
\proto{numerator}{ \vr{q}}{procedure}
\proto{denominator}{ \vr{q}}{procedure}}

These procedures return the numerator or denominator of their
argument; the result is computed as if the argument was represented as
a fraction in lowest terms.  The denominator is always positive.  The
denominator of 0 is defined to be 1.
\todo{More description and examples needed.}
\begin{scheme}
(numerator (/ 6 4))  \ev  3
(denominator (/ 6 4))  \ev  2
(denominator
  (exact->inexact (/ 6 4))) \ev 2.0%
\end{scheme}

\end{entry}


\begin{entry}{%
\proto{floor}{ x}{procedure}
\proto{ceiling}{ x}{procedure}
\proto{truncate}{ x}{procedure}
\proto{round}{ x}{procedure}
}

These procedures return integers.
\vest {\cf Floor} returns the largest integer not larger than \vr{x}.
{\cf Ceiling} returns the smallest integer not smaller than~\vr{x}.
{\cf Truncate} returns the integer closest to \vr{x} whose absolute
value is not larger than the absolute value of \vr{x}.  {\cf Round} returns the
closest integer to \vr{x}, rounding to even when \vr{x} is halfway between two
integers.

\begin{rationale}
{\cf Round} rounds to even for consistency with the default rounding
mode specified by the IEEE floating-point standard.
\end{rationale}

\begin{note}
If the argument to one of these procedures is inexact, then the result
will also be inexact.  If an exact value is needed, the
result can be passed to the {\cf inexact->exact} procedure.
\end{note}

\begin{scheme}
(floor -4.3)          \ev  -5.0
(ceiling -4.3)        \ev  -4.0
(truncate -4.3)       \ev  -4.0
(round -4.3)          \ev  -4.0

(floor 3.5)           \ev  3.0
(ceiling 3.5)         \ev  4.0
(truncate 3.5)        \ev  3.0
(round 3.5)           \ev  4.0  ; inexact

(round 7/2)           \ev  4    ; exact
(round 7)             \ev  7%
\end{scheme}

\end{entry}

\begin{entry}{%
\proto{rationalize}{ x y}{procedure}
}

{\cf Rationalize} returns the {\em simplest} rational number
differing from \vr{x} by no more than \vr{y}.  A rational number $r_1$ is
{\em simpler} \mainindex{simplest rational} than another rational number
$r_2$ if $r_1 = p_1/q_1$ and $r_2 = p_2/q_2$ (in lowest terms) and $|p_1|
\leq |p_2|$ and $|q_1| \leq |q_2|$.  Thus $3/5$ is simpler than $4/7$.
Although not all rationals are comparable in this ordering (consider $2/7$
and $3/5$) any interval contains a rational number that is simpler than
every other rational number in that interval (the simpler $2/5$ lies
between $2/7$ and $3/5$).  Note that $0 = 0/1$ is the simplest rational of
all.

\begin{scheme}
(rationalize
  (inexact->exact .3) 1/10)  \ev 1/3    ; exact
(rationalize .3 1/10)        \ev \#i1/3  ; inexact%
\end{scheme}

\end{entry}

\begin{entry}{%
\proto{exp}{ \vr{z}}{inexact library procedure}
\proto{log}{ \vr{z}}{inexact library procedure}
\proto{sin}{ \vr{z}}{inexact library procedure}
\proto{cos}{ \vr{z}}{inexact library procedure}
\proto{tan}{ \vr{z}}{inexact library procedure}
\proto{asin}{ \vr{z}}{inexact library procedure}
\proto{acos}{ \vr{z}}{inexact library procedure}
\proto{atan}{ \vr{z}}{inexact library procedure}
\rproto{atan}{ \vr{y} \vr{x}}{inexact library procedure}}

These procedures 
compute the usual transcendental functions.  {\cf Log}
computes the natural logarithm of \vr{z} (not the base ten logarithm).
{\cf Asin}, {\cf acos}, and {\cf atan} compute arcsine ($\sin^{-1}$),
arccosine ($\cos^{-1}$), and arctangent ($\tan^{-1}$), respectively.
The two-argument variant of {\cf atan} computes {\tt (angle
(make-rectangular \vr{x} \vr{y}))} (see below), even in implementations
that don't support the {\cf complex} library.

In general, the mathematical functions log, arcsine, arccosine, and
arctangent are multiply defined.
The value of $\log z$ is defined to be the one whose imaginary
part lies in the range from $-\pi$ (exclusive) to $\pi$ (inclusive).
The value of $\log 0$ is undefined.
With $\log$ defined this way, the values of $\sin^{-1} z$, $\cos^{-1} z$,
and $\tan^{-1} z$ are according to the following formul\ae:
$$\sin^{-1} z = -i \log (i z + \sqrt{1 - z^2})$$
$$\cos^{-1} z = \pi / 2 - \sin^{-1} z$$
$$\tan^{-1} z = (\log (1 + i z) - \log (1 - i z)) / (2 i)$$

The above specification follows~\cite{CLtL}, which in turn
cites~\cite{Penfield81}; refer to these sources for more detailed
discussion of branch cuts, boundary conditions, and implementation of
these functions.  When it is possible these procedures produce a real
result from a real argument.


\end{entry}


\begin{entry}{%
\proto{sqrt}{ \vr{z}}{inexact library procedure}}

Returns the principal square root of \vr{z}.  The result will have
either positive real part, or zero real part and non-negative imaginary
part.
\end{entry}


\begin{entry}{%
\proto{exact-integer-sqrt}{ k}{procedure}}

Returns two non-negative exact integers $s$ and $r$ where
$\var{k} = s^2 + r$ and $\var{k} < (s+1)^2$.

\begin{scheme}
(exact-integer-sqrt 4) \ev 2 0
(exact-integer-sqrt 5) \ev 2 1
\end{scheme}
\end{entry}


\begin{entry}{%
\proto{expt}{ \vri{z} \vrii{z}}{procedure}}

Returns \vri{z} raised to the power \vrii{z}.  For nonzero z1, this is
$${z_1}^{z_2} = e^{z_2 \log {z_1}}$$
$0.0^z$ is $1.0$ if $z = 0.0$, and $0.0$ if {\cf (real-part z)} is positive.
For other cases in which the first argument is zero, either an error is
signalled or an unspecified number is returned.
\end{entry}




\begin{entry}{%
\proto{make-rectangular}{ \vri{x} \vrii{x}}{complex library procedure}
\proto{make-polar}{ \vriii{x} \vriv{x}}{complex library procedure}
\proto{real-part}{ \vr{z}}{complex library procedure}
\proto{imag-part}{ \vr{z}}{complex library procedure}
\proto{magnitude}{ \vr{z}}{complex library procedure}
\proto{angle}{ \vr{z}}{complex library procedure}}

Suppose \vri{x}, \vrii{x}, \vriii{x}, and \vriv{x} are
real numbers and \vr{z} is a complex number such that
 $$ \vr{z} = \vri{x} + \vrii{x}\hbox{$i$}
 = \vriii{x} \cdot e^{{\displaystyle{\hbox{$i$}} \vriv{x}}}$$
Then
\begin{scheme}
(make-rectangular \vri{x} \vrii{x}) \ev \vr{z}
(make-polar \vriii{x} \vriv{x})     \ev \vr{z}
(real-part \vr{z})                  \ev \vri{x}
(imag-part \vr{z})                  \ev \vrii{x}
(magnitude \vr{z})                  \ev $|\vriii{x}|$
(angle \vr{z})                      \ev $x_{angle}$
\end{scheme}
where $-\pi < x_{angle} \le \pi$ with $x_{angle} = \vriv{x} + 2\pi n$
for some integer $n$.

{\cf Make-polar} may return an inexact complex number even if its
arguments are exact.


\begin{rationale}
{\cf Magnitude} is the same as \ide{abs} for a real argument,
but {\cf abs} is in the {\cf base} library, whereas
{\cf magnitude} is in the optional {\cf complex} library.
\end{rationale}

\end{entry}


\begin{entry}{%
\proto{exact->inexact}{ \vr{z}}{procedure}
\proto{inexact->exact}{ \vr{z}}{procedure}}

The procedure {\cf exact\coerce{}inexact} returns an \tupe{inexact} representation of \vr{z}.
The value returned is the
\tupe{inexact} number that is numerically closest to the argument.  
For inexact arguments, the result is the same as the argument. For exact
complex numbers, the result is a complex number whose real and imaginary
parts are the result of applying {\cf exact\coerce{}inexact} to the real
and imaginary parts of the argument, respectively.
If an \tupe{exact} argument has no reasonably close \tupe{inexact} equivalent,
then a violation of an implementation restriction may be reported.

The procedure {\cf inexact\coerce{}exact} returns an \tupe{exact} representation of
\vr{z}.  The value returned is the \tupe{exact} number that is numerically
closest to the argument.
For exact arguments, the result is the same as the argument. For inexact
non-integral real arguments, the implementation may return a rational
approximation, or may report an implementation violation. For inexact
complex arguments, the result is a complex number whose real and
imaginary parts are result of applying {\cf inexact\coerce{}exact} to the
real and imaginary parts of the argument, respectively.
If an \tupe{inexact} argument has no reasonably close \tupe{exact} equivalent,
then a violation of an implementation restriction may be reported.


These procedures implement the natural one-to-one correspondence between
\tupe{exact} and \tupe{inexact} integers throughout an
implementation-dependent range.  See section~\ref{restrictions}.

\begin{note}
The names {\cf exact\coerce{}inexact} and {\cf
inexact\coerce{}exact} are historical anomalies; the argument to each of
these procedures may be either exact or inexact.
\end{note}

\end{entry}

\medskip

\subsection{Numerical input and output}

\begin{entry}{%
\proto{number->string}{ z}{procedure}
\rproto{number->string}{ z radix}{procedure}}

It is an error if\vr{radix} is not one of 2, 8, 10, or 16.  If omitted,
\vr{radix} defaults to 10.
The procedure {\cf number\coerce{}string} takes a
number and a radix and returns as a string an external representation of
the given number in the given radix such that
\begin{scheme}
(let ((number \vr{number})
      (radix \vr{radix}))
  (eqv? number
        (string->number (number->string number
                                        radix)
                        radix)))
\end{scheme}
is true.  It is an error if no possible result makes this expression true.

If \vr{z} is inexact, the radix is 10, and the above expression
can be satisfied by a result that contains a decimal point,
then the result contains a decimal point and is expressed using the
minimum number of digits (exclusive of exponent and trailing
zeroes) needed to make the above expression
true~\cite{howtoprint,howtoread};
otherwise the format of the result is unspecified.

The result returned by {\cf number\coerce{}string}
never contains an explicit radix prefix.

\begin{note}
The error case can occur only when \vr{z} is not a complex number
or is a complex number with a non-rational real or imaginary part.
\end{note}

\begin{rationale}
If \vr{z} is an inexact number and
the radix is 10, then the above expression is normally satisfied by
a result containing a decimal point.  The unspecified case
allows for infinities, NaNs, and unusual representations.
\end{rationale}

\end{entry}


\begin{entry}{%
\proto{string->number}{ string}{procedure}
\rproto{string->number}{ string radix}{procedure}}


Returns a number of the maximally precise representation expressed by the
given \vr{string}.
It is an error if \vr{radix} is not 2, 8, 10,
or 16.  If supplied, \vr{radix} is a default radix that will be overridden
by an explicit radix prefix in \vr{string} (e.g. {\tt "\#o177"}).  If \vr{radix}
is not supplied, then the default radix is 10.  If \vr{string} is not
a syntactically valid notation for a number, then {\cf string->number}
returns \schfalse{}.

\begin{scheme}
(string->number "100")        \ev  100
(string->number "100" 16)     \ev  256
(string->number "1e2")        \ev  100.0
(string->number "15\#\#")       \ev  1500.0%
\end{scheme}

\begin{note}
The domain of {\cf string->number} may be restricted by implementations
in the following ways.  {\cf String->number} is permitted to return
\schfalse{} whenever \vr{string} contains an explicit radix prefix.
If all numbers supported by an implementation are real, then
{\cf string->number} is permitted to return \schfalse{} whenever
\vr{string} uses the polar or rectangular notations for complex
numbers.  If all numbers are integers, then
{\cf string->number} may return \schfalse{} whenever
the fractional notation is used.  If all numbers are exact, then
{\cf string->number} may return \schfalse{} whenever
an exponent marker or explicit exactness prefix is used.
If all inexact
numbers are integers, then
{\cf string->number} may return \schfalse{} whenever
a decimal point is used.
\end{note}

\end{entry}

\section{Booleans}
\label{booleansection}

The standard boolean objects for true and false are written as
\schtrue{} and \schfalse.\sharpindex{t}\sharpindex{f}  
Alternatively, they may be written \sharpfoo{true} and \sharpfoo{false}
respectively.  What really
matters, though, are the objects that the Scheme conditional expressions
({\cf if}, {\cf cond}, {\cf and}, {\cf or}, {\cf do}) treat as
true\index{true} or false\index{false}.  The phrase ``a true value''\index{true}
(or sometimes just ``true'') means any object treated as true by the
conditional expressions, and the phrase ``a false value''\index{false} (or
``false'') means any object treated as false by the conditional expressions.

\vest Of all the standard Scheme values, only \schfalse{}
counts as false in conditional expressions.
Except for \schfalse{},
all standard Scheme values, including \schtrue,
the empty list, and all pairs, symbols, numbers, strings, vectors,
bytevectors, records, and procedures, count as true.

\begin{note}
Programmers accustomed to other dialects of Lisp need to realize that
Scheme distinguishes \schfalse{} and the empty list \index{empty list}
from each other and from the symbol \ide{nil}.
\end{note}

\vest Boolean constants evaluate to themselves, so they do not need to be quoted
in programs.

\begin{scheme}
\schtrue         \ev  \schtrue
\schfalse        \ev  \schfalse
'\schfalse       \ev  \schfalse%
\end{scheme}


\begin{entry}{%
\proto{not}{ obj}{procedure}}

{\cf Not} returns \schtrue{} if \var{obj} is false, and returns
\schfalse{} otherwise.

\begin{scheme}
(not \schtrue)   \ev  \schfalse
(not 3)          \ev  \schfalse
(not (list 3))   \ev  \schfalse
(not \schfalse)  \ev  \schtrue
(not '())        \ev  \schfalse
(not (list))     \ev  \schfalse
(not 'nil)       \ev  \schfalse%
\end{scheme}

\end{entry}


\begin{entry}{%
\proto{boolean?}{ obj}{procedure}}

{\cf Boolean?} returns \schtrue{} if \var{obj} is either \schtrue{} or
\schfalse{} and returns \schfalse{} otherwise.

\begin{scheme}
(boolean? \schfalse)  \ev  \schtrue
(boolean? 0)          \ev  \schfalse
(boolean? '())        \ev  \schfalse%
\end{scheme}

\end{entry}

 
\section{Pairs and lists}
\label{listsection}

A \defining{pair} (sometimes called a \defining{dotted pair}) is a
record structure with two fields called the car and cdr fields (for
historical reasons).  Pairs are created by the procedure {\cf cons}.
The car and cdr fields are accessed by the procedures {\cf car} and
{\cf cdr}.  The car and cdr fields are assigned by the procedures
{\cf set-car!}\ and {\cf set-cdr!}.

Pairs are used primarily to represent lists.  A \defining{list} can
be defined recursively as either the empty list\index{empty list} or a pair whose
cdr is a list.  More precisely, the set of lists is defined as the smallest
set \var{X} such that

\begin{itemize}
\item The empty list is in \var{X}.
\item If \var{list} is in \var{X}, then any pair whose cdr field contains
      \var{list} is also in \var{X}.
\end{itemize}

The objects in the car fields of successive pairs of a list are the
elements of the list.  For example, a two-element list is a pair whose car
is the first element and whose cdr is a pair whose car is the second element
and whose cdr is the empty list.  The length of a list is the number of
elements, which is the same as the number of pairs.

The empty list\mainindex{empty list} is a special object of its own type
It is not a pair, it has no elements, and its length is zero.

\begin{note}
The above definitions imply that all lists have finite length and are
terminated by the empty list.
\end{note}

The most general notation (external representation) for Scheme pairs is
the ``dotted'' notation \hbox{\cf (\vari{c} .\ \varii{c})} where
\vari{c} is the value of the car field and \varii{c} is the value of the
cdr field.  For example {\cf (4 .\ 5)} is a pair whose car is 4 and whose
cdr is 5.  Note that {\cf (4 .\ 5)} is the external representation of a
pair, not an expression that evaluates to a pair.

A more streamlined notation can be used for lists: the elements of the
list are simply enclosed in parentheses and separated by spaces.  The
empty list\index{empty list} is written {\tt()} .  For example,

\begin{scheme}
(a b c d e)%
\end{scheme}

and

\begin{scheme}
(a . (b . (c . (d . (e . ())))))%
\end{scheme}

are equivalent notations for a list of symbols.

A chain of pairs not ending in the empty list is called an
\defining{improper list}.  Note that an improper list is not a list.
The list and dotted notations can be combined to represent
improper lists:

\begin{scheme}
(a b c . d)%
\end{scheme}

is equivalent to

\begin{scheme}
(a . (b . (c . d)))%
\end{scheme}

Whether a given pair is a list depends upon what is stored in the cdr
field.  When the \ide{set-cdr!} procedure is used, an object can be a
list one moment and not the next:

\begin{scheme}
(define x (list 'a 'b 'c))
(define y x)
y                       \ev  (a b c)
(list? y)               \ev  \schtrue
(set-cdr! x 4)          \ev  \unspecified
x                       \ev  (a . 4)
(eqv? x y)              \ev  \schtrue
y                       \ev  (a . 4)
(list? y)               \ev  \schfalse
(set-cdr! x x)          \ev  \unspecified
(list? x)               \ev  \schfalse%
\end{scheme}

Within literal expressions and representations of objects read by the
\ide{read} procedure, the forms \singlequote\hyper{datum}\schindex{'},
\backquote\hyper{datum}, {\tt,}\hyper{datum}\schindex{,}, and
{\tt,@}\hyper{datum} denote two-ele\-ment lists whose first elements are
the symbols \ide{quote}, \ide{quasiquote}, \hbox{\ide{unquote}}, and
\ide{unquote-splicing}, respectively.  The second element in each case
is \hyper{datum}.  This convention is supported so that arbitrary Scheme
programs can be represented as lists.  
That is, according to Scheme's grammar, every
\meta{expression} is also a \meta{datum} (see section~\ref{datum}).
Among other things, this permits the use of the {\cf read} procedure to
parse Scheme programs.  See section~\ref{externalreps}. 
 

\begin{entry}{%
\proto{pair?}{ obj}{procedure}}

{\cf Pair?} returns \schtrue{} if \var{obj} is a pair, and otherwise
returns \schfalse.

\begin{scheme}
(pair? '(a . b))        \ev  \schtrue
(pair? '(a b c))        \ev  \schtrue
(pair? '())             \ev  \schfalse
(pair? '\#(a b))         \ev  \schfalse%
\end{scheme}
\end{entry}


\begin{entry}{%
\proto{cons}{ \vari{obj} \varii{obj}}{procedure}}

Returns a newly allocated pair whose car is \vari{obj} and whose cdr is
\varii{obj}.  The pair is guaranteed to be different (in the sense of
{\cf eqv?}) from every existing object.

\begin{scheme}
(cons 'a '())           \ev  (a)
(cons '(a) '(b c d))    \ev  ((a) b c d)
(cons "a" '(b c))       \ev  ("a" b c)
(cons 'a 3)             \ev  (a . 3)
(cons '(a b) 'c)        \ev  ((a b) . c)%
\end{scheme}
\end{entry}


\begin{entry}{%
\proto{car}{ pair}{procedure}}

Returns the contents of the car field of \var{pair}.  Note that it is an
error to take the car of the empty list\index{empty list}.

\begin{scheme}
(car '(a b c))          \ev  a
(car '((a) b c d))      \ev  (a)
(car '(1 . 2))          \ev  1
(car '())               \ev  \scherror%
\end{scheme}
 
\end{entry}


\begin{entry}{%
\proto{cdr}{ pair}{procedure}}

Returns the contents of the cdr field of \var{pair}.
Note that it is an error to take the cdr of the empty list.

\begin{scheme}
(cdr '((a) b c d))      \ev  (b c d)
(cdr '(1 . 2))          \ev  2
(cdr '())               \ev  \scherror%
\end{scheme}
 
\end{entry}


\begin{entry}{%
\proto{set-car!}{ pair obj}{procedure}}

Stores \var{obj} in the car field of \var{pair}.
The value returned by {\cf set-car!}\ is unspecified.  % <!>

\begin{scheme}
(define (f) (list 'not-a-constant-list))
(define (g) '(constant-list))
(set-car! (f) 3)             \ev  \unspecified
(set-car! (g) 3)             \ev  \scherror%
\end{scheme}

\end{entry}


\begin{entry}{%
\proto{set-cdr!}{ pair obj}{procedure}}

Stores \var{obj} in the cdr field of \var{pair}.
The value returned by {\cf set-cdr!}\ is unspecified.  % <!>

\end{entry}

\setbox0\hbox{\tt(cadr \var{pair})}
\setbox1\hbox{procedure}


\begin{entry}{%
\proto{caar}{ pair}{procedure}
\proto{cadr}{ pair}{procedure}
\pproto{\hbox to 1\wd0 {\hfil$\vdots$\hfil}}{\hbox to 1\wd1 {\hfil$\vdots$\hfil}}
\proto{cdddar}{ pair}{procedure}
\proto{cddddr}{ pair}{procedure}}

These procedures are compositions of {\cf car} and {\cf cdr}, where
for example {\cf caddr} could be defined by

\begin{scheme}
(define caddr (lambda (x) (car (cdr (cdr x))))){\rm.}%
\end{scheme}

Arbitrary compositions, up to four deep, are provided.  There are
twenty-eight of these procedures in all.

\end{entry}


\begin{entry}{%
\proto{null?}{ obj}{procedure}}

Returns \schtrue{} if \var{obj} is the empty list\index{empty list},
otherwise returns \schfalse.

\end{entry}

\begin{entry}{%
\proto{list?}{ obj}{procedure}}

Returns \schtrue{} if \var{obj} is a list.  Otherwise, it returns \schfalse{}.
By definition, all lists have finite length and are terminated by
the empty list.

\begin{scheme}
        (list? '(a b c))     \ev  \schtrue
        (list? '())          \ev  \schtrue
        (list? '(a . b))     \ev  \schfalse
        (let ((x (list 'a)))
          (set-cdr! x x)
          (list? x))         \ev  \schfalse%
\end{scheme}


\end{entry}

\begin{entry}{%
\proto{make-list}{ k}{procedure}
\rproto{make-list}{ k fill}{procedure}}

Returns a newly allocated list of \var{k} elements.  If a second
argument is given, then each element is initialized to \var{fill}.
Otherwise the initial contents of each element is unspecified.

\begin{scheme}
(make-list 2 3)   \ev   (3 3)
\end{scheme}

\end{entry}



\begin{entry}{%
\proto{list}{ \var{obj} \dotsfoo}{procedure}}

Returns a newly allocated list of its arguments.

\begin{scheme}
(list 'a (+ 3 4) 'c)            \ev  (a 7 c)
(list)                          \ev  ()%
\end{scheme}
\end{entry}


\begin{entry}{%
\proto{length}{ list}{procedure}}

Returns the length of \var{list}.

\begin{scheme}
(length '(a b c))               \ev  3
(length '(a (b) (c d e)))       \ev  3
(length '())                    \ev  0%
\end{scheme}


\end{entry}


\begin{entry}{%
\proto{append}{ list \dotsfoo}{procedure}}

Returns a list consisting of the elements of the first \var{list}
followed by the elements of the other \var{list}s.

\begin{scheme}
(append '(x) '(y))              \ev  (x y)
(append '(a) '(b c d))          \ev  (a b c d)
(append '(a (b)) '((c)))        \ev  (a (b) (c))%
\end{scheme}

The resulting list is always newly allocated, except that it shares
structure with the last \var{list} argument.  The last argument may
actually be any object; an improper list results if the last argument is not a
proper list.  

\begin{scheme}
(append '(a b) '(c . d))        \ev  (a b c . d)
(append '() 'a)                 \ev  a%
\end{scheme}
\end{entry}


\begin{entry}{%
\proto{reverse}{ list}{procedure}}

Returns a newly allocated list consisting of the elements of \var{list}
in reverse order.

\begin{scheme}
(reverse '(a b c))              \ev  (c b a)
(reverse '(a (b c) d (e (f))))  \lev  ((e (f)) d (b c) a)%
\end{scheme}
\end{entry}


\begin{entry}{%
\proto{list-tail}{ list \vr{k}}{procedure}}

Returns the sublist of \var{list} obtained by omitting the first \vr{k}
elements.  It is an error if \var{list} has fewer than \vr{k} elements.
{\cf List-tail} could be defined by

\begin{scheme}
(define list-tail
  (lambda (x k)
    (if (zero? k)
        x
        (list-tail (cdr x) (- k 1)))))%
\end{scheme} 
\end{entry}


\begin{entry}{%
\proto{list-ref}{ list \vr{k}}{procedure}}

Returns the \vr{k}th element of \var{list}.  (This is the same
as the car of {\tt(list-tail \var{list} \vr{k})}.)
It is an error if \var{list} has fewer than \vr{k} elements.

\begin{scheme}
(list-ref '(a b c d) 2)                 \ev  c
(list-ref '(a b c d)
          (inexact->exact (round 1.8))) \lev  c%
\end{scheme}
\end{entry}

\begin{entry}{%
\proto{list-set!}{ list k obj}{procedure}}

\domain{It is an error if \vr{k} is not a valid index of \var{list}.}
{\cf List-set!} stores \var{obj} in element \vr{k} of \var{list}.
The value returned by {\cf list-set!}\ is unspecified.  % <!>

\begin{scheme}
(let ((lst (list 0 '(2 2 2 2) "Anna")))
  (list-set! lst 1 '("Sue" "Sue"))
  vec)      \lev  (0 ("Sue" "Sue") "Anna")

(list-set! '(0 1 2) 1 "doe")  \lev  \scherror  ; constant list%
\end{scheme}
\end{entry}




\begin{entry}{%
\proto{memq}{ obj list}{procedure}
\proto{memv}{ obj list}{procedure}
\proto{member}{ obj list}{procedure}
\proto{member}{ obj list compare}{procedure}}

These procedures return the first sublist of \var{list} whose car is
\var{obj}, where the sublists of \var{list} are the non-empty lists
returned by {\tt (list-tail \var{list} \var{k})} for \var{k} less
than the length of \var{list}.  If
\var{obj} does not occur in \var{list}, then \schfalse{} (not the empty list) is
returned.  {\cf Memq} uses {\cf eq?}\ to compare \var{obj} with the elements of
\var{list}, while {\cf memv} uses {\cf eqv?} and 
{\cf member} uses \var{compare}, if given, and {\cf equal?} otherwise.

\begin{scheme}
(memq 'a '(a b c))              \ev  (a b c)
(memq 'b '(a b c))              \ev  (b c)
(memq 'a '(b c d))              \ev  \schfalse
(memq (list 'a) '(b (a) c))     \ev  \schfalse
(member (list 'a)
        '(b (a) c))             \ev  ((a) c)
(member "B"
        '("a" "b" "c")
        string-ci=?)            \ev  ("b" "c")
(memq 101 '(100 101 102))       \ev  \unspecified
(memv 101 '(100 101 102))       \ev  (101 102)%
\end{scheme} 
 
\end{entry}


\begin{entry}{%
\proto{assq}{ obj alist}{procedure}
\proto{assv}{ obj alist}{procedure}
\proto{assoc}{ obj alist}{procedure}
\proto{assoc}{ obj alist compare}{procedure}}

\domain{It is an error if\var{alist} (for ``association list'') is not a list of
pairs.}  These procedures find the first pair in \var{alist} whose car field is \var{obj},
and returns that pair.  If no pair in \var{alist} has \var{obj} as its
car, then \schfalse{} (not the empty list) is returned.  {\cf Assq} uses
{\cf eq?}\ to compare \var{obj} with the car fields of the pairs in \var{alist},
while {\cf assv} uses {\cf eqv?}\ and {\cf assoc} uses \var{compare} if given
and {\cf equal?} otherwise.

\begin{scheme}
(define e '((a 1) (b 2) (c 3)))
(assq 'a e)     \ev  (a 1)
(assq 'b e)     \ev  (b 2)
(assq 'd e)     \ev  \schfalse
(assq (list 'a) '(((a)) ((b)) ((c))))
                \ev  \schfalse
(assoc (list 'a) '(((a)) ((b)) ((c))))   
                           \ev  ((a))
(assoc 2.0 '((1 1) (2 4) (3 9)) =)
                           \ev (2 4)
(assq 5 '((2 3) (5 7) (11 13)))    
                           \ev  \unspecified
(assv 5 '((2 3) (5 7) (11 13)))    
                           \ev  (5 7)%
\end{scheme}


\begin{rationale}
Although they are often used as predicates,
{\cf memq}, {\cf memv}, {\cf member}, {\cf assq}, {\cf assv}, and {\cf assoc} do not
have question marks in their names because they return 
potentially useful values rather than just \schtrue{} or \schfalse{}.
\end{rationale}
\end{entry}

\begin{entry}{%
\proto{list-copy}{ list}{procedure}}

Returns a newly allocated copy of the given \var{list}.
Only the spine is copied; the cars of \var{list} are not.

\end{entry}


\section{Symbols}
\label{symbolsection}

Symbols are objects whose usefulness rests on the fact that two
symbols are identical (in the sense of {\cf eqv?}) if and only if their
names are spelled the same way.  For instance, they can be used
the way enumerated values are used in other languages.

\vest The rules for writing a symbol are exactly the same as the rules for
writing an identifier; see sections~\ref{syntaxsection}
and~\ref{identifiersyntax}.

\vest It is guaranteed that any symbol that has been returned as part of
a literal expression, or read using the {\cf read} procedure, and
subsequently written out using the {\cf write} procedure, will read back
in as the identical symbol (in the sense of {\cf eqv?}).

\begin{note}
Some implementations have values known as ``uninterned symbols,'' which
defeat write/read invariance,
and also violate the rule that two symbols are the same
if and only if their names are spelled the same.
\end{note}


\begin{entry}{%
\proto{symbol?}{ obj}{procedure}}

Returns \schtrue{} if \var{obj} is a symbol, otherwise returns \schfalse.

\begin{scheme}
(symbol? 'foo)          \ev  \schtrue
(symbol? (car '(a b)))  \ev  \schtrue
(symbol? "bar")         \ev  \schfalse
(symbol? 'nil)          \ev  \schtrue
(symbol? '())           \ev  \schfalse
(symbol? \schfalse)     \ev  \schfalse%
\end{scheme}
\end{entry}


\begin{entry}{%
\proto{symbol->string}{ symbol}{procedure}}

Returns the name of \var{symbol} as a string.  It is an error
to apply mutation procedures like \ide{string-set!} to strings returned
by this procedure.

\begin{scheme}
(symbol->string 'flying-fish)     
                                  \ev  "flying-fish"
(symbol->string 'Martin)          \ev  "Martin"
(symbol->string
   (string->symbol "Malvina"))     
                                  \ev  "Malvina"%
\end{scheme}
\end{entry}


\begin{entry}{%
\proto{string->symbol}{ string}{procedure}}

Returns the symbol whose name is \var{string}.  This procedure can
create symbols with names containing special characters that would
require escaping when written.

\begin{scheme}
(string->symbol "mISSISSIppi")  \lev%
  mISSISSIppi
(eq? 'bitBlt (string->symbol "bitBlt"))     \lev  \schtrue
(eq? 'JollyWog
     (string->symbol
       (symbol->string 'JollyWog)))  \lev  \schtrue
(string=? "K. Harper, M.D."
          (symbol->string
            (string->symbol "K. Harper, M.D.")))  \lev  \schtrue%
\end{scheme}

\end{entry}


\section{Characters}
\label{charactersection}

Characters are objects that represent printed characters such as
letters and digits.  
All Scheme implementations must support at least the ASCII character
repertoire: that is, Unicode characters U+0000 through U+007F.
Implementations may support any other Unicode characters they see fit,
and may also support non-Unicode characters as well.
Except as otherwise specified, the result of applying any of the
following procedures to a non-Unicode character is implementation-dependent.

Characters are written using the notation \sharpsign\backwhack\hyper{character}
or \sharpsign\backwhack\hyper{character name} or
\sharpsign\backwhack{}x\meta{hex scalar value}.

Here are some examples:

$$
\begin{tabular}{ll}
{\tt \#\backwhack{}a}&; lower case letter\\
{\tt \#\backwhack{}A}&; upper case letter\\
{\tt \#\backwhack{}(}&; left parenthesis\\
{\tt \#\backwhack{} }&; the space character\\
{\tt \#\backwhack{}iota}&; $\iota$ (if supported)\\
{\tt \#\backwhack{}x03BB}&; $\lambda$ (if supported)\\
\end{tabular}
$$

The following character names must be supported
by all implementations:

$$
\begin{tabular}{ll}
{\tt \#\backwhack{}alarm}&; \textrm{U+0007}\\
{\tt \#\backwhack{}backspace}&; \textrm{U+0008}\\
{\tt \#\backwhack{}delete}&; \textrm{U+007F}\\
{\tt \#\backwhack{}escape}&; \textrm{U+001B}\\
{\tt \#\backwhack{}newline}&; the linefeed character, \textrm{U+000A}\\
{\tt \#\backwhack{}null}&; the null character, \textrm{U+0000}\\
{\tt \#\backwhack{}return}&; the return character, \textrm{U+000D}\\
{\tt \#\backwhack{}space}&; the preferred way to write a space\\
{\tt \#\backwhack{}tab}&; the tab character, \textrm{U+0009}\\
\end{tabular}
$$

Case is significant in \sharpsign\backwhack\hyper{character}, and in
\sharpsign\backwhack{\rm$\langle$character name$\rangle$},
but not in {\cf\sharpsign\backwhack{}x}\meta{hex scalar value}.  
If \hyper{character} in
\sharpsign\backwhack\hyper{character} is alphabetic, then the character
following \hyper{character} must be a delimiter character such as a
space or parenthesis.  This rule resolves the ambiguous case where, for
example, the sequence of characters ``{\tt\sharpsign\backwhack space}''
could be taken to be either a representation of the space character or a
representation of the character ``{\tt\sharpsign\backwhack s}'' followed
by a representation of the symbol ``{\tt pace}.''

\todo{Fix}
Characters written in the \sharpsign\backwhack{} notation are self-evaluating.
That is, they do not have to be quoted in programs.  

\vest Some of the procedures that operate on characters ignore the
difference between upper case and lower case.  The procedures that
ignore case have \hbox{``{\tt -ci}''} (for ``case
insensitive'') embedded in their names.


\begin{entry}{%
\proto{char?}{ obj}{procedure}}

Returns \schtrue{} if \var{obj} is a character, otherwise returns \schfalse.

\end{entry}


\begin{entry}{%
\proto{char=?}{ \vri{char} \vrii{char} \vriii{char} \dotsfoo}{procedure}
\proto{char<?}{ \vri{char} \vrii{char} \vriii{char} \dotsfoo}{procedure}
\proto{char>?}{ \vri{char} \vrii{char} \vriii{char} \dotsfoo}{procedure}
\proto{char<=?}{ \vri{char} \vrii{char} \vriii{char} \dotsfoo}{procedure}
\proto{char>=?}{ \vri{char} \vrii{char} \vriii{char} \dotsfoo}{procedure}}

\label{characterequality}

These procedures return \schtrue{} if the Unicode codepoints corresponding to their arguments are (respectively):
equal, monotonically increasing, monotonically decreasing,
monotonically nondecreasing, or monotonically nonincreasing.

These predicates are required to be transitive.

These procedures impose a total ordering on the set of characters which
is the same as the Unicode code point ordering.  This is true independent
of whether the implementation uses the Unicode representation internally.

\end{entry}


\begin{entry}{%
\proto{char-ci=?}{ \vri{char} \vrii{char} \vriii{char} \dotsfoo}{char library procedure}
\proto{char-ci<?}{ \vri{char} \vrii{char} \vriii{char} \dotsfoo}{char library procedure}
\proto{char-ci>?}{ \vri{char} \vrii{char} \vriii{char} \dotsfoo}{char library procedure}
\proto{char-ci<=?}{ \vri{char} \vrii{char} \vriii{char} \dotsfoo}{char library procedure}
\proto{char-ci>=?}{ \vri{char} \vrii{char} \vriii{char} \dotsfoo}{char library procedure}}

These procedures are similar to {\cf char=?}\ et cetera, but they treat
upper case and lower case letters as the same.  For example, {\cf
(char-ci=?\ \#\backwhack{}A \#\backwhack{}a)} returns \schtrue.

Specifically, these procedures behave as if {\cf char-foldcase} were
applied to their arguments before they were compared.

\end{entry}


\begin{entry}{%
\proto{char-alphabetic?}{ char}{char library procedure}
\proto{char-numeric?}{ char}{char library procedure}
\proto{char-whitespace?}{ char}{char library procedure}
\proto{char-upper-case?}{ letter}{char library procedure}
\proto{char-lower-case?}{ letter}{char library procedure}}

These procedures return \schtrue{} if their arguments are alphabetic,
numeric, whitespace, upper case, or lower case characters, respectively,
otherwise they return \schfalse.  

Specifically, they must return \schtrue{} when applied to characters with
the Unicode properties Alphabetic, Numeric\_Digit, White\_Space, Uppercase, and
Lowercase respectively, and \schfalse{} when applied to any other Unicode
characters.  Note that many Unicode characters are alphabetic but neither
upper nor lower case.

\end{entry}


\begin{entry}{%
\proto{digit-value}{ char}{char library procedure}}

This procedure returns the numeric value (0 to 9) of its argument
if it is a numeric digit (that is, if {\cf char-numeric?} returns \schtrue{}),
or \schfalse{} on any other character.

\begin{scheme}
(digit-value \#\backwhack{}3) \ev 3
(digit-value \#\backwhack{}x0664) \ev 4
(digit-value \#\backwhack{}x0EA6) \ev 0
\end{scheme}
\end{entry}


\begin{entry}{%
\proto{char->integer}{ char}{procedure}
\proto{integer->char}{ \vr{n}}{procedure}}

Given a Unicode character, 
{\cf char\coerce{}integer} returns an exact integer 
between 0 and {\tt \#xD7FF} or 
between {\tt \#xE000} and {\tt \#x10FFFF} 
which is equal to the Unicode code point of that character.
Given a non-Unicode character, 
it returns an exact integer greater than {\tt \#x10FFFF}.  
This is true independent of whether the implementation uses
the Unicode representation internally.

Given an exact integer that is the value returned by
a character when {\cf char\coerce{}integer} is applied to it, {\cf integer\coerce{}char}
returns that character.
\end{entry}


\begin{entry}{%
\proto{char-upcase}{ char}{char library procedure}
\proto{char-downcase}{ char}{char library procedure}
\proto{char-foldcase}{ char}{char library procedure}}


The {\cf char-upcase} procedure, given an argument that forms the
lowercase part of a Unicode casing pair, returns the uppercase member
of the pair, provided that both characters are supported by the Scheme
implementation.  Note that language-sensitive casing pairs are not used.  If the
argument is not the lowercase part of such a pair, it is returned.

The {\cf char-downcase} procedure, given an argument that forms the
uppercase part of a Unicode casing pair, returns the lowercase member
of the pair, provided that both characters are supported by the Scheme
implementation.  Note that language-sensitive casing pairs are not used.  If the
argument is not the uppercase part of such a pair, it is returned.

The {\cf char-foldcase} procedure applies the Unicode simple
case-folding algorithm to its argument and returns the result.  Note that
language-sensitive folding is not used.  If the argument is an uppercase
letter, the result will be either a lowercase letter
or the same as the argument if the lowercase letter does not exist or
is not supported by the implementation.
See UAX \#29 (part of the Unicode Standard) for details.

Note that many Unicode lowercase characters do not have uppercase
equivalents.

\end{entry}


\section{Strings}
\label{stringsection}

Strings are sequences of characters.  
\vest Strings are written as sequences of characters enclosed within doublequotes
({\cf "}).  Within a string literal, various escape
sequences\mainindex{escape sequence} represent characters other than
themselves.  Escape sequences always start with a backslash (\backwhack{}):

\begin{itemize}
\item{\cf\backwhack{}a} : alarm, U+0007
\item{\cf\backwhack{}b} : backspace, U+0008 
\item{\cf\backwhack{}t} : character tabulation, U+0009 
\item{\cf\backwhack{}n} : linefeed, U+000A 
\item{\cf\backwhack{}r} : return, U+000D 
\item{\cf\backwhack{}}\verb|"| : doublequote, U+0022 
\item{\cf\backwhack{}\backwhack{}} : backslash, U+005C 
\item{\cf\backwhack{}\hyper{intraline whitespace}\hyper{line ending}
      \hyper{intraline whitespace}} : nothing
\item{\cf\backwhack{}x\meta{hex scalar value};} : specified character (note the
  terminating semi-colon).
\end{itemize}

The result is unspecified if any other character in a string occurs
after a backslash.

\vest Except for a line ending, any character outside of an escape
sequence stands for itself in the string literal.  A line ending which
is preceded by {\cf\backwhack{}\hyper{intraline whitespace}} expands
to nothing (along with any trailing intraline whitespace), and can be
used to indent strings for improved legibility. Any other line ending
has the same effect as inserting a {\cf\backwhack{}n} character into
the string.

Example:

\begin{scheme}
"The word \backwhack{}"recursion\backwhack{}" has many meanings."%
\end{scheme}

\vest The {\em length} of a string is the number of characters that it
contains.  This number is an exact, non-negative integer that is fixed when the
string is created.  The \defining{valid indexes} of a string are the
exact non-negative integers less than the length of the string.  The first
character of a string has index 0, the second has index 1, and so on.

\vest In phrases such as ``the characters of \var{string} beginning with
index \var{start} and ending with index \var{end},'' it is understood
that the index \var{start} is inclusive and the index \var{end} is
exclusive.  Thus if \var{start} and \var{end} are the same index, a null
substring is referred to, and if \var{start} is zero and \var{end} is
the length of \var{string}, then the entire string is referred to.

\vest Some of the procedures that operate on strings ignore the
difference between upper and lower case.  The versions that ignore case
have \hbox{``{\cf -ci}''} (for ``case insensitive'') embedded in their
names.

Implementations may forbid certain characters from appearing in strings.
For example, an implementation might support the entire Unicode repertoire,
but only allow characters U+0000 to U+00FF (the Latin-1 repertoire) in
strings.  It is an error to pass such a forbidden character to
{\cf make-string}, {\cf string}, {\cf string-set!}, or {\cf string-fill!}.


\begin{entry}{%
\proto{string?}{ obj}{procedure}}

Returns \schtrue{} if \var{obj} is a string, otherwise returns \schfalse.
\end{entry}


\begin{entry}{%
\proto{make-string}{ \vr{k}}{procedure}
\rproto{make-string}{ \vr{k} char}{procedure}}

{\cf Make-string} returns a newly allocated string of
length \vr{k}.  If \var{char} is given, then all the characters of the string
are initialized to \var{char}, otherwise the contents of the
string are unspecified.

\end{entry}

\begin{entry}{%
\proto{string}{ char \dotsfoo}{procedure}}

Returns a newly allocated string composed of the arguments.
It is analogous to {\cf list}.

\end{entry}

\begin{entry}{%
\proto{string-length}{ string}{procedure}}

Returns the number of characters in the given \var{string}.
\end{entry}


\begin{entry}{%
\proto{string-ref}{ string \vr{k}}{procedure}}

\domain{It is an error if\vr{k} is not a valid index of \var{string}.}
{\cf String-ref} returns character \vr{k} of \var{string} using zero-origin indexing.
\end{entry}


\begin{entry}{%
\proto{string-set!}{ string k char}{procedure}}

\domain{It is an error if \vr{k} is not a valid index of \var{string}.}
{\cf String-set!} stores \var{char} in element \vr{k} of \var{string}
and returns an unspecified value.  % <!>

\begin{scheme}
(define (f) (make-string 3 \sharpsign\backwhack{}*))
(define (g) "***")
(string-set! (f) 0 \sharpsign\backwhack{}?)  \ev  \unspecified
(string-set! (g) 0 \sharpsign\backwhack{}?)  \ev  \scherror
(string-set! (symbol->string 'immutable)
             0
             \sharpsign\backwhack{}?)  \ev  \scherror%
\end{scheme}

\end{entry}


\begin{entry}{%
\proto{string=?}{ \vri{char} \vrii{char} \vriii{char} \dotsfoo}{procedure}}

Returns \schtrue{} if all the strings are the same length and contain
exactly the same characters in the same positions, otherwise returns
\schfalse.

\end{entry}

\begin{entry}{%
\proto{string-ci=?}{ \vri{char} \vrii{char} \vriii{char} \dotsfoo}{char library procedure}}

Returns \schtrue{} if, after case-folding, all the strings are the same
length and contain the same characters in the same positions, otherwise
returns \schfalse.  Specifically, these procedures behave as if 
{\cf string-foldcase} were applied to their arguments before comparing them.

\end{entry}


\begin{entry}{%
\proto{string-ni=?}{ \vri{char} \vrii{char} \vriii{char} \dotsfoo}{procedure}}

Returns \schtrue{} if, after an implementation-defined normalization,
all the strings are the same length and contain the same characters in
the same positions, otherwise returns \schfalse.  The intent is to
provide a means of comparing strings that are considered
equivalent in some situations but are represented by a different
sequence of characters.

Specifically, an implementation which
supports Unicode should use Unicode normalization NFC or
NFD as specified by Unicode TR\#15.  Implementations which only
support ASCII or some other character set which provides no ambiguous
representations of character sequences may define the normalization to
be the identity operation, in which case {\cf string-ni=?}\ is
equivalent to {\cf string=?}.

\end{entry}


\begin{entry}{%
\proto{string<?}{ \vri{string} \vrii{string} \vriii{string} \dotsfoo}{procedure}
\proto{string-ci<?}{ \vri{string} \vrii{string} \vriii{string} \dotsfoo}{char library procedure}
\proto{string-ni<?}{ \vri{string} \vrii{string} \vriii{string} \dotsfoo}{char library procedure}
\proto{string>?}{ \vri{string} \vrii{string} \vriii{string} \dotsfoo}{procedure}
\proto{string-ci>?}{ \vri{string} \vrii{string} \vriii{string} \dotsfoo}{char library procedure}
\proto{string-ni>?}{ \vri{string} \vrii{string} \vriii{string} \dotsfoo}{char library procedure}
\proto{string<=?}{ \vri{string} \vrii{string} \vriii{string} \dotsfoo}{procedure}
\proto{string-ci<=?}{ \vri{string} \vrii{string} \vriii{string} \dotsfoo}{char library procedure}
\proto{string-ni<=?}{ \vri{string} \vrii{string} \vriii{string} \dotsfoo}{char library procedure}
\proto{string>=?}{ \vri{string} \vrii{string} \vriii{string} \dotsfoo}{procedure}
\proto{string-ci>=?}{ \vri{string} \vrii{string} \vriii{string} \dotsfoo}{char library procedure}
\proto{string-ni>=?}{ \vri{string} \vrii{string} \vriii{string} \dotsfoo}{char library procedure}}

These procedures return \schtrue{} if their arguments are (respectively):
equal, monotonically increasing, monotonically decreasing,
monotonically nondecreasing, or monotonically nonincreasing.

These predicates are required to be transitive.

These procedures compare strings in an implementation-defined way.
One approach is to make them the lexicographic extensions to strings of
the corresponding orderings on characters.  In that case, {\cf string<?}\
would be the lexicographic ordering on strings induced by the ordering
{\cf char<?}\ on characters, and if the two strings differ in length but
are the same up to the length of the shorter string, the shorter string
would be considered to be lexicographically less than the longer string.
However, it is also permitted to use the natural ordering imposed by the
internal representation of strings, or a more complex locale-specific
ordering.

In all cases, a pair of strings must satisfy exactly one of
{\cf string<?}, {\cf string=?}, and {\cf string>?}, and must satisfy
{\cf string<=?} if and only if they do not satisfy {\cf string>?} and
{\cf string>=?} if and only if they do not satisfy {\cf string<?}.

The \hbox{``{\tt -ci}''} procedures behave as if they applied
{\cf string-foldcase} to their arguments before invoking the corresponding
procedures without  \hbox{``{\tt -ci}''}.

The \hbox{``{\tt -ni}''} procedures behave as if they applied the
implementation-defined normalization used by {\cf string-ni=?}
to their arguments before
invoking the corresponding procedures without \hbox{``{\tt -ni}''}.

\end{entry}

\begin{entry}{%
\proto{string-upcase}{ string}{char library procedure}
\proto{string-downcase}{ string}{char library procedure}
\proto{string-foldcase}{ string}{char library procedure}}


These procedures apply the Unicode full string uppercasing, lowercasing,
and case-folding algorithms to their arguments and return the result.
If the result is equal to the argument, a new string need not be allocated.
Note that language-sensitive mappings and foldings are not used.  
The result may differ in length from the argument.  What is more, a
few characters have case-mappings that depend on the surrounding context.
For example, Greek capital sigma normally lowercases to Greek small sigma,
but at the end of a word it downcases to Greek small final sigma instead.
See UAX \#29 (part of the Unicode Standard) for details.

\end{entry}


\begin{entry}{%
\proto{substring}{ string start end}{procedure}}

\domain{It is an error unless \var{String} is a string, and \var{start} and \var{end}
are exact integers satisfying
$$0 \leq \var{start} \leq \var{end} \leq \hbox{\tt(string-length \var{string})\rm.}$$}
{\cf substring} returns a newly allocated string formed from the characters of
\var{string} beginning with index \var{start} (inclusive) and ending with index
\var{end} (exclusive).
\end{entry}


\begin{entry}{%
\proto{string-append}{ \var{string} \dotsfoo}{procedure}}

Returns a newly allocated string whose characters form the concatenation of the
given strings.

\end{entry}


\begin{entry}{%
\proto{string->list}{ string}{procedure}
\proto{list->string}{ list}{procedure}}

{\cf string\coerce{}list} returns a newly allocated list of the
characters that make up the given string.  {\cf list\coerce{}string}
returns a newly allocated string formed from the elements in the list
\var{list}.  It is an error if any element is not a character.
{\cf string\coerce{}list}
and {\cf list\coerce{}string} are
inverses so far as {\cf equal?}\ is concerned.  

\end{entry}


\begin{entry}{%
\proto{string-copy}{ string}{procedure}}

Returns a newly allocated copy of the given \var{string}.

\end{entry}


\begin{entry}{%
\proto{string-fill!}{ string char}{procedure}
\proto{string-fill!}{ string char start end}{procedure}}

If two arguments are given,
stores \var{char} in every character location of the given \var{string} and returns an
unspecified value.  % <!>
If four arguments are given,
stores \var{char} in every character location of the given \var{string}
between \var{start} (inclusive) and \var{end} exclusive and returns an
unspecified value.

\end{entry}


\section{Vectors}
\label{vectorsection}

Vectors are heterogenous structures whose elements are indexed
by integers.  A vector typically occupies less space than a list
of the same length, and the average time needed to access a randomly
chosen element is typically less for the vector than for the list.

\vest The {\em length} of a vector is the number of elements that it
contains.  This number is a non-negative integer that is fixed when the
vector is created.  The {\em valid indexes}\index{valid indexes} of a
vector are the exact non-negative integers less than the length of the
vector.  The first element in a vector is indexed by zero, and the last
element is indexed by one less than the length of the vector.

Vectors are written using the notation {\tt\#(\var{obj} \dotsfoo)}.
For example, a vector of length 3 containing the number zero in element
0, the list {\cf(2 2 2 2)} in element 1, and the string {\cf "Anna"} in
element 2 can be written as following:

\begin{scheme}
\#(0 (2 2 2 2) "Anna")%
\end{scheme}

Note that this is the external representation of a vector, not an
expression evaluating to a vector.  
It is an error not to quote a vector constant:

\begin{scheme}
'\#(0 (2 2 2 2) "Anna")  \lev  \#(0 (2 2 2 2) "Anna")%
\end{scheme}


\begin{entry}{%
\proto{vector?}{ obj}{procedure}}
 
Returns \schtrue{} if \var{obj} is a vector; otherwise returns \schfalse.
\end{entry}


\begin{entry}{%
\proto{make-vector}{ k}{procedure}
\rproto{make-vector}{ k fill}{procedure}}

Returns a newly allocated vector of \var{k} elements.  If a second
argument is given, then each element is initialized to \var{fill}.
Otherwise the initial contents of each element is unspecified.

\end{entry}


\begin{entry}{%
\proto{vector}{ obj \dotsfoo}{procedure}}

Returns a newly allocated vector whose elements contain the given
arguments.  It is analogous to {\cf list}.

\begin{scheme}
(vector 'a 'b 'c)               \ev  \#(a b c)%
\end{scheme}
\end{entry}


\begin{entry}{%
\proto{vector-length}{ vector}{procedure}}

Returns the number of elements in \var{vector} as an exact integer.
\end{entry}


\begin{entry}{%
\proto{vector-ref}{ vector k}{procedure}}

\domain{It is an error if\vr{k} is not a valid index of \var{vector}.}
{\cf vector-ref} returns the contents of element \vr{k} of
\var{vector}.

\begin{scheme}
(vector-ref '\#(1 1 2 3 5 8 13 21)
            5)  \lev  8
(vector-ref '\#(1 1 2 3 5 8 13 21)
            (let ((i (round (* 2 (acos -1)))))
              (if (inexact? i)
                  (inexact->exact i)
                  i))) \lev 13%
\end{scheme}
\end{entry}


\begin{entry}{%
\proto{vector-set!}{ vector k obj}{procedure}}

\domain{It is an error if\vr{k} is not a valid index of \var{vector}.}
{\cf vector-set!} stores \var{obj} in element \vr{k} of \var{vector}.
The value returned by {\cf vector-set!}\ is unspecified.  % <!>

\begin{scheme}
(let ((vec (vector 0 '(2 2 2 2) "Anna")))
  (vector-set! vec 1 '("Sue" "Sue"))
  vec)      \lev  \#(0 ("Sue" "Sue") "Anna")

(vector-set! '\#(0 1 2) 1 "doe")  \lev  \scherror  ; constant vector%
\end{scheme}
\end{entry}


\begin{entry}{%
\proto{vector->list}{ vector}{procedure}
\proto{list->vector}{ list}{procedure}}

{\cf vector->list} returns a newly allocated list of the objects contained
in the elements of \var{vector}.  {\cf list->vector} returns a newly
created vector initialized to the elements of the list \var{list}.

\begin{scheme}
(vector->list '\#(dah dah didah))  \lev  (dah dah didah)
(list->vector '(dididit dah))   \lev  \#(dididit dah)%
\end{scheme}
\end{entry}

\begin{entry}{%
\proto{vector->string}{ string}{procedure}
\proto{string->vector}{ vector}{procedure}}

{\cf vector->string} returns a newly allocated string of the objects contained
in the elements of \var{vector}.
It is an error if any element is not a character allowed in strings.
{\cf string->vector} returns a newly
created vector initialized to the elements of the string \var{string}.


\begin{scheme}
(string->vector "ABC")  \ev   \#(\#\backwhack{}A \#\backwhack{}B \#\backwhack{}C)
(vector->string
  \#(\#\backwhack{}1 \#\backwhack{}2 \#\backwhack{}3) \ev "123"
\end{scheme}
\end{entry}

\begin{entry}{%
\proto{vector-copy}{ vector}{procedure}}

Returns a newly allocated copy of the given \var{vector}.
The elements of the new vector are the same (in the sense of
{\cf eqv?} as the elements of the old.

\end{entry}


\begin{entry}{%
\proto{vector-fill!}{ vector fill}{procedure}
\proto{vector-fill!}{ vector fill start end}{procedure}}

If two arguments are passed,
stores \var{fill} in every element of \var{vector}
and returns an unspecified value.
If four arguments are given,
stores \var{fill} in every element of the given \var{vector}
between \var{start} (inclusive) and \var{end} exclusive and returns an
unspecified value.

\end{entry}


\section{Bytevectors}
\label{bytevectorsection}

\defining{Bytevectors} are a disjoint type for representing blocks of binary data.
A \defining{byte} is an exact integer in the range $[0..255]$.
Conceptually, bytevectors can be thought of as homogeneous vectors of
bytes, but they typically occupy less space than a vector of the same bytes.

\vest The {\em length} of a bytevector is the number of elements that it
contains.  This number is a non-negative integer that is fixed when
the bytevector is created.  The {\em valid indexes}\index{valid indexes} of
a bytevector are the exact non-negative integers less than the length of the
bytevector, starting at index zero as with vectors.

\begin{entry}{%
\proto{bytevector?}{ obj}{procedure}}

Returns \schtrue{} if \var{obj} is a bytevector.
Otherwise, \schfalse{} is returned.
\end{entry}

\begin{entry}{%
\proto{make-bytevector}{ k}{procedure}
\rproto{make-bytevector}{ k byte}{procedure}}

{\cf make-bytevector} returns a newly allocated bytevector of
length \vr{k}.  If \var{byte} is given, then all elements of the bytevector
are initialized to \var{byte}, otherwise the contents of each
element are unspecified.
\end{entry}

\begin{entry}{%
\proto{bytevector-length}{ bytevector}{procedure}}

Returns the length of \var{bytevector} in bytes as an exact integer.
\end{entry}

\begin{entry}{%
\proto{bytevector-u8-ref}{ bytevector k}{procedure}}

Returns the \var{k}th byte of \var{bytevector}.
\end{entry}

\begin{entry}{%
\proto{bytevector-u8-set!}{ bytevector k byte}{procedure}}

Stores \var{byte} as the \var{k}th byte of \var{bytevector}.
The value returned by
{\cf bytevector-u8-set!} is unspecified.
\end{entry}

\begin{entry}{%
\proto{bytevector-copy}{ bytevector}{procedure}}

Returns a newly allocated bytevector containing the same bytes as
\var{bytevector}.
\end{entry}

\begin{entry}{%
\proto{bytevector-copy!}{ from to}{procedure}}

Copies the bytes of bytevector \var{from} to bytevector \var{to}.
It is an error if \var{to} is shorter than \var{from}.
The value returned by {\cf bytevector-copy!} is unspecified.
\end{entry}

\begin{entry}{%
\proto{bytevector-copy-partial}{ bytevector start end}{procedure}}

Returns a newly allocated bytevector containing the bytes in \var{bytevector}
between \var{start} (inclusive) and \var{end}
(exclusive).
\end{entry}

\begin{entry}{%
\proto{bytevector-copy-partial!}{ from start end to at}{procedure}}

Copies the bytes of bytevector \var{from} between \var{start} and \var{end}
to bytevector \var{to}, starting at \var{at}.  The order in which bytes are
copied is unspecified, except that if the source and destination overlap,
copying takes place as if the source is first copied into a temporary
bytevector and then into the destination.  This can be achieved without
allocating storage by making sure to copy in the correct direction in
such circumstances.

It is an error if the inequality
\texttt{({\cf >=} ({\cf -} ({\cf bytevector-length} \var{to}) \var{at}) ({\cf -} \var{end} \var{start}))}
is false.  The value returned by {\cf bytevector-copy-partial!} is unspecified.
\end{entry}

\begin{entry}{%
\proto{utf8->string}{ bytevector} {procedure}
\proto{string->utf8}{ string} {procedure}}

These procedures translate between strings and bytevectors
that encode those strings using the UTF-8 encoding.
The {\cf utf8\coerce{}string} procedure decodes a bytevector
and returns the corresponding string;
the {\cf string\coerce{}utf8} procedure
encodes a string and returns the corresponding bytevector.
It is an error to pass invalid byte sequences
or byte sequences representing characters which are forbidden in strings,
to {\cf utf8\coerce{}string}.

\begin{scheme}
(utf8->string \#u8(\#x41)) \ev "A"
(string->utf8 "$\lambda$") \ev \#u8(\#xCE \#xBB)
\end{scheme}

\end{entry}

\section{Control features}
\label{proceduresection}
 
This chapter describes various primitive procedures which control the
flow of program execution in special ways.
The {\cf procedure?}\ predicate is also described here.

\begin{entry}{%
\proto{procedure?}{ obj}{procedure}}

Returns \schtrue{} if \var{obj} is a procedure, otherwise returns \schfalse.

\begin{scheme}
(procedure? car)            \ev  \schtrue
(procedure? 'car)           \ev  \schfalse
(procedure? (lambda (x) (* x x)))   
                            \ev  \schtrue
(procedure? '(lambda (x) (* x x)))  
                            \ev  \schfalse
(call-with-current-continuation procedure?)
                            \ev  \schtrue%
\end{scheme}

\end{entry}


\begin{entry}{%
\proto{apply}{ proc \vari{arg} $\ldots$ args}{procedure}}

\domain{It is an error unless \var{proc} is a procedure and \var{args} is a list.}
{\cf apply} calls \var{proc} with the elements of the list
{\cf(append (list \vari{arg} \dotsfoo) \var{args})} as the actual
arguments.

\begin{scheme}
(apply + (list 3 4))              \ev  7

(define compose
  (lambda (f g)
    (lambda args
      (f (apply g args)))))

((compose sqrt *) 12 75)              \ev  30%
\end{scheme}
\end{entry}


\begin{entry}{%
\proto{map}{ proc \vari{list} \varii{list} \dotsfoo}{procedure}}

\domain{It is an error unless the \var{list}s are lists, and \var{proc} is a
procedure that accepts as many arguments as there are {\it list}s
and that returns a single value.  If more
than one \var{list} is given and not all lists have the same length,
{\cf map} terminates when the shortest list runs out.}
{\cf map} applies \var{proc} element-wise to the elements of the
\var{list}s and returns a list of the results, in order.  It is an
error for \var{proc} to mutate any of the lists.
The dynamic order in which \var{proc} is applied to the elements of the
\var{list}s is unspecified.  If multiple returns occur from {\cf map},
the values returned by earlier returns are not mutated.

\begin{scheme}
(map cadr '((a b) (d e) (g h)))   \lev  (b e h)

(map (lambda (n) (expt n n))
     '(1 2 3 4 5))                \lev  (1 4 27 256 3125)

(map + '(1 2 3) '(4 5 6 7))         \ev  (5 7 9)

(let ((count 0))
  (map (lambda (ignored)
         (set! count (+ count 1))
         count)
       '(a b)))                 \ev  (1 2) \var{or} (2 1)
\end{scheme}

\end{entry}

\begin{entry}{%
\proto{string-map}{ proc \vari{string} \varii{string} \dotsfoo}{procedure}}

\domain{It is an error unless the \var{string}s are strings, and \var{proc} is a
procedure taking as many arguments as there are {\it string}s
and returning a single value.  If more
than one \var{string} is given and not all strings have the same length,
{\cf string-map} terminates when the shortest string runs out.}
{\cf string-map} applies \var{proc} element-wise to the elements of the
\var{string}s and returns a string of the results, in order.
The dynamic order in which \var{proc} is applied to the elements of the
\var{string}s is unspecified.
If multiple returns occur from {\cf string-map},
the values returned by earlier returns are not mutated.

\begin{scheme}
(string-map char-foldcase "AbdEgH") \lev  "abdegh"

(string-map
 (lambda (c)
   (integer->char (+ 1 (char->integer c))))
 "HAL")                \lev  "IBM"

(string-map
 (lambda (c k)
   (if (eqv? k \sharpsign\backwhack{}u)
       (char-upcase c)
       (char-downcase c)))
 "studlycaps xxx"
 "ululululul")   \lev   "StUdLyCaPs"
\end{scheme}

\end{entry}

\begin{entry}{%
\proto{vector-map}{ proc \vari{vector} \varii{vector} \dotsfoo}{procedure}}

\domain{It is an error unless the \var{vector}s are vectors, and \var{proc} is a
procedure taking as many arguments as there are {\it vector}s
and returning a single value.  If more
than one \var{vector} is given and not all vectors have the same length,
{\cf vector-map} terminates when the shortest vector runs out.}
{\cf vector-map} applies \var{proc} element-wise to the elements of the
\var{vector}s and returns a vector of the results, in order.
The dynamic order in which \var{proc} is applied to the elements of the
\var{vector}s is unspecified.
If multiple returns occur from {\cf vector-map},
the values returned by earlier returns are not mutated.

\begin{scheme}
(vector-map cadr '\#((a b) (d e) (g h)))   \lev  \#(b e h)

(vector-map (lambda (n) (expt n n))
            '\#(1 2 3 4 5))                \lev  \#(1 4 27 256 3125)

(vector-map + '\#(1 2 3) '\#(4 5 6 7))       \lev  \#(5 7 9)

(let ((count 0))
  (vector-map
   (lambda (ignored)
     (set! count (+ count 1))
     count)
   '\#(a b)))                     \ev  \#(1 2) \var{or} \#(2 1)
\end{scheme}

\end{entry}


\begin{entry}{%
\proto{for-each}{ proc \vari{list} \varii{list} \dotsfoo}{procedure}}

The arguments to {\cf for-each} are like the arguments to {\cf map}, but
{\cf for-each} calls \var{proc} for its side effects rather than for its
values.  Unlike {\cf map}, {\cf for-each} is guaranteed to call \var{proc} on
the elements of the \var{list}s in order from the first element(s) to the
last, and the value returned by {\cf for-each} is unspecified.
It is an error for \var{proc} to mutate any of the lists.
If more than one \var{list} is given and not all lists have the same length,
{\cf for-each} terminates when the shortest list runs out.

\begin{scheme}
(let ((v (make-vector 5)))
  (for-each (lambda (i)
              (vector-set! v i (* i i)))
            '(0 1 2 3 4))
  v)                                \ev  \#(0 1 4 9 16)%
\end{scheme}

\end{entry}

\begin{entry}{%
\proto{string-for-each}{ proc \vari{string} \varii{string} \dotsfoo}{procedure}}

The arguments to {\cf string-for-each} are like the arguments to {\cf
string-map}, but {\cf string-for-each} calls \var{proc} for its side
effects rather than for its values.  Unlike {\cf string-map}, {\cf
string-for-each} is guaranteed to call \var{proc} on the elements of
the \var{list}s in order from the first element(s) to the last, and the
value returned by {\cf string-for-each} is unspecified.
If more than one \var{string} is given and not all strings have the same length,
{\cf string-for-each} terminates when the shortest string runs out.

\begin{scheme}
(let ((v '()))
  (string-for-each
   (lambda (c) (set! v (cons (char->integer c) v))
   "abcde")
  v)                                \ev  (101 100 99 98 97)%
\end{scheme}

\end{entry}

\begin{entry}{%
\proto{vector-for-each}{ proc \vari{vector} \varii{vector} \dotsfoo}{procedure}}

The arguments to {\cf vector-for-each} are like the arguments to {\cf
vector-map}, but {\cf vector-for-each} calls \var{proc} for its side
effects rather than for its values.  Unlike {\cf vector-map}, {\cf
vector-for-each} is guaranteed to call \var{proc} on the elements of
the \var{vector}s in order from the first element(s) to the last, and
the value returned by {\cf vector-for-each} is unspecified.
If more than one \var{vector} is given and not all vectors have the same length,
{\cf vector-for-each} terminates when the shortest vector runs out.

\begin{scheme}
(let ((v (make-list 5)))
  (vector-for-each
   (lambda (i) (list-set! v i (* i i)))
   '\#(0 1 2 3 4))
  v)                                \ev  (0 1 4 9 16)%
\end{scheme}

\end{entry}


\begin{entry}{%
\proto{call-with-current-continuation}{ proc}{procedure}
\proto{call/cc}{ proc}{procedure}}

\label{continuations} \domain{It is an error unless \var{proc} accepts one
argument.} The procedure {\cf call-with-current-continuation} (or its
equivalent abbreviation {\cf call/cc}) packages
the current continuation (see the rationale below) as an ``escape
procedure''\mainindex{escape procedure} and passes it as an argument to
\var{proc}.  The escape procedure is a Scheme procedure that, if it is
later called, will abandon whatever continuation is in effect at that later
time and will instead use the continuation that was in effect
when the escape procedure was created.  Calling the escape procedure
may cause the invocation of \var{before} and \var{after} thunks installed using
\ide{dynamic-wind}.

The escape procedure accepts the same number of arguments as the continuation to
the original call to \callcc.
Except for continuations created by the {\cf call-with-values}
procedure (including the initialization expressions of
{\cf let-values} and {\cf let*-values} syntax forms),
all continuations take exactly one value.  
\todo{consider removing unspecified effect}
The effect of passing no value or more than one value to continuations
that were not created by {\tt call-with-values} is unspecified.

However, the continuations of all non-final expressions within a sequence
of expressions, such as in {\cf lambda}, {\cf case-lambda}, {\cf begin},
{\cf let}, {\cf let*}, {\cf letrec}, {\cf letrec*}, {\cf let-values}, 
{\cf let*-values}, {\cf let-syntax}, {\cf letrec-syntax}, {\cf parameterize},
{\cf guard}, {\cf case}, {\cf cond}, {\cf when}, and {\cf unless} forms,
take an arbitrary number of values, because they discard the values passed
to them in any event.

\vest The escape procedure that is passed to \var{proc} has
unlimited extent just like any other procedure in Scheme.  It can be stored
in variables or data structures and can be called as many times as desired.

\vest The following examples show only the most common ways in which
{\cf call-with-current-continuation} is used.  If all real uses were as
simple as these examples, there would be no need for a procedure with
the power of {\cf call-with-current-continuation}.

\begin{scheme}
(call-with-current-continuation
  (lambda (exit)
    (for-each (lambda (x)
                (if (negative? x)
                    (exit x)))
              '(54 0 37 -3 245 19))
    \schtrue))                        \ev  -3

(define list-length
  (lambda (obj)
    (call-with-current-continuation
      (lambda (return)
        (letrec ((r
                  (lambda (obj)
                    (cond ((null? obj) 0)
                          ((pair? obj)
                           (+ (r (cdr obj)) 1))
                          (else (return \schfalse))))))
          (r obj))))))

(list-length '(1 2 3 4))            \ev  4

(list-length '(a b . c))            \ev  \schfalse%
\end{scheme}

\begin{rationale}

\vest A common use of {\cf call-with-current-continuation} is for
structured, non-local exits from loops or procedure bodies, but in fact
{\cf call-with-current-continuation} is useful for implementing a
wide variety of advanced control structures.

\vest Whenever a Scheme expression is evaluated there is a
\defining{continuation} wanting the result of the expression.  The continuation
represents an entire (default) future for the computation.  If the expression is
evaluated at top level, for example, then the continuation might take the
result, print it on the screen, prompt for the next input, evaluate it, and
so on forever.  Most of the time the continuation includes actions
specified by user code, as in a continuation that will take the result,
multiply it by the value stored in a local variable, add seven, and give
the answer to the top level continuation to be printed.  Normally these
ubiquitous continuations are hidden behind the scenes and programmers do not
think much about them.  On rare occasions, however, a programmer
needs to deal with continuations explicitly.
{\cf Call-with-current-continuation} allows Scheme programmers to do
that by creating a procedure that acts just like the current
continuation.

\vest Most programming languages incorporate one or more special-purpose
escape constructs with names like {\tt exit}, \hbox{{\cf return}}, or
even {\tt goto}.  In 1965, however, Peter Landin~\cite{Landin65}
invented a general purpose escape operator called the J-operator.  John
Reynolds~\cite{Reynolds72} described a simpler but equally powerful
construct in 1972.  The {\cf catch} special form described by Sussman
and Steele in the 1975 report on Scheme is exactly the same as
Reynolds's construct, though its name came from a less general construct
in MacLisp.  Several Scheme implementors noticed that the full power of the
\ide{catch} construct could be provided by a procedure instead of by a
special syntactic construct, and the name
{\cf call-with-current-continuation} was coined in 1982.  This name is
descriptive, but opinions differ on the merits of such a long name, and
some people use the name \ide{call/cc} instead.
\end{rationale}

%% \begin{note}
%% {\cf Call/cc} is capable of capturing continuations originating
%% outside of Scheme when using Scheme as a scripting
%% language embedded in some host language. It is not always practical
%% or even meaningful to restore these continuations.
%% \end{note}
\todo{Shinn: I'm not sure we want to say anything about this, and
  ``scripting'' is not relevant to the issue.}

\end{entry}

\begin{entry}{%
\proto{values}{ obj $\ldots$}{procedure}}

Delivers all of its arguments to its continuation.
{\tt Values} might be defined as follows:
\begin{scheme}
(define (values . things)
  (call-with-current-continuation 
    (lambda (cont) (apply cont things))))
\end{scheme}

\end{entry}

\begin{entry}{%
\proto{call-with-values}{ producer consumer}{procedure}}

Calls its \var{producer} argument with no values and
a continuation that, when passed some values, calls the
\var{consumer} procedure with those values as arguments.
The continuation for the call to \var{consumer} is the
continuation of the call to {\tt call-with-values}.

\begin{scheme}
(call-with-values (lambda () (values 4 5))
                  (lambda (a b) b))
                                                   \ev  5

(call-with-values * -)                             \ev  -1
\end{scheme}

\end{entry}

\begin{entry}{%
\proto{dynamic-wind}{ before thunk after}{procedure}}

Calls \var{thunk} without arguments, returning the result(s) of this call.
\var{Before} and \var{after} are called, also without arguments, as required
by the following rules.  Note that, in the absence of calls to continuations
captured using \ide{call-with-current-continuation}, the three arguments are
called once each, in order.  \var{Before} is called whenever execution
enters the dynamic extent of the call to \var{thunk} and \var{after} is called
whenever it exits that dynamic extent.  The dynamic extent of a procedure
call is the period between when the call is initiated and when it
returns.  \var{Before} and \var{after} are excluded from the dynamic extent.
In Scheme, because of {\cf call-with-current-continuation}, the
dynamic extent of a call is not always a single, connected time period.
It is defined as follows:
\begin{itemize}
\item The dynamic extent is entered when execution of the body of the
called procedure begins.

\item The dynamic extent is also entered when execution is not within
the dynamic extent and a continuation is invoked that was captured
(using {\cf call-with-current-continuation}) during the dynamic extent.

\item It is exited when the called procedure returns.

\item It is also exited when execution is within the dynamic extent and
a continuation is invoked that was captured while not within the
dynamic extent.
\end{itemize}

If a second call to {\cf dynamic-wind} occurs within the dynamic extent of the
call to \var{thunk} and then a continuation is invoked in such a way that the
\var{after}s from these two invocations of {\cf dynamic-wind} are both to be
called, then the \var{after} associated with the second (inner) call to
{\cf dynamic-wind} is called first.

If a second call to {\cf dynamic-wind} occurs within the dynamic extent of the
call to \var{thunk} and then a continuation is invoked in such a way that the
\var{before}s from these two invocations of {\cf dynamic-wind} are both to be
called, then the \var{before} associated with the first (outer) call to
{\cf dynamic-wind} is called first.

If invoking a continuation requires calling the \var{before} from one call
to {\cf dynamic-wind} and the \var{after} from another, then the \var{after}
is called first.

\todo{consider removing unspecified effect}
The effect of using a captured continuation to enter or exit the dynamic
extent of a call to \var{before} or \var{after} is unspecified.

\begin{scheme}
(let ((path '())
      (c \#f))
  (let ((add (lambda (s)
               (set! path (cons s path)))))
    (dynamic-wind
      (lambda () (add 'connect))
      (lambda ()
        (add (call-with-current-continuation
               (lambda (c0)
                 (set! c c0)
                 'talk1))))
      (lambda () (add 'disconnect)))
    (if (< (length path) 4)
        (c 'talk2)
        (reverse path))))
    \lev (connect talk1 disconnect
               connect talk2 disconnect)
\end{scheme}
\end{entry}

\section{Exceptions}
\label{exceptionsection}

This section describes Scheme's exception-handling and
exception-raising procedures.
For the concept of Scheme exceptions, see section \ref{errorsituations}.
See also \ref{guard} for the {\cf guard} syntax.

Exception handlers are one-argument procedures that determine the
action the program takes when an exceptional situation is signalled.
The system implicitly maintains a current exception handler.

\index{current exception handler}The program raises an exception by
invoking the current exception handler, passing it an object
encapsulating information about the exception.  Any procedure
accepting one argument may serve as an exception handler and any
object may be used to represent an exception.

\begin{entry}{%
\proto{with-exception-handler}{ \var{handler} \var{thunk}}{procedure}}

\domain{It is an error unless \var{handler} accepts one argument,
and \var{thunk} is a procedure that accepts zero arguments.}  The {\cf
with-exception-handler} procedure returns the results of invoking
\var{thunk}.  \var{Handler} is installed as the current
exception handler for the dynamic extent (as determined by {\cf
  dynamic-wind}) of the invocation of \var{thunk}.


\end{entry}

\begin{entry}{%
\proto{raise}{ \var{obj}}{procedure}}

Raises an exception by invoking the current exception
handler on \var{obj}. The handler is called with a continuation whose
dynamic extent is that of the call to {\cf raise}, except that
the current exception handler is the one that was in place when the
handler being called was installed.  If the handler returns, an
exception is raised in the same dynamic extent as the handler.
\end{entry}

\begin{entry}{%
\proto{raise-continuable}{ \var{obj}}{procedure}}

Raises an exception by invoking the current
exception handler on \var{obj}. The handler is called with a
continuation that is equivalent to the continuation of the call to
{\cf raise-continuable}, except that: (1) the current
exception handler is the one that was in place when the handler being
called was installed, and (2) if the handler being called returns,
then it will again become the current exception handler.  If the
handler returns, the values it returns become the values returned by
the call to {\cf raise-continuable}.
\end{entry}

\begin{scheme}
(with-exception-handler
  (lambda (con)
    (cond
      ((string? con)
       (display con))
      (else
       (display "a warning has been issued")))
    42)
  (lambda ()
    (+ (raise-continuable "should be a number")
       23)))
   {\it prints:} should be a number
   \ev 65
\end{scheme}

\begin{entry}{%
\proto{error}{ \var{message} \var{obj} $\ldots$}{procedure}}

\domain{\var{Message} should be a string.}
Raises an exception as if by calling
{\cf raise} on a newly allocated implementation-defined object which encapsulates
the information provided by \var{message},
as well as any \var{obj}s, known as the \defining{irritants}.
The procedure {\cf error-object?} must return \schtrue{} on such objects.

\begin{scheme}
(define (null-list? l)
  (cond ((pair? l) \#f)
        ((null? l) \#t)
        (else
          (error
            "null-list?: argument out of domain"
            l))))
\end{scheme}

\end{entry}

\begin{entry}{%
\proto{error-object?}{ obj}{procedure}}

Returns \schtrue{} if \var{obj} is an object created by {\cf error} 
or one of an implementation-defined set of objects, otherwise returns
\schfalse.  

\end{entry}

\begin{entry}{%
\proto{error-object-message}{ error-object}{procedure}}

Returns the message encapsulated by \var{error-object}.

\end{entry}

\begin{entry}{%
\proto{error-object-irritants}{ error-object}{procedure}}

Returns a list of the irritants encapsulated by \var{error-object}.

\end{entry}

\section{\tt{Eval}}

\begin{entry}{%
\proto{eval}{ expression environment-specifier}{eval library procedure}}

Evaluates \var{expression} in the specified environment and returns its value.
It is an error unless \var{expression} is a valid Scheme expression represented as data,
and \var{environment-specifier} is a value returned by one of the
four procedures described below.
Implementations may extend {\cf eval} to allow non-expression programs
(definitions) as the first argument and to allow other
values as environments, with the restriction that {\cf eval} is not
allowed to create new bindings in the environments returned by
{\cf null-environment} or {\cf scheme-report-environment}.

\begin{scheme}
(eval '(* 7 3) (scheme-report-environment 7))
                                                   \ev  21

(let ((f (eval '(lambda (f x) (f x x))
               (null-environment 7))))
  (f + 10))
                                                   \ev  20
\end{scheme}

\end{entry}

\begin{entry}{%
\proto{scheme-report-environment}{ version}{eval library procedure}
\proto{null-environment}{ version}{eval library procedure}}

It is an error if \var{Version} is not the exact integer {\cf \integerversion},
corresponding to this revision of the Scheme report (the
Revised$^\integerversion$ Report on Scheme).
{\cf scheme-report-environment} returns a specifier for an
environment that is empty except for all bindings 
defined either in the {\cf base} library or in the other libraries
of this report that the implementation supports.
{\cf null-environment} returns
a specifier for an environment that is empty except for the
bindings for all syntactic keywords
defined either in the {\cf base} library or in the other libraries
of this report that the implementation supports.

Other values of \var{version} can be used to specify environments
matching past revisions of this report, but their support is not
required.  
If \var{version}
is neither {\cf \integerversion} nor another value supported by
the implementation, an error is signalled.

\todo{consider removing unspecified effect}
The effect of assigning (through the use of {\cf eval}) a variable
bound in a {\cf scheme-report-environment}
(for example {\cf car}) is unspecified.  Thus the environments specified
by {\cf scheme-report-environment} may be immutable.

\end{entry}

\begin{entry}{%
\proto{environment}{ \vri{list} \dotsfoo}{eval library procedure}}

This procedure returns a specifier for the environment that results by
starting with an empty environment and then importing each \var{list},
considered as an import set, into it.  (See section~\ref{libraries} for
a description of import sets.)  The bindings of the environment
represented by the specifier are immutable.

\end{entry}

\begin{entry}{%
\proto{interaction-environment}{}{repl library procedure}}

This procedure returns a specifier for an environment that
contains an imple\-men\-ta\-tion-defined set of bindings, typically a superset of
those listed in the report.  The intent is that this procedure
will return the environment in which the implementation would evaluate
expressions dynamically typed by the user.

\end{entry}

\section{Input and output}

\subsection{Ports}
\label{portsection}

Ports represent input and output devices.  To Scheme, an input port is
a Scheme object that can deliver data upon command, while an output
port is a Scheme object that can accept data.\mainindex{port}
Whether the input and output port types are disjoint is
implementation-dependent.

Different {\em port types} operate on different data.  Scheme
imple\-men\-ta\-tions are required to support {\em textual ports}
and {\em binary ports}, but may also provide other port types.

A textual port supports reading or writing of individual characters
from or to a backing store containing characters
using {\cf read-char} and {\cf write-char} below, as well as operations
defined in terms of characters such as {\cf read} and {\cf write}.

A binary port supports reading or writing of individual bytes from
or to a backing store containing bytes using {\cf read-u8} and {\cf
write-u8} below, as well as operations defined in terms of bytes.
Whether the textual and binary port types are disjoint is
implementation-dependent.

Ports can be used to access files, devices, and similar things on the host
system on which the Scheme program is running.

\begin{entry}{%
\proto{call-with-input-file}{ string proc}{file library procedure}
\proto{call-with-output-file}{ string proc}{file library procedure}}

It is an error unless \var{proc} accepts one argument.
\todo{consider removing unspecified effect}
For {\cf call-with-input-file},
the file named by \var{string} should already exist; for
{\cf call-with-output-file},
the effect is unspecified if the file
already exists. These procedures call \var{proc} with one argument: the
textual port obtained by opening the named file for input or output
as if by {\cf open-input-file} or {\cf open-output-file}.  If the
file cannot be opened, an error is signalled.  If \var{proc} returns,
then the port is closed automatically and the values yielded by the
\var{proc} are returned.  If \var{proc} does not return, then 
the port must not be closed automatically unless it is possible to
prove that the port will never again be used for a read or write
operation.

\begin{rationale}
Because Scheme's escape procedures have unlimited extent, it  is
possible to escape from the current continuation but later to escape back in. 
If implementations were permitted to close the port on any escape from the
current continuation, then it would be impossible to write portable code using
both {\cf call-with-current-continuation} and {\cf call-with-input-file} or
{\cf call-with-output-file}.
\end{rationale} 
\end{entry}

\begin{entry}{%
\proto{call-with-port}{ port proc}{procedure}}

It is an error unless \var{proc} accepts one argument.  The {\cf call-with-port}
procedure calls \var{proc} with \var{port} as an argument.  If
\var{proc} returns, \var{port} is closed automatically and the values
returned by \var{proc} are returned.

\end{entry}

\begin{entry}{%
\proto{input-port?}{ obj}{procedure}
\proto{output-port?}{ obj}{procedure}
\proto{textual-port?}{ obj}{procedure}
\proto{binary-port?}{ obj}{procedure}
\proto{port?}{ obj}{procedure}}

Returns \schtrue{} if \var{obj} is an input port, output port, 
textual port, binary port, or any
kind of port, respectively.  Otherwise returns \schfalse.

\end{entry}


\begin{entry}{%
\proto{port-open?}{ port}{procedure}}

Returns \schtrue{} if \var{port} is still open and capable of
performing input or output, and \schfalse{} otherwise.


\end{entry}


\begin{entry}{%
\proto{current-input-port}{}{procedure}
\proto{current-output-port}{}{procedure}
\proto{current-error-port}{}{procedure}}

Returns the current default input port, output port, or error port (an
output port), respectively.  These procedures are parameter objects, which can be
overridden with {\cf parameterize} (see
section~\ref{make-parameter}). The initial bindings for these
are system-defined textual ports.

\end{entry}


\begin{entry}{%
\proto{with-input-from-file}{ string thunk}{file library procedure}
\proto{with-output-to-file}{ string thunk}{file library procedure}}

It is an error unless \var{thunk} is a procedure of no arguments.
\todo{consider removing unspecified effect}
For {\cf with-input-from-file},
it is an error if the file named by \var{string} does not already exist; for
{\cf with-output-to-file},
the effect is unspecified if the file
already exists.
The file is opened for input or output
as if by {\cf open-input-file} or {\cf open-output-file}, 
and the new port is made the default value returned by
{\cf current-input-port} or {\cf current-output-port}
(and is used by {\tt (read)}, {\tt (write \var{obj})}, and so forth).
The \var{thunk} is then called with no arguments.  When the \var{thunk} returns,
the port is closed and the previous default is restored.
{\cf with-input-from-file} and {\cf with-output-to-file} return the
values yielded by \var{thunk}.
If an escape procedure
is used to escape from the continuation of these procedures, their
behavior is implementation-dependent.


\end{entry}


\begin{entry}{%
\proto{open-input-file}{ string}{file library procedure}
\proto{open-binary-input-file}{ string}{file library procedure}}
 
Takes a \var{string} for an existing file and returns a textual
input port or binary input port capable of delivering data from the
file.  If the file cannot be opened, an error is signalled.

\end{entry}


\begin{entry}{%
\proto{open-output-file}{ string}{file library procedure}
\proto{open-binary-output-file}{ string}{file library procedure}}

Takes a \var{string} naming an output file to be created and returns a
textual output port or binary output port capable of writing
data to a new file by that name.  If the file cannot be opened,
an error is signalled.
\todo{consider removing unspecified effect}
If a file with the given name already exists,
the effect is unspecified.

\end{entry}


\begin{entry}{%
\proto{close-port}{ port}{procedure}
\proto{close-input-port}{ port}{procedure}
\proto{close-output-port}{ port}{procedure}}

Closes the resource associated with \var{port}, rendering the \var{port}
incapable of delivering or accepting data.  
It is an error
to apply the last two procedures to a port which is not an input
or output port, respectively.
Scheme implementations may provide ports which are simultaneously
input and output ports, such as sockets; the {\cf close-input-port}
and {\cf close-output-port} procedures can then be used to close the
input and output sides of the port independently.

These routines have no effect if the file has already been closed.
The value returned is unspecified.


\end{entry}

\begin{entry}{%
\proto{open-input-string}{ string}{procedure}}

Takes a string and returns a textual input port that delivers
characters from the string.

\end{entry}

\begin{entry}{%
\proto{open-output-string}{}{procedure}}

Returns a textual output port that will accumulate characters for
retrieval by {\cf get-output-string}.

\end{entry}

\begin{entry}{%
\proto{get-output-string}{ port}{procedure}}

Given an output port created by {\cf open-output-string}, returns a
string consisting of the characters that have been output to the port
so far in the order they were output.
\end{entry}



\begin{entry}{%
\proto{open-input-bytevector}{ bytevector}{procedure}}

Takes a bytevector and returns a binary input port that delivers
bytes from the bytevector.

\end{entry}

\begin{entry}{%
\proto{open-output-bytevector}{}{procedure}}

Returns a binary output port that will accumulate bytes for
retrieval by {\cf get-output-bytevector}.

\end{entry}

\begin{entry}{%
\proto{get-output-bytevector}{ port}{procedure}}

Given an output port created by {\cf open-output-bytevector}, returns a
bytevector consisting of the bytes that have been output to the port
so far in the order they were output.
\end{entry}


\subsection{Input}
\label{inputsection}

\noindent \hbox{ }  %???
\vspace{-5ex}
\todo{The input routines have some things in common, maybe explain here.}

\begin{entry}{%
\proto{read}{}{read library procedure}
\rproto{read}{ port}{read library procedure}}

{\cf read} converts external representations of Scheme objects into the
objects themselves.  That is, it is a parser for the nonterminal
\meta{datum} (see sections~\ref{datum} and
\ref{listsection}).  {\cf read} returns the next
object parsable from the given textual input \var{port}, updating
\var{port} to point to
the first character past the end of the external representation of the object.

\vest If an end of file is encountered in the input before any
characters are found that can begin an object, then an end of file
object is returned.  The port remains open, and further attempts
to read will also return an end of file object.  If an end of file is
encountered after the beginning of an object's external representation,
but the external representation is incomplete and therefore not parsable,
an error is signalled.

\var{Port} may be omitted, in which case it defaults to the
value returned by {\cf current-input-port}.  It is an error to read from
a closed port.
\end{entry}

\begin{entry}{%
\proto{read-char}{}{procedure}
\rproto{read-char}{ port}{procedure}}

Returns the next character available from the textual input \var{port},
updating
the \var{port} to point to the following character.  If no more characters
are available, an end of file object is returned.  \var{Port} may be
omitted, in which case it defaults to the value returned by {\cf current-input-port}.

\end{entry}


\begin{entry}{%
\proto{peek-char}{}{procedure}
\rproto{peek-char}{ port}{procedure}}

Returns the next character available from the textual input \var{port},
{\em without} updating
the \var{port} to point to the following character.  If no more characters
are available, an end of file object is returned.  \var{Port} may be
omitted, in which case it defaults to the value returned by {\cf current-input-port}.

\begin{note}
The value returned by a call to {\cf peek-char} is the same as the
value that would have been returned by a call to {\cf read-char} with the
same \var{port}.  The only difference is that the very next call to
{\cf read-char} or {\cf peek-char} on that \var{port} will return the
value returned by the preceding call to {\cf peek-char}.  In particular, a call
to {\cf peek-char} on an interactive port will hang waiting for input
whenever a call to {\cf read-char} would have hung.
\end{note}

\end{entry}

\begin{entry}{%
\proto{read-line}{}{procedure}
\rproto{read-line}{ port}{procedure}}

Returns the next line of text available from the textual input
\var{port}, updating the \var{port} to point to the following character.
If an end of line is read, a string containing all of the text up to
(but not including) the end of line is returned, and the port is updated
to point just past the end of line. If an end of file is encountered
before any end of line is read, but some characters have been
read, a string containing those characters is returned. If an end of
file is encountered before any characters are read, an end-of-file
object is returned.  For the purpose of this procedure, an end of line
consists of either a linefeed character, a carriage return character, or a
sequence of a carriage return character followed by a linefeed character.
\var{Port} may be omitted, in which case it defaults to the value returned
by {\cf current-input-port}.

\end{entry}


\begin{entry}{%
\proto{eof-object?}{ obj}{procedure}}

Returns \schtrue{} if \var{obj} is an end of file object, otherwise returns
\schfalse.  The precise set of end of file objects will vary among
implementations, but in any case no end of file object will ever be an object
that can be read in using {\cf read}.

\end{entry}


\begin{entry}{%
\proto{char-ready?}{}{procedure}
\rproto{char-ready?}{ port}{procedure}}

Returns \schtrue{} if a textual is ready on the character input \var{port} and
returns \schfalse{} otherwise.  If {\cf char-ready} returns \schtrue{} then
the next {\cf read-char} operation on the given \var{port} is guaranteed
not to hang.  If the \var{port} is at end of file then {\cf char-ready?}\
returns \schtrue.  \var{Port} may be omitted, in which case it defaults to
the value returned by {\cf current-input-port}.

\begin{rationale}
{\cf char-ready?}\ exists to make it possible for a program to
accept characters from interactive ports without getting stuck waiting for
input.  Any input editors associated with such ports must ensure that
characters whose existence has been asserted by {\cf char-ready?}\ cannot
be rubbed out.  If {\cf char-ready?}\ were to return \schfalse{} at end of
file, a port at end of file would be indistinguishable from an interactive
port that has no ready characters.
\end{rationale}
\end{entry}

\begin{entry}{%
\proto{read-u8}{}{procedure}
\rproto{read-u8}{ port}{procedure}}

Returns the next byte available from the binary input \var{port},
updating the \var{port} to point to the following byte.  
If no more bytes are
available, an end of file object is returned.  \var{Port} may be
omitted, in which case it defaults to the value returned by {\cf
  current-input-port}.

\end{entry}

\begin{entry}{%
\proto{peek-u8}{}{procedure}
\rproto{peek-u8}{ port}{procedure}}

Returns the next byte available from the binary input \var{port},
{\em without} updating the \var{port} to point to the following
byte.  If no more bytes are available, an end of file object is
returned.  \var{Port} may be omitted, in which case it defaults to the
value returned by {\cf current-input-port}.

\end{entry}

\begin{entry}{%
\proto{u8-ready?}{}{procedure}
\rproto{u8-ready?}{ port}{procedure}}

Returns \schtrue{} if a byte is ready on the binary input \var{port}
and returns \schfalse{} otherwise.  If {\cf u8-ready?} returns
\schtrue{} then the next {\cf read-u8} operation on the given
\var{port} is guaranteed not to hang.  If the \var{port} is at end of
file then {\cf u8-ready?}\ returns \schtrue.  \var{Port} may be
omitted, in which case it defaults to the value returned by {\cf
  current-input-port}.

\end{entry}

\begin{entry}{%
\proto{read-bytevector}{ length}{procedure}
\rproto{read-bytevector}{ length port}{procedure}}

Reads the next \var{length} bytes, or as many as are available before the end of file,
from the binary
input \var{port} into a newly allocated bytevector in left-to-right order
and returns the bytevector.
If no bytes are available,
an end of file object is returned.
\var{Port} may be
omitted, in which case it defaults to the value returned by {\cf
  current-input-port}.

\end{entry}

\begin{entry}{%
\proto{read-bytevector!}{ bytevector start end}{procedure}
\rproto{read-bytevector!}{ bytevector start end port}{procedure}}

Places the next $end-start$ bytes, or as many as are available
before the end of file,
from the binary
input \var{port} into \var{bytevector} in left-to-right order
beginning at the \var{start} position.
Returns the number of bytes read.
If no bytes are available,
an end of file object is returned.
\var{Port} may be
omitted, in which case it defaults to the value returned by {\cf
  current-input-port}.

\end{entry}


\subsection{Output}
\label{outputsection}

% We've got to put something here to fix the indentation!!
\noindent \hbox{}
\vspace{-5ex}

\begin{entry}{%
\proto{write}{ obj}{write library procedure}
\rproto{write}{ obj port}{write library procedure}}

Writes a written representation of \var{obj} to the given textual output
\var{port}.  Strings
that appear in the written representation are enclosed in doublequotes, and
within those strings backslash and doublequote characters are
escaped by backslashes.  Symbols that contain non-ASCII characters
are escaped either with inline hex escapes or with vertical bars.
Character objects are written using the {\cf \#\backwhack} notation.
Shared list structure is represented using reader labels.
{\cf write} returns an unspecified value.
\var{Port} may be omitted, in which case it defaults to the value
returned by {\cf current-output-port}.

\end{entry}

\begin{entry}{%
\proto{write-simple}{ obj}{write library procedure}
\rproto{write-simple}{ obj port}{write library procedure}}

{\cf write-simple} is the same as {\cf write}, except that shared structure is
not represented using reader labels.  This may cause {\cf write-simple} not to
terminate if \var{obj} contains circular structure.

\end{entry}


\begin{entry}{%
\proto{display}{ obj}{write library procedure}
\rproto{display}{ obj port}{write library procedure}}

Writes a representation of \var{obj} to the given textual output \var{port}.
Strings that appear in the written representation are not enclosed in
doublequotes, and no characters are escaped within those strings.  
Symbols are not escaped.  Character
objects appear in the representation as if written by {\cf write-char}
instead of by {\cf write}.  {\cf display} returns an unspecified value.
\var{Port} may be omitted, in which case it defaults to the
value returned by {\cf current-output-port}.

\begin{rationale}
{\cf write} is intended
for producing mach\-ine-readable output and {\cf display} for producing
human-readable output.  
\end{rationale}
\end{entry}


\begin{entry}{%
\proto{newline}{}{procedure}
\rproto{newline}{ port}{procedure}}

Writes an end of line to textual output \var{port}.  Exactly how this
is done differs
from one operating system to another.  Returns an unspecified value.
\var{Port} may be omitted, in which case it defaults to the
value returned by {\cf current-output-port}.

\end{entry}


\begin{entry}{%
\proto{write-char}{ char}{procedure}
\rproto{write-char}{ char port}{procedure}}

Writes the character \var{char} (not an external representation of the
character) to the given textual output \var{port} and returns an unspecified
value.  
\var{Port} may be omitted, in which case it defaults to the value
returned by {\cf current-output-port}.

\end{entry}

\begin{entry}{%
\proto{write-u8}{ byte}{procedure}
\rproto{write-u8}{ byte port}{procedure}}

Writes the \var{byte} to
the given binary output \var{port} and returns an unspecified value.
\var{Port} may be omitted, in which case it defaults to
the value returned by {\cf current-output-port}.

\end{entry}

\begin{entry}{%
\proto{write-bytevector}{ bytevector}{procedure}
\rproto{write-bytevector}{ bytevector port}{procedure}}

Writes the bytes of \var{bytevector} in left-to-right order to the
binary output \var{port}.
\var{Port} may be
omitted, in which case it defaults to the value returned by {\cf
  current-output-port}.

\end{entry}

\begin{entry}{%
\proto{write-partial-bytevector}{ bytevector start end}{procedure}
\rproto{write-partial-bytevector}{ bytevector start end port}{procedure}}

Writes the bytes of \var{bytevector}
from \var{start} (inclusive) to \var{end} (exclusive)
in left-to-right order to the
binary output \var{port}.
\var{Port} may be
omitted, in which case it defaults to the value returned by {\cf
  current-output-port}.

\end{entry}

\begin{entry}{%
\proto{flush-output-port}{}{procedure}
\rproto{flush-output-port}{ port}{procedure}}

Flushes any buffered output from the buffer of output-port to the
underlying file or device and returns an unspecified value.
\var{Port} may be omitted, in which case it defaults to
the value returned by {\cf current-output-port}.

\end{entry}


\subsection{System interface}

Questions of system interface generally fall outside of the domain of this
report.  However, the following operations are important enough to
deserve description here.


\begin{entry}{%
\proto{load}{ filename}{load library procedure}
\rproto{load}{ filename environment-specifier}{load library procedure}}

An implementation-dependent operation is used to transform
\var{filename} into the name of an existing file
containing Scheme source code.  The {\cf load} procedure reads
expressions and definitions from the file and evaluates them
sequentially in the environment specified by \var{environment-specifier}.
If \var{environment-specifier} is omitted, {\cf (interaction-environment)}
is assumed.

It is unspecified whether the results of the expressions
are printed.  The {\cf load} procedure does not affect the values
returned by {\cf current-input-port} and {\cf current-output-port}.
It returns an unspecified value.


\begin{rationale}
For portability, {\cf load} must operate on source files.
Its operation on other kinds of files necessarily varies among
implementations.
\end{rationale}
\end{entry}

\begin{entry}{%
\proto{file-exists?}{ filename}{file library procedure}}

It is an error unless \var{filename} is a string. The {\cf file-exists?} procedure returns
\schtrue{} if the named file exists at the time the procedure is called,
\schfalse{} otherwise.

\end{entry}

\begin{entry}{%
\proto{delete-file}{ filename}{file library procedure}}

It is an error unless \var{filename} is a string. The {\cf delete-file} procedure deletes the
named file if it exists and can be deleted, and returns an unspecified
value.  If the file does not exist or cannot be deleted, an error
is signalled.

\end{entry}

\begin{entry}{%
\proto{command-line}{}{process-context library procedure}}

Returns the command line passed to the process as a list of
strings.  The first string corresponds to the command name, and is
implementation-dependent.  It is an error to mutate any of these strings.
\end{entry}

\begin{entry}{%
\proto{exit}{}{process-context library procedure}
\rproto{exit}{ obj}{process-context library procedure}}

Exits the running program and communicates an exit value to the
operating system.  If no argument is supplied, the {\cf exit}
procedure should communicate to the operating system that the program
exited normally.  If an argument is supplied, the exit procedure
should translate the argument into an appropriate exit value for the
operating system. If obj is \schfalse{}, the exit is assumed to be
abnormal.

\end{entry}

\todo{Shinn: Do we need any description of what an environment variable is?}

\begin{entry}{%
\proto{get-environment-variable}{ name}{process-context library procedure}}

Most operating systems provide each running process with an
\defining{environment} consisting of \defining{environment variables}.
Both the name and value of an environment variable are strings.
{\cf get-environment-variable} returns the value 
of the environment variable \var{name},
or \schfalse{} if the named
environment variable is not found.  {\cf get-environment-variable} may
use locale-setting information to encode the name and decode the value
of the environment variable.  It is an error if
{\cf get-environment-variable} can't decode the value.
It is also an error to mutate the resulting string.

\begin{scheme}
(get-environment-variable "PATH") \lev "/usr/local/bin:/usr/bin:/bin"
\end{scheme}

\end{entry}

\begin{entry}{%
\proto{get-environment-variables}{}{process-context library procedure}}

Returns the names and values of all the environment variables as an
alist, where the car of each entry is the name of an environment
variable and the cdr is its value, both as strings.  The order of the list is unspecified.
It is an error to mutate any of these strings.

\begin{scheme}
(get-environment-variables) \lev (("USER" . "root") ("HOME" . "/"))
\end{scheme}

\end{entry}

\begin{entry}{%
\proto{current-second}{}{time library procedure}}

Returns an inexact number representing time on the International Atomic
Time (TAI) scale.  The value 0.0 represents ten seconds after midnight
on January 1, 1970 TAI (equivalent to midnight Universal Time)
and the value 1.0 represents one TAI
second later.  Neither high-accuracy nor high-precision values are required; in particular,
returning Coordinated Universal Time plus a suitable constant may be
the best an implementation can do.
\end{entry}

\begin{entry}{%
\proto{current-jiffy}{}{time library procedure}}

Returns the number of \defining{jiffies} that have elapsed since an arbitrary,
implementation-defined epoch. A jiffy is an implementation-defined
fraction of a second which is defined by the return value of the
{\cf jiffies-per-second} procedure. The starting epoch is guaranteed to be
constant during a run of the program, but may vary between different runs.
\end{entry}

\begin{entry}{%
\proto{jiffies-per-second}{}{time library procedure}}

Returns an exact integer representing the number of jiffies per SI
second. This value is an implementation-specified constant.

\begin{scheme}
(define (time-length)
  (let ((list (make-list 100000))
        (start (current-jiffy)))
    (length list)
    (/ (- (current-jiffy) start)
       (jiffies-per-second))))
\end{scheme}
\end{entry}

