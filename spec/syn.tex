\chapter{Formal syntax and semantics}
\label{formalchapter}

This chapter provides formal descriptions of what has already been
described informally in previous chapters of this report.



\section{Formal syntax}
\label{BNF}

This section provides a formal syntax for Scheme written in an extended
BNF.

All spaces in the grammar are for legibility.  Case is not significant
except in the definitions of \meta{letter} and \meta{character name}; for example, {\cf \#x1A}
and {\cf \#X1a} are equivalent, but {\cf foo} and {\cf Foo}
and {\cf \#\backwhack{}space} and {\cf \#\backwhack{}Space} are distinct.
\meta{empty} stands for the empty string.

The following extensions to BNF are used to make the description more
concise:  \arbno{\meta{thing}} means zero or more occurrences of
\meta{thing}; and \atleastone{\meta{thing}} means at least one
\meta{thing}.


\subsection{Lexical structure}

This section describes how individual tokens\index{token} (identifiers,
numbers, etc.) are formed from sequences of characters.  The following
sections describe how expressions and programs are formed from sequences
of tokens.

\meta{Intertoken space} may occur on either side of any token, but not
within a token.

\vest Identifiers that do not begin with a vertical line are
terminated by a \meta{delimiter} or by the end of the input.
So are dot, numbers, characters, and booleans.
Identifiers that begin with a vertical line are terminated by another vertical line.

The following four characters from the ASCII repertoire
are reserved for future extensions to the
language: {\tt \verb"[" \verb"]" \verb"{" \verb"}"}

In addition to the identifier characters of the ASCII repertoire specified
below, Scheme implementations may permit any additional repertoire of
Unicode characters to be employed in identifiers,
provided that each such character has a Unicode general category of Lu,
Ll, Lt, Lm, Lo, Mn, Mc, Me, Nd, Nl, No, Pd, Pc, Po, Sc, Sm, Sk, So,
or Co, or is U+200C or U+200D (the zero-width non-joiner and joiner,
respectively, which are needed for correct spelling in Persian, Hindi,
and other languages).
However, it is an error for the first character to have a general category
of Nd, Mc, or Me.  It is also an error to use a non-Unicode character
in symbols or identifiers.

All Scheme implementations must permit the escape sequence
{\tt \backwhack{}x<hexdigits>;}
to appear in Scheme identifiers that are enclosed in vertical lines. If the character
with the given Unicode scalar value is supported by the implementation,
identifiers containing such a sequence are equivalent to identifiers
containing the corresponding character. 

\begin{grammar}%
\meta{token} \: \meta{identifier} \| \meta{boolean} \| \meta{number}\index{identifier}
\>  \| \meta{character} \| \meta{string}
\>  \| ( \| ) \| \sharpsign( \| \sharpsign u8( \| \singlequote{} \| \backquote{} \| , \| ,@ \| {\bf.}
\meta{delimiter} \: \meta{whitespace} \| \meta{vertical line}
\> \| ( \| ) \| " \| ;
\meta{intraline whitespace} \: \meta{space or tab}
\meta{whitespace} \: \meta{intraline whitespace} \| \meta{line ending}
\meta{vertical line} \: |
\meta{line ending} \: \meta{newline} \| \meta{return} \meta{newline}
\> \| \meta{return}
\meta{comment} \: ; \= $\langle$\rm all subsequent characters up to a
		    \>\ \rm line ending$\rangle$\index{comment}
\> \| \meta{nested comment}
\> \| \#; \meta{intertoken space} \meta{datum}
\meta{nested comment} \: \#| \= \meta{comment text}
\> \arbno{\meta{comment cont}} |\#
\meta{comment text} \: \= $\langle$\rm character sequence not containing
\>\ \rm {\tt \#|} or {\tt |\#}$\rangle$
\meta{comment cont} \: \meta{nested comment} \meta{comment text}
\meta{directive} \: \#!fold-case \| \#!no-fold-case%
\end{grammar}

Note that each \meta{directive} must be followed by a \meta{delimiter}
or the end of file.

\begin{grammar}%
\meta{atmosphere} \: \meta{whitespace} \| \meta{comment} \| \meta{directive}
\meta{intertoken space} \: \arbno{\meta{atmosphere}}%
\end{grammar}

\label{extendedalphas}
\label{identifiersyntax}

% This is a kludge, but \multicolumn doesn't work in tabbing environments.
\setbox0\hbox{\cf\meta{identifier} \goesto{} $\langle$}

Note that {\cf +i}, {\cf -i} and \meta{infnan} below are exceptions to the
\meta{peculiar identifier} rule; they are parsed as numbers, not
identifiers.

\begin{grammar}%
\meta{identifier} \: \meta{initial} \arbno{\meta{subsequent}}
 \>  \| \meta{vertical line} \arbno{\meta{symbol element}} \meta{vertical line}
 \>  \| \meta{peculiar identifier}
\meta{initial} \: \meta{letter} \| \meta{special initial}
\meta{letter} \: a \| b \| c \| ... \| z
\> \| A \| B \| C \| ... \| Z
\meta{special initial} \: ! \| \$ \| \% \| \verb"&" \| * \| / \| : \| < \| =
 \>  \| > \| ? \| \verb"^" \| \verb"_" \| \verb"~"
\meta{subsequent} \: \meta{initial} \| \meta{digit}
 \>  \| \meta{special subsequent}
\meta{digit} \: 0 \| 1 \| 2 \| 3 \| 4 \| 5 \| 6 \| 7 \| 8 \| 9
\meta{hex digit} \: \meta{digit} \| a \| b \| c \| d \| e \| f
\meta{explicit sign} \: + \| -
\meta{special subsequent} \: \meta{explicit sign} \| . \| @
\meta{inline hex escape} \: \backwhack{}x\meta{hex scalar value};
\meta{hex scalar value} \: \atleastone{\meta{hex digit}}
\meta{mnemonic escape} \: \backwhack{}a \| \backwhack{}b \| \backwhack{}t \| \backwhack{}n \| \backwhack{}r
\meta{peculiar identifier} \: \meta{explicit sign}
 \> \| \meta{explicit sign} \meta{sign subsequent} \arbno{\meta{subsequent}}
 \> \| \meta{explicit sign} . \meta{dot subsequent} \arbno{\meta{subsequent}}
 \> \| . \meta{dot subsequent} \arbno{\meta{subsequent}}
 %\| 1+ \| -1+
\meta{dot subsequent} \: \meta{sign subsequent} \| .
\meta{sign subsequent} \: \meta{initial} \| \meta{explicit sign} \| @
\meta{symbol element} \:
 \> \meta{any character other than \meta{vertical line} or \backwhack}
 \> \| \meta{inline hex escape} \| \meta{mnemonic escape} \| \backwhack{}|

\meta{boolean} \: \schtrue{} \| \schfalse{} \| \sharptrue{} \| \sharpfalse{}
\label{charactersyntax}
\meta{character} \: \#\backwhack{} \meta{any character}
 \>  \| \#\backwhack{} \meta{character name}
 \>  \| \#\backwhack{}x\meta{hex scalar value}
\meta{character name} \: alarm \| backspace \| delete 
\> \| escape \| newline \| null \| return \| space \| tab
\todo{Explain what happens in the ambiguous case.}
\meta{string} \: " \arbno{\meta{string element}} "
\meta{string element} \: \meta{any character other than \doublequote{} or \backwhack}
 \> \| \meta{mnemonic escape} \| \backwhack\doublequote{} \| \backwhack\backwhack 
 \>  \| \backwhack{}\arbno{\meta{intraline whitespace}}\meta{line ending}
 \>  \> \arbno{\meta{intraline whitespace}}
 \>  \| \meta{inline hex escape}
\meta{bytevector} \: \#u8(\arbno{\meta{byte}})
\meta{byte} \: \meta{any exact integer between 0 and 255}%
\end{grammar}


\label{numbersyntax}

\begin{grammar}%
\meta{number} \: \meta{num $2$} \| \meta{num $8$}
   \>  \| \meta{num $10$} \| \meta{num $16$}
\end{grammar}

The following rules for \meta{num $R$}, \meta{complex $R$}, \meta{real
$R$}, \meta{ureal $R$}, \meta{uinteger $R$}, and \meta{prefix $R$}
are implicitly replicated for \hbox{$R = 2, 8, 10,$}
and $16$.  There are no rules for \meta{decimal $2$}, \meta{decimal
$8$}, and \meta{decimal $16$}, which means that numbers containing
decimal points or exponents are always in decimal radix.
Although not shown below, all alphabetic characters used in the grammar
of numbers may appear in either upper or lower case.
\begin{grammar}%
\meta{num $R$} \: \meta{prefix $R$} \meta{complex $R$}
\meta{complex $R$} \: %
         \meta{real $R$} %
      \| \meta{real $R$} @ \meta{real $R$}
   \> \| \meta{real $R$} + \meta{ureal $R$} i %
      \| \meta{real $R$} - \meta{ureal $R$} i
   \> \| \meta{real $R$} + i %
      \| \meta{real $R$} - i %
      \| \meta{real $R$} \meta{infnan} i 
   \> \| + \meta{ureal $R$} i %
      \| - \meta{ureal $R$} i
   \> \| \meta{infnan} i %
      \| + i %
      \| - i
\meta{real $R$} \: \meta{sign} \meta{ureal $R$}
   \> \| \meta{infnan}
\meta{ureal $R$} \: %
         \meta{uinteger $R$}
   \> \| \meta{uinteger $R$} / \meta{uinteger $R$}
   \> \| \meta{decimal $R$}
\meta{decimal $10$} \: %
         \meta{uinteger $10$} \meta{suffix}
   \> \| . \atleastone{\meta{digit $10$}} \meta{suffix}
   \> \| \atleastone{\meta{digit $10$}} . \arbno{\meta{digit $10$}} \meta{suffix}
\meta{uinteger $R$} \: \atleastone{\meta{digit $R$}}
\meta{prefix $R$} \: %
         \meta{radix $R$} \meta{exactness}
   \> \| \meta{exactness} \meta{radix $R$}
\meta{infnan} \: +inf.0 \| -inf.0 \| +nan.0 \| -nan.0
\end{grammar}

\begin{grammar}%
\meta{suffix} \: \meta{empty} 
   \> \| \meta{exponent marker} \meta{sign} \atleastone{\meta{digit $10$}}
\meta{exponent marker} \: e \| s \| f \| d \| l
\meta{sign} \: \meta{empty}  \| + \|  -
\meta{exactness} \: \meta{empty} \| \#i\sharpindex{i} \| \#e\sharpindex{e}
\meta{radix 2} \: \#b\sharpindex{b}
\meta{radix 8} \: \#o\sharpindex{o}
\meta{radix 10} \: \meta{empty} \| \#d
\meta{radix 16} \: \#x\sharpindex{x}
\meta{digit 2} \: 0 \| 1
\meta{digit 8} \: 0 \| 1 \| 2 \| 3 \| 4 \| 5 \| 6 \| 7
\meta{digit 10} \: \meta{digit}
\meta{digit 16} \: \meta{digit $10$} \| a \| b \| c \| d \| e \| f %
\end{grammar}


\subsection{External representations}
\label{datumsyntax}

\meta{Datum} is what the \ide{read} procedure (section~\ref{read})
successfully parses.  Note that any string that parses as an
\meta{ex\-pres\-sion} will also parse as a \meta{datum}.  \label{datum}

\begin{grammar}%
\meta{datum} \: \meta{simple datum} \| \meta{compound datum}
\>  \| \meta{label} = \meta{datum} \| \meta{label} \#
\meta{simple datum} \: \meta{boolean} \| \meta{number}
\>  \| \meta{character} \| \meta{string} \|  \meta{symbol} \| \meta{bytevector}
\meta{symbol} \: \meta{identifier}
\meta{compound datum} \: \meta{list} \| \meta{vector} \| \meta{abbreviation}
\meta{list} \: (\arbno{\meta{datum}}) \| (\atleastone{\meta{datum}} .\ \meta{datum})
\meta{abbreviation} \: \meta{abbrev prefix} \meta{datum}
\meta{abbrev prefix} \: ' \| ` \| , \| ,@
\meta{vector} \: \#(\arbno{\meta{datum}})
\meta{label} \: \# \atleastone{\meta{digit $10$}} %
\end{grammar}


\subsection{Expressions}

The definitions in this and the following subsections assume that all
the syntax keywords defined in this report have been properly imported
from their libraries, and that none of them have been redefined or shadowed.

\begin{grammar}%
\meta{expression} \: \meta{identifier}
\>  \| \meta{literal}
\>  \| \meta{procedure call}
\>  \| \meta{lambda expression}
\>  \| \meta{conditional}
\>  \| \meta{assignment}
\>  \| \meta{derived expression}
\>  \| \meta{macro use}
\>  \| \meta{macro block}
\>  \| \meta{includer}

\meta{literal} \: \meta{quotation} \| \meta{self-evaluating}
\meta{self-evaluating} \: \meta{boolean} \| \meta{number} \| \meta{vector}
\>  \| \meta{character} \| \meta{string} \| \meta{bytevector}
\meta{quotation} \: '\meta{datum} \| (quote \meta{datum})
\meta{procedure call} \: (\meta{operator} \arbno{\meta{operand}})
\meta{operator} \: \meta{expression}
\meta{operand} \: \meta{expression}

\meta{lambda expression} \: (lambda \meta{formals} \meta{body})
\meta{formals} \: (\arbno{\meta{identifier}}) \| \meta{identifier}
\>  \| (\atleastone{\meta{identifier}} .\ \meta{identifier})
\meta{body} \:  \arbno{\meta{definition}} \meta{sequence}
\meta{sequence} \: \arbno{\meta{command}} \meta{expression}
\meta{command} \: \meta{expression}

\meta{conditional} \: (if \meta{test} \meta{consequent} \meta{alternate})
\meta{test} \: \meta{expression}
\meta{consequent} \: \meta{expression}
\meta{alternate} \: \meta{expression} \| \meta{empty}

\meta{assignment} \: (set! \meta{identifier} \meta{expression})

\meta{derived expression} \:
\>  \> (cond \atleastone{\meta{cond clause}})
\>  \| (cond \arbno{\meta{cond clause}} (else \meta{sequence}))
\>  \| (c\=ase \meta{expression}
\>       \>\atleastone{\meta{case clause}})
\>  \| (c\=ase \meta{expression}
\>       \>\arbno{\meta{case clause}}
\>       \>(else \meta{sequence}))
\>  \| (c\=ase \meta{expression}
\>       \>\arbno{\meta{case clause}}
\>       \>(else => \meta{recipient}))
\>  \| (and \arbno{\meta{test}})
\>  \| (or \arbno{\meta{test}})
\>  \| (when \meta{test} \meta{sequence})
\>  \| (unless \meta{test} \meta{sequence})
\>  \| (let (\arbno{\meta{binding spec}}) \meta{body})
\>  \| (let \meta{identifier} (\arbno{\meta{binding spec}}) \meta{body})
\>  \| (let* (\arbno{\meta{binding spec}}) \meta{body})
\>  \| (letrec (\arbno{\meta{binding spec}}) \meta{body})
\>  \| (letrec* (\arbno{\meta{binding spec}}) \meta{body})
\>  \| (let-values (\arbno{\meta{mv binding spec}}) \meta{body})
\>  \| (let*-values (\arbno{\meta{mv binding spec}}) \meta{body})
\>  \| (begin \meta{sequence})
\>  \| (d\=o \=(\arbno{\meta{iteration spec}})
\>       \>  \>(\meta{test} \meta{do result})
\>       \>\arbno{\meta{command}})
\>  \| (delay \meta{expression})
\>  \| (delay-force \meta{expression})
\>  \| (p\=arameterize (\arbno{(\meta{expression} \meta{expression})})
\>       \> \meta{body})
\>  \| (guard (\meta{identifier} \arbno{\meta{cond clause}}) \meta{body})
\>  \| \meta{quasiquotation}
\>  \| (c\=ase-lambda \arbno{\meta{case-lambda clause}})

\meta{cond clause} \: (\meta{test} \meta{sequence})
\>   \| (\meta{test})
\>   \| (\meta{test} => \meta{recipient})
\meta{recipient} \: \meta{expression}
\meta{case clause} \: ((\arbno{\meta{datum}}) \meta{sequence})
\>   \| ((\arbno{\meta{datum}}) => \meta{recipient})
\meta{binding spec} \: (\meta{identifier} \meta{expression})
\meta{mv binding spec} \: (\meta{formals} \meta{expression})
\meta{iteration spec} \: (\meta{identifier} \meta{init} \meta{step})
\> \| (\meta{identifier} \meta{init})
\meta{case-lambda clause} \: (\meta{formals} \meta{body})
\meta{init} \: \meta{expression}
\meta{step} \: \meta{expression}
\meta{do result} \: \meta{sequence} \| \meta{empty}

\meta{macro use} \: (\meta{keyword} \arbno{\meta{datum}})
\meta{keyword} \: \meta{identifier}

\meta{macro block} \:
\>  (let-syntax (\arbno{\meta{syntax spec}}) \meta{body})
\>  \| (letrec-syntax (\arbno{\meta{syntax spec}}) \meta{body})
\meta{syntax spec} \: (\meta{keyword} \meta{transformer spec})

\meta{includer} \:
\> \| (include \atleastone{\meta{string}})
\> \| (include-ci \atleastone{\meta{string}})
\end{grammar}

\subsection{Quasiquotations}

The following grammar for quasiquote expressions is not context-free.
It is presented as a recipe for generating an infinite number of
production rules.  Imagine a copy of the following rules for $D = 1, 2,
3, \ldots$, where $D$ is the nesting depth.

\begin{grammar}%
\meta{quasiquotation} \: \meta{quasiquotation 1}
\meta{qq template 0} \: \meta{expression}
\meta{quasiquotation $D$} \: `\meta{qq template $D$}
\>    \| (quasiquote \meta{qq template $D$})
\meta{qq template $D$} \: \meta{simple datum}
\>    \| \meta{list qq template $D$}
\>    \| \meta{vector qq template $D$}
\>    \| \meta{unquotation $D$}
\meta{list qq template $D$} \: (\arbno{\meta{qq template or splice $D$}})
\>    \| (\atleastone{\meta{qq template or splice $D$}} .\ \meta{qq template $D$})
\>    \| '\meta{qq template $D$}
\>    \| \meta{quasiquotation $D+1$}
\meta{vector qq template $D$} \: \#(\arbno{\meta{qq template or splice $D$}})
\meta{unquotation $D$} \: ,\meta{qq template $D-1$}
\>    \| (unquote \meta{qq template $D-1$})
\meta{qq template or splice $D$} \: \meta{qq template $D$}
\>    \| \meta{splicing unquotation $D$}
\meta{splicing unquotation $D$} \: ,@\meta{qq template $D-1$}
\>    \| (unquote-splicing \meta{qq template $D-1$}) %
\end{grammar}

In \meta{quasiquotation}s, a \meta{list qq template $D$} can sometimes
be confused with either an \meta{un\-quota\-tion $D$} or a \meta{splicing
un\-quo\-ta\-tion $D$}.  The interpretation as an
\meta{un\-quo\-ta\-tion} or \meta{splicing
un\-quo\-ta\-tion $D$} takes precedence.

\subsection{Transformers}

\begin{grammar}%
\meta{transformer spec} \:
\> (syntax-rules (\arbno{\meta{identifier}}) \arbno{\meta{syntax rule}})
\> \| (syntax-rules \meta{identifier} (\arbno{\meta{identifier}})
\> \> \ \ \arbno{\meta{syntax rule}})
\meta{syntax rule} \: (\meta{pattern} \meta{template})
\meta{pattern} \: \meta{pattern identifier}
\>  \| \meta{underscore}
\>  \| (\arbno{\meta{pattern}})
\>  \| (\atleastone{\meta{pattern}} . \meta{pattern})
\>  \| (\arbno{\meta{pattern}} \meta{pattern} \meta{ellipsis} \arbno{\meta{pattern}})
\>  \| (\arbno{\meta{pattern}} \meta{pattern} \meta{ellipsis} \arbno{\meta{pattern}}
\> \> \ \ . \meta{pattern})
\>  \| \#(\arbno{\meta{pattern}})
\>  \| \#(\arbno{\meta{pattern}} \meta{pattern} \meta{ellipsis} \arbno{\meta{pattern}})
\>  \| \meta{pattern datum}
\meta{pattern datum} \: \meta{string}
\>  \| \meta{character}
\>  \| \meta{boolean}
\>  \| \meta{number}
\meta{template} \: \meta{pattern identifier}
\>  \| (\arbno{\meta{template element}})
\>  \| (\atleastone{\meta{template element}} .\ \meta{template})
\>  \| \#(\arbno{\meta{template element}})
\>  \| \meta{template datum}
\meta{template element} \: \meta{template}
\>  \| \meta{template} \meta{ellipsis}
\meta{template datum} \: \meta{pattern datum}
\meta{pattern identifier} \: \meta{any identifier except {\cf ...}}
\meta{ellipsis} \: \meta{an identifier defaulting to {\cf ...}}
\meta{underscore} \: \meta{the identifier {\cf \_}}
\end{grammar}

\subsection{Programs and definitions}

\begin{grammar}%
\meta{program} \:
\> \atleastone{\meta{import declaration}}
\> \atleastone{\meta{command or definition}}
\meta{command or definition} \: \meta{command}
\> \| \meta{definition}
\> \| (begin \atleastone{\meta{command or definition}})
\meta{definition} \: (define \meta{identifier} \meta{expression})
\>   \| (define (\meta{identifier} \meta{def formals}) \meta{body})
\>   \| \meta{syntax definition}
\>   \| (define-values \meta{def formals} \meta{body})
\>   \| (define-record-type \meta{identifier}
\> \> \ \ \meta{constructor} \meta{identifier} \arbno{\meta{field spec}})
\>   \| (begin \arbno{\meta{definition}})
\meta{def formals} \: \arbno{\meta{identifier}}
\>   \| \arbno{\meta{identifier}} .\ \meta{identifier}
\meta{constructor} \: (\meta{identifier} \arbno{\meta{field name}})
\meta{field spec} \: (\meta{field name} \meta{accessor})
\>   \| (\meta{field name} \meta{accessor} \meta{mutator})
\meta{field name} \: \meta{identifier}
\meta{accessor} \: \meta{identifier}
\meta{mutator} \: \meta{identifier}
\meta{syntax definition} \:
\>  (define-syntax \meta{keyword} \meta{transformer spec})
\end{grammar}

\subsection{Libraries}

\begin{grammar}%
\meta{library} \:
\> (d\=efine-library \meta{library name}
\>   \> \arbno{\meta{library declaration}})
\meta{library name} \: (\atleastone{\meta{library name part}})
\meta{library name part} \: \meta{identifier} \| \meta{uinteger 10}
\meta{library declaration} \: (export \arbno{\meta{export spec}})
\> \| \meta{import declaration}
\> \| (begin \arbno{\meta{command or definition}})
\> \| \meta{includer}
\> \| (include-library-declarations \atleastone{\meta{string}})
\> \| (cond-expand \atleastone{\meta{cond-expand clause}})
\> \| (cond-expand \atleastone{\meta{cond-expand clause}}
\hbox to 1\wd0{\hfill}\ (else \arbno{\meta{library declaration}}))
\meta{import declaration} \: (import \atleastone{\meta{import set}})
\meta{export spec} \: \meta{identifier}
\> \| (rename \meta{identifier} \meta{identifier})
\meta{import set} \: \meta{library name}
\> \| (only \meta{import set} \atleastone{\meta{identifier}})
\> \| (except \meta{import set} \atleastone{\meta{identifier}})
\> \| (prefix \meta{import set} \meta{identifier})
\> \| (rename \meta{import set} \atleastone{(\meta{identifier} \meta{identifier})})
\meta{cond-expand clause} \:
\> (\meta{feature requirement} \arbno{\meta{library declaration}})
\meta{feature requirement} \: \meta{identifier}
\> \| \meta{library name}
\> \| (and \arbno{\meta{feature requirement}})
\> \| (or \arbno{\meta{feature requirement}})
\> \| (not \meta{feature requirement})
\end{grammar}
