\chapter{Program structure}
\label{programchapter}

\section{Programs}

A Scheme program consists of  
one or more import declarations followed by a sequence of
expressions and generalized definitions.
Import declarations specify the libraries on which a program or library depends;
a subset of the identifiers exported by the libraries are made available to
the program.
Expressions are described in chapter~\ref{expressionchapter}.
Generalized definitions are either definitions, syntax definitions, or
record type definitions, all of which are explained in this chapter.
They are valid in some, but not all, contexts where expressions
are allowed, specifically at the top level of a \hyper{program}
and at the beginning of a \hyper{body}.
\mainindex{generalized definition}

At the top level of a program {\tt(begin \hyperi{expr-or-def} \dotsfoo)} is
equivalent to the sequence of expressions and generalized definitions
in the \ide{begin}.   
Similarly, in a \hyper{body}, {\tt(begin \hyperi{generalized definition} \dotsfoo)} is equivalent
to the sequence \hyperi{generalized definition} \dotsfoo.
Macros can expand into such {\cf begin} forms.

Import declarations and generalized definitions can be interpreted declaratively.
They cause bindings to be created in the top level
environment or modify the value of existing top-level bindings.
The initial environment of a program is empty,
so at least one import declaration is required to introduce initial bindings.

Expressions occurring at the top level of a program are
executed in order when the program is
invoked or loaded, and typically perform some kind of initialization.


Programs as well as libraries are typically stored in files, although
in some implementations they can be entered interactively to a running
Scheme system.  Other paradigms are possible.
Implementations which store libraries in files should document the
mapping from the name of a library to its location in the file system.

\section{Import declarations}

An import declaration takes the following form:
\begin{scheme}
(import \hyper{import-set} \dotsfoo)
\end{scheme}

An \ide{import} declaration provides a way to import identifiers
exported by a library.  Each \hyper{import set} names a set of bindings
from a library and possibly specifies local names for the
imported bindings. It takes one of the following forms:

\begin{itemize}
\item {\tt\hyper{library name}}
\item {\tt(only \hyper{import set} \hyper{identifier} \dotsfoo)}
\item {\tt(except \hyper{import set} \hyper{identifier} \dotsfoo)}
\item {\tt(prefix \hyper{import set} \hyper{identifier})}
\item {\tt(rename \hyperi{import set} (\hyperii{$identifier$} \hyper{$identifier$}) \dotsfoo)}
\end{itemize}

In the first form, all of the identifiers in the named library's export
clauses are imported with the same names (or the exported names if
exported with \ide{rename}).  The additional \hyper{import set}
forms modify this set as follows:

\begin{itemize}

\item \ide{only} produces a subset of the given
  \hyper{import set}, including only the listed identifiers (after any
  renaming).  It is an error if any of the listed identifiers are
  not found in the original set.

\item \ide{except} produces a subset of the given
  \hyper{import set}, excluding the listed identifiers (after any
  renaming). It is an error if any of the listed identifiers are not
  found in the original set.

\item \ide{rename} modifies the given \hyper{import set},
  replacing each instance of \hyperi{$identifier$} with
  \hyperii{$identifier$}. It is an error if any of the listed
  identifiers are not found in the original set.

\item \ide{prefix} automatically renames all identifiers in
  the given \hyper{import set}, prefixing each with the specified
  \hyper{identifier}.

\end{itemize}

In a program or library declaration, it is an error to import the same
identifier more than once with different bindings, or to redefine or
mutate an imported binding with {\cf define}, {\cf define-syntax}
or {\cf set!}, or to refer to an identifier before it is imported.
However, a REPL should permit these actions.

\section{Definitions}
\label{defines}
\mainindex{definition}

A definition binds one or more identifiers and specifies an initial value for them.
It takes one of the following forms:\mainschindex{define}

\begin{itemize}

\item{\tt(define \hyper{variable} \hyper{expression})}

\item{\tt(define (\hyper{variable} \hyper{formals}) \hyper{body})}

\hyper{Formals} should be either a
sequence of zero or more variables, or a sequence of one or more
variables followed by a space-delimited period and another variable (as
in a lambda expression).  This form is equivalent to
\begin{scheme}
(define \hyper{variable}
  (lambda (\hyper{formals}) \hyper{body}))\rm.%
\end{scheme}

\item{\tt(define (\hyper{variable} .\ \hyper{formal}) \hyper{body})}

\hyper{Formal} should be a single
variable.  This form is equivalent to
\begin{scheme}
(define \hyper{variable}
  (lambda \hyper{formal} \hyper{body}))\rm.%
\end{scheme}

\end{itemize}

\subsection{Top level definitions}

At the top level of a program, a definition
\begin{scheme}
(define \hyper{variable} \hyper{expression})%
\end{scheme}
has essentially the same effect as the assignment expression
\begin{scheme}
(\ide{set!}\ \hyper{variable} \hyper{expression})%
\end{scheme}
if \hyper{variable} is bound to a non-syntax value.  However, if
\hyper{variable} is not bound, 
or is bound to a {\em syntax definition} (see below),
then the definition will bind
\hyper{variable} to a new location before performing the assignment,
whereas it would be an error to perform a {\cf set!}\ on an
unbound\index{unbound} variable.

\begin{scheme}
(define add3
  (lambda (x) (+ x 3)))
(add3 3)                            \ev  6
(define first car)
(first '(1 2))                      \ev  1%
\end{scheme}

Implementations are permitted to provide an initial environment in
which all possible variables are bound to locations, most of
which contain unspecified values.  Top level definitions in
such an implementation are truly equivalent to assignments,
unless the identifier is defined as a syntax keyword.



\subsection{Internal definitions}
\label{internaldefines}

Definitions may occur at the
beginning of a \hyper{body} (that is, the body of a \ide{lambda},
\ide{let}, \ide{let*}, \ide{letrec}, \ide{letrec*},
\ide{let-values}, \ide{let-values*}, \ide{let-syntax}, \ide{letrec-syntax},
\ide{parameterize}, \ide{guard}, or \ide{case-lambda}
expression or that of a definition of an appropriate form).
Such definitions are known as {\em internal definitions} \mainindex{internal
definition} as opposed to the top-level definitions described above.
The variable defined by an internal definition is local to the
\hyper{body}.  That is, \hyper{variable} is bound rather than assigned,
and the region of the binding is the entire \hyper{body}.  For example,

\begin{scheme}
(let ((x 5))
  (define foo (lambda (y) (bar x y)))
  (define bar (lambda (a b) (+ (* a b) a)))
  (foo (+ x 3)))                \ev  45%
\end{scheme}

An expanded \hyper{body} containing internal definitions
(but not syntax definitions or record definitions) can always be
converted into a completely equivalent {\cf letrec*} expression.  For
example, the {\cf let} expression in the above example is equivalent
to

\begin{scheme}
(let ((x 5))
  (letrec* ((foo (lambda (y) (bar x y)))
            (bar (lambda (a b) (+ (* a b) a))))
    (foo (+ x 3))))%
\end{scheme}

Just as for the equivalent {\cf letrec*} expression, it is an error if it is not
possible to evaluate each \hyper{expression} of every internal
definition in a \hyper{body} without assigning or referring to
the value of the corresponding \hyper{variable} or the \hyper{variable}
of any of the definitions that follow it in \hyper{body}.

It is an error to define the same identifier more than once in the
same \hyper{body}.

Wherever an internal definition may occur,
{\tt(begin \hyperi{definition} \dotsfoo)}
is equivalent to the sequence of definitions
that form the body of the \ide{begin}.

\subsection{Multiple-value definitions}

The construct {\cf define-values} introduces new definitions like
{\cf define}, but can create multiple definitions from a single
expression returning multiple values.
It is allowed wherever {\cf define} is allowed.

\begin{entry}{%
\proto{define-values}{ \hyper{formals} \hyper{expression}}{\exprtype}}\nobreak

It is an error if a variable appears more than once in the set of \hyper{formals}.

\semantics
\hyper{Expression} is evaluated, and the \hyper{formals} are bound
to the return values in the same way that the \hyper{formals} in a
{\cf lambda} expression are matched to the arguments in a procedure
call.

\begin{scheme}
(let ()
  (define-values (x y) (values 1 2))
  (+ x y))     \ev 3%
\end{scheme}

\end{entry}

\section{Syntax definitions}

\mainindex{syntax definition}
Syntax definitions have the following form:\mainschindex{define-syntax}

{\tt(define-syntax \hyper{keyword} \hyper{transformer spec})}

\hyper{Keyword} is an identifier, and
the \hyper{transformer spec} should be an instance of \ide{syntax-rules}.
If the {\cf define-syntax} occurs at the top level, then the top-level
syntactic environment is extended by binding the
\hyper{keyword} to the specified transformer, but existing references
to any top-level binding for \hyper{keyword} remain unchanged.
This also means that if a syntax keyword is used before
it is defined, it is an error; any definition from an outer scope will not
be applied.
Otherwise, it is an {\em internal syntax definition}, and is local to the
\hyper{body} in which it is defined.

\begin{scheme}
(let ((x 1) (y 2))
  (define-syntax swap!
    (syntax-rules ()
      ((swap! a b)
       (let ((tmp a))
         (set! a b)
         (set! b tmp)))))
  (swap! x y)
  (list x y))                \ev (2 1)%
\end{scheme}

\todo{Shinn: This description is hideous.
Cowan: But now less hideous than before.}

Macros can expand into definitions in any context that permits
them. However, it is an error for a definition to define an
identifier whose binding has to be known in order to determine the meaning of the
definition itself, or of any preceding definition that belongs to the
same group of internal definitions. Similarly, it is an error for an
internal definition to define an identifier whose binding has to be known
in order
to determine the boundary between the internal definitions and the
expressions of the body it belongs to. For example, the following are
errors:

\begin{scheme}
(define define 3)

(begin (define begin list))

(let-syntax
    ((foo (syntax-rules ()
            ((foo (proc args ...) body ...)
             (define proc
               (lambda (args ...)
                 body ...))))))
  (let ((x 3))
    (foo (plus x y) (+ x y))
    (define foo x)
    (plus foo x)))%
\end{scheme}

\todo{Shinn: Example using internal define-syntax.}

\section{Record type definitions}
\label{usertypes}

\defining{Record type definitions} are used to introduce new data types, called
\defining{record types}.
The values of a record type are called \defining{records} and are
aggregations of zero or more \defining{fields}, each of which holds a single location.
A predicate, a constructor, and field accessors and
mutators are defined for each record type.

\begin{entry}{%
\mainschindex{define-record-type}
\pproto{(define-record-type \hyper{name}}{syntax}
\hspace*{4em}{\tt \hyper{constructor} \hyper{pred} \hyper{field} \dotsfoo})}

\syntax
\hyper{name} and \hyper{pred} should be identifiers.
The \hyper{constructor} should be of the form
\begin{scheme}
(\hyper{constructor name} \hyper{field name} \dotsfoo)%
\end{scheme}
and each \hyper{field} should be either of the form
\begin{scheme}
(\hyper{field name} \hyper{accessor name})%
\end{scheme}
or of the form
\begin{scheme}
(\hyper{field name} \hyper{accessor name} \hyper{modifier name})%
\end{scheme}

It is an error for the same identifier to occur more than once as a
field name.

{\cf define-record-type} is generative: each use creates a new record
type that is distinct from all existing types, including Scheme's
predefined types and other record types --- even record types of
the same name or structure.

An instance of {\cf define-record-type} is equivalent to the following
definitions:

\begin{itemize}

\item \hyper{name} is bound to a representation of the record type itself.
This may be a run-time object or a purely syntactic representation.
The representation is not utilized in this report, but it serves as a
means to identify the record type for use by further language extensions.

\item \hyper{constructor name} is bound to a procedure that takes as
  many arguments as there are \hyper{field name}s in the
  \texttt{(\hyper{constructor name} \dotsfoo)} subexpression and returns a
  new record of type \hyper{name}.  Fields whose names are listed with
  \hyper{constructor name} have the corresponding argument as their
  initial value.  The initial values of all other fields are
  unspecified.  It is an error for a field name to appear in
  \hyper{constructor} but not as a \hyper{field name}.

\item \hyper{pred} is bound to a predicate that returns \schtrue{} when given a
  value returned by the procedure bound to  \hyper{constructor name} and \schfalse{} for
  everything else.

\item Each \hyper{accessor name} is bound to a procedure that takes a record of
  type \hyper{name} and returns the current value of the corresponding
  field.  It is an error to pass an accessor a value which is not a
  record of the appropriate type.

\item Each \hyper{modifier name} is bound to a procedure that takes a record of
  type \hyper{name} and a value which becomes the new value of the
  corresponding field; an unspecified value is returned.  It is an
  error to pass a modifier a first argument which is not a record of
  the appropriate type.

\end{itemize}

For instance, the following definition

\begin{scheme}
(define-record-type <pare>
  (kons x y)
  pare?
  (x kar set-kar!)
  (y kdr))
\end{scheme}

defines {\cf kons} to be a constructor, {\cf kar} and {\cf kdr}
to be accessors, {\cf set-kar!} to be a modifier, and {\cf pare?}
to be a predicate for instances of {\cf <pare>}.

\begin{scheme}
  (pare? (kons 1 2))        \ev \schtrue
  (pare? (cons 1 2))        \ev \schfalse
  (kar (kons 1 2))          \ev 1
  (kdr (kons 1 2))          \ev 2
  (let ((k (kons 1 2)))
    (set-kar! k 3)
    (kar k))                \ev 3
\end{scheme}

\end{entry}


\section{Libraries}
\label{libraries}

Libraries provide a way to organize Scheme programs into reusable parts
with explicitly defined interfaces to the rest of the program.  This
section defines the notation and semantics for libraries.


\subsection{Library Syntax}

A library definition takes the following form:
\mainschindex{define-library}

\begin{scheme}
(define-library \hyper{library name}
  \hyper{library declaration} \dotsfoo)
\end{scheme}

\hyper{library name} is a list whose members are identifiers and exact nonnegative integers.  It is used to
identify the library uniquely when importing from other programs or
libraries.
Libraries whose first identifier is {\cf scheme} are reserved for use by this
report and future versions of this report.
Libraries whose first identifier is {\cf srfi} are reserved for libraries
implementing Scheme Requests for Implementation.
It is inadvisable, but not an error, for identifiers in library names to
contain any of the characters {\cf | \backwhack{} ? * < " : > + [ ] /}
or control characters after escapes are expanded.

A \hyper{library declaration} may be any of:

\begin{itemize}

\item{\tt(export \hyper{export spec} \dotsfoo)}

\item{\tt(import \hyper{import set} \dotsfoo)}

\item{\tt(begin \hyper{command or definition} \dotsfoo)}

\item{\tt(include \hyperi{filename} \hyperii{filename} \dotsfoo)}

\item{\tt(include-ci \hyperi{filename} \hyperii{filename} \dotsfoo)}

\item{\tt(include-library-declarations \hyperi{filename} \hyperii{filename} \dotsfoo)}

\item{\tt(cond-expand \hyper{cond-expand clause} \dotsfoo)}

\end{itemize}

An \ide{export} declaration specifies a list of identifiers which
can be made visible to other libraries or programs.
An \hyper{export spec} takes one of the following forms:

\begin{itemize}
\item{\hyper{identifier}}
\item{\tt{(rename \hyperi{identifier} \hyperii{identifier})}}
\end{itemize}

In an \hyper{export spec}, an \hyper{identifier} names a single
binding defined within or imported into the library, where the
external name for the export is the same as the name of the binding
within the library. A \ide{rename} spec exports the binding 
defined within or imported into the library and named by
\hyperi{identifier} in each
{\tt(\hyperi{identifier} \hyperii{identifier})} pairing,
using \hyperii{identifier} as the external name.

An \ide{import} declaration provides a way to import the identifiers
exported by another library.  It has the same syntax and semantics
as an import declaration used in a program or at the REPL.

The \ide{begin}, \ide{include}, and \ide{include-ci} declarations are
used to specify the body of
the library.  They have the same syntax and semantics as the corresponding
expression types, except that
\ide{begin} may contain any library declarations rather than
just expressions.

The \ide{include-library-declarations} declaration is similar to
\ide{include} except that the contents of the file are spliced directly into the
current library definition.  This can be used, for example, to share the
same \ide{export} declaration among multiple libraries as a simple
form of library interface.

The \ide{cond-expand} declaration has the same syntax and semantics as
the \ide{cond-expand} expression type, except that it may contain any
library declarations rather than just expressions.

\todo{Shinn: Perhaps make this a separate subsection describing a
  library ``resolution'' phase which runs prior to library expansion.}

One possible implementation of libraries is as follows:
After all \ide{cond-expand} library declarations are expanded, a new
environment is constructed for the library consisting of all
imported bindings.  The expressions and
generalized declarations from all \ide{begin}, \ide{include} and \ide{include-ci}
library declarations are expanded in that environment in the order in which
they occur in the library.
Alternatively, \ide{cond-expand} and \ide{import} declarations may be processed
in left to right order interspersed with the processing of expressions
and declarations, with the environment growing as imported bindings are
added to it by each \ide{import} declaration.

When a library is loaded, its top-level expressions are executed
in textual order.
If a library's definitions are referenced in the expanded form of a
program or library body, then that library must be loaded before the
expanded program or library body is evaluated. This rule applies
transitively.  If a library is imported by more than one program or
library, it may possibly be loaded additional times.

Similarly, during the expansion of a library, if a syntax keyword
imported from a library is needed to expand the library, then the
imported library must be visited before the expansion of the importing
library.

Regardless of the number of times that a library is loaded, each
program or library that imports bindings from a library will do so from a
single loading of that library, regardless of the number of import
declarations in which it appears.
That is, {\cf (import (only (foo) a))} followed by {\cf (import (only (foo) b))}
has the same effect as {\cf (import (only (foo) a b))}.

\subsection{Library example}
The following example shows
how a program may be divided into libraries plus a relatively small
main program~\cite{life}.
If the main program is entered into a REPL, it is not necessary to import
the base library.

\begin{scheme}
(define-library (example grid)
  (export make rows cols ref each
          (rename put! set!))
  (import (scheme base))
  (begin
    ;; Create an NxM grid.
    (define (make n m)
      (let ((grid (make-vector n)))
        (do ((i 0 (+ i 1)))
            ((= i n) grid)
          (let ((v (make-vector m \schfalse{})))
            (vector-set! grid i v)))))
    (define (rows grid)
      (vector-length grid))
    (define (cols grid)
      (vector-length (vector-ref grid 0)))
    ;; Return \sharpfalse{} if out of range.
    (define (ref grid n m)
      (and (< -1 n (rows grid))
           (< -1 m (cols grid))
           (vector-ref (vector-ref grid n) m)))
    (define (put! grid n m v)
      (vector-set! (vector-ref grid n) m v))
    (define (each grid proc)
      (do ((j 0 (+ j 1)))
          ((= j (rows grid)))
        (do ((k 0 (+ k 1)))
            ((= k (cols grid)))
          (proc j k (ref grid j k)))))))

(define-library (example life)
  (export life)
  (import (except (scheme base) set!)
          (scheme write)
          (example grid))
  (begin
    (define (life-count grid i j)
      (define (count i j)
        (if (ref grid i j) 1 0))
      (+ (count (- i 1) (- j 1))
         (count (- i 1) j)
         (count (- i 1) (+ j 1))
         (count i (- j 1))
         (count i (+ j 1))
         (count (+ i 1) (- j 1))
         (count (+ i 1) j)
         (count (+ i 1) (+ j 1))))
    (define (life-alive? grid i j)
      (case (life-count grid i j)
        ((3) \sharptrue{})
        ((2) (ref grid i j))
        (else \sharpfalse{})))
    (define (life-print grid)
      (display "\backwhack{}x1B;[1H\backwhack{}x1B;[J")  ; clear vt100
      (each grid
       (lambda (i j v)
         (display (if v "*" " "))
         (when (= j (- (cols grid) 1))
           (newline)))))
    (define (life grid iterations)
      (do ((i 0 (+ i 1))
           (grid0 grid grid1)
           (grid1 (make (rows grid) (cols grid))
                  grid0))
          ((= i iterations))
        (each grid0
         (lambda (j k v)
           (let ((a (life-alive? grid0 j k)))
             (set! grid1 j k a))))
        (life-print grid1)))))

;; Main program.
(import (scheme base)
        (only (example life) life)
        (rename (prefix (example grid) grid-)
                (grid-make make-grid)))

;; Initialize a grid with a glider.
(define grid (make-grid 24 24))
(grid-set! grid 1 1 \sharptrue{})
(grid-set! grid 2 2 \sharptrue{})
(grid-set! grid 3 0 \sharptrue{})
(grid-set! grid 3 1 \sharptrue{})
(grid-set! grid 3 2 \sharptrue{})

;; Run for 80 iterations.
(life grid 80)

\end{scheme}

\section{The REPL}

Implementations may provide an interactive session called a
\defining{REPL} (Read-Eval-Print Loop), where import declarations,
expressions, definitions, syntax definitions, and record type definitions can be
entered and evaluated one at a time.  For convenience and ease of use,
the ``top-level'' Scheme environment in a REPL
must not be not empty, but must start out with a number of variables
bound to locations containing at least the bindings provided by the
base library.  This library includes the core syntax of Scheme
and generally useful procedures that manipulate data.  For example, the
variable {\cf abs} is bound to a
procedure of one argument that computes the absolute value of a
number, and the variable {\cf +} is bound to a procedure that computes
sums.  The full list of {\cf(scheme base)} bindings can be found in
Appendix~\ref{stdlibraries}.

An implementation may provide a mode of operation in which the REPL
reads its input from a file, sometimes called a defining{script}.  A script is not, in general, the same
as a program, because it can contain import declarations in places other than
the beginning.

