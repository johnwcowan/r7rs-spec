\clearextrapart{Introduction}

\label{historysection}

Programming languages should be designed not by piling feature on top of
feature, but by removing the weaknesses and restrictions that make additional
features appear necessary.  Scheme demonstrates that a very small number
of rules for forming expressions, with no restrictions on how they are
composed, suffice to form a practical and efficient programming language
that is flexible enough to support most of the major programming
paradigms in use today.

Scheme
was one of the first programming languages to incorporate first class
procedures as in the lambda calculus, thereby proving the usefulness of
static scope rules and block structure in a dynamically typed language.
Scheme was the first major dialect of Lisp to distinguish procedures
from lambda expressions and symbols, to use a single lexical
environment for all variables, and to evaluate the operator position
of a procedure call in the same way as an operand position.  By relying
entirely on procedure calls to express iteration, Scheme emphasized the
fact that tail-recursive procedure calls are essentially goto's that
pass arguments.  Scheme was the first widely used programming language to
embrace first class escape procedures, from which all previously known
sequential control structures can be synthesized.  A subsequent
version of Scheme introduced the concept of exact and inexact numbers,
an extension of Common Lisp's generic arithmetic.
More recently, Scheme became the first programming language to support
hygienic macros, which permit the syntax of a block-structured language
to be extended in a consistent and reliable manner.
\todo{Ramsdell:
I would like to make a few comments on presentation.  The most
important comment is about section organization.  Newspaper writers
spend most of their time writing the first three paragraphs of any
article.  This part of the article is often the only part read by
readers, and is important in enticing readers to continue.  In the
same way, The first page is most likely to be the only page read by
many SIGPLAN readers.  If I had my choice of what I would ask them to
read, it would be the material in section 1.1, the Semantics section
that notes that scheme is lexically scoped, tail recursive, weakly
typed, ... etc.  I would expand on the discussion on continuations,
as they represent one important difference between Scheme and other
languages.  The introduction, with its history of scheme, its history
of scheme reports and meetings, and acknowledgements giving names of
people that the reader will not likely know, is not that one page I
would like all to read.  I suggest moving the history to the back of
the report, and use the first couple of pages to convince the reader
that the language documented in this report is worth studying.
}

\subsection*{Background}

\vest The first description of Scheme was written in
1975~\cite{Scheme75}.  A revised report~\cite{Scheme78}
appeared in 1978, which described the evolution
of the language as its MIT implementation was upgraded to support an
innovative compiler~\cite{Rabbit}.  Three distinct projects began in
1981 and 1982 to use variants of Scheme for courses at MIT, Yale, and
Indiana University~\cite{Rees82,MITScheme,Scheme311}.  An introductory
computer science textbook using Scheme was published in
1984~\cite{SICP}.

\vest As Scheme became more widespread,
local dialects began to diverge until students and researchers
occasionally found it difficult to understand code written at other
sites.
Fifteen representatives of the major implementations of Scheme therefore
met in October 1984 to work toward a better and more widely accepted
standard for Scheme.
Their report, the RRRS~\cite{RRRS},
was published at MIT and Indiana University in the summer of 1985.
Further revision took place in the spring of 1986, resulting in the
R$^{3}$RS~\cite{R3RS}.
Work in the spring of 1988 the resulted in R$^{4}$RS~\cite{R4RS},
which became the basis for the
IEEE Standard for the Scheme Programming Language in 1991~\cite{IEEEScheme}.
In 1998, several additions to the IEEE standard, including high-level
hygienic macros, multiple return values and {\cf eval}, were finalized
as the R$^{5}$RS~\cite{R5RS}.

\todo{Perhaps flesh out a little and mention the forming of the
 Steering Committee.}

In the Fall of 2006, work began on a more ambitious standard, including
many new improvements and a general change in style from descriptive,
reporting on the state of existing implementations, to prescriptive,
specifying how a conformant implementation should behave.  The
resulting standard, the R$^{6}$RS, was completed in August
2007~\cite{R6RS}, and was organized as a core language and set of
standard libraries.  The size and goals of the R$^{6}$RS, however,
were controversial, and in a poll taken many Scheme implementors
reported no intention of updating to the new standard.

In August 2009, the Scheme Steering Committee decided to divide the
standard into two separate but compatible languages --- a ``small''
language, suitable for educators, researchers and embedded languages,
focused on R$^{5}$RS compatibility, and a ``large'' language focused
on the practical needs of mainstream software development.
The present report describes the ``small'' language of that effort.



\medskip

We intend this report to belong to the entire Scheme community, and so
we grant permission to copy it in whole or in part without fee.  In
particular, we encourage implementors of Scheme to use this report as
a starting point for manuals and other documentation, modifying it as
necessary.




\subsection*{Acknowledgements}

We would like to thank the members of the Steering Committee, William
Clinger, Marc Feeley, Chris Hanson, Jonathan Rees and Olin Shivers for
their support and guidance.  We'd like to thank the following people
for their help:
Per Bothner, Taylor Campbell, Ray Dillinger, Brian Harvey, Shiro Kawai,
Jonathan Kraut, Thomas Lord, Vincent Manis, Jeronimo Pellegrini,
Jussi Piitulainen, Alex Queiroz, Jim Rees, Jay Reynolds Freeman,
Malcolm Tredinnick, Denis Washington, Andy Wingo, Andre van Tonder and
hopefully many before before this report is finalized!

In addition we would like to thank all the past editors, and the
people who helped them in turn: Alan Bawden, Michael
Blair, George Carrette, Andy Cromarty, Pavel Curtis, Jeff Dalton, Olivier Danvy,
Ken Dickey, Bruce Duba,
Andy Freeman, Richard Gabriel, Yekta G\"ursel, Ken Haase, Robert
Hieb, Paul Hudak, Morry Katz, Chris Lindblad, Mark Meyer, Jim Miller, Jim Philbin,
John Ramsdell, Mike Shaff, Jonathan Shapiro, Julie Sussman,
Perry Wagle, Daniel Weise, Henry Wu, and Ozan Yigit.
We thank Carol Fessenden, Daniel
Friedman, and Christopher Haynes for permission to use text from the Scheme 311
version 4 reference manual.  We thank Texas Instruments, Inc.~for permission to
use text from the {\em TI Scheme Language Reference Manual}\cite{TImanual85}.
We gladly acknowledge the influence of manuals for MIT Scheme\cite{MITScheme},
T\cite{Rees84}, Scheme 84\cite{Scheme84},Common Lisp\cite{CLtL},
and Algol 60\cite{Naur63}.

%% \vest We also thank Betty Dexter for the extreme effort she put into
%% setting this report in \TeX, and Donald Knuth for designing the program
%% that caused her troubles.

%% \vest The Artificial Intelligence Laboratory of the
%% Massachusetts Institute of Technology, the Computer Science
%% Department of Indiana University, the Computer and Information
%% Sciences Department of the University of Oregon, and the NEC Research
%% Institute supported the preparation of this report.  Support for the MIT
%% work was provided in part by
%% the Advanced Research Projects Agency of the Department of Defense under Office
%% of Naval Research contract N00014-80-C-0505.  Support for the Indiana
%% University work was provided by NSF grants NCS 83-04567 and NCS
%% 83-03325.
